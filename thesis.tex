% !TEX root = thesis.tex  % 将 "main.tex" 替换为你的主文件名
%%
%% This is file `thesis.tex',
%% 
%1. 如果是研究生论文,常用的选项是:
% \documentclass[master,twoside]{scnuthesis}
%2. 如果是博士生论文,常用的选项是:
% \documentclass[doctor,twoside]{scnuthesis}
%3. 如果是匿名评审,常用的选项是:
% \documentclass[anon,master]{scnuthesis}
%4. private选项用于将隐私信息替换为“隐藏”,不会显示空白,可供人参阅。
%5. title选项用于生成封面,默认是空白的。


\documentclass[anon,master,twoprof,private]{scnuthesis}
\usepackage{myscnu}
\usepackage{xeCJK}  % 中文支持
\usepackage{CJKnumb} 

\begin{document}
\raggedbottom
\graphicspath{{figures/}}
% % wzq 说明 修改时只需要修改以下的内容即可
\newcommand\hidetext{*隐去*}
\newcommand\enhidetext{*Hiden*}
\classification{}  % 分类号
\udc{}             % UDC号
\mastertype{专业}  % 硕士学位类型(只用于硕士论文)
\confidentiality{内部三年}  % 密级
\title{FRET双杂交分析数据处理软件的设计与开发}
\entitle{Design and Development of Data Processing Software for FRET Two-Hybrid Assay}
\displaytitle{FRET双杂交分析数据处理软件的设计与开发}
\serialno{2022023414}
\author{魏智强}
\enauthor{Zhiqiang Wei}
\subject{光电信息工程}
\ensubject{Optoelectronic Information Engineering}
\researchfield{光学显微成像及应用}
\enresearchfield{Optical Microimaging and Application}
\school{光电科学与工程学院}
\enschool{School of Optoelectronic Science and Engineering}
\supervisor{胡敏}
\ensupervisor{Min Hu}
\protitle{实验师}
\enprotitle{Engineer}
\secondsupervisor{陈同生}
\ensecondsupervisor{Tongsheng Chen}
\secondprotitle{教授}
\ensecondprotitle{Prof.}
\zhdate{二〇二五~年~五~月}
\endate{MAY. 2025}
% wzq 结束说明

% 插入摘要,制作封面
\maketitle % 中文封面
\enmaketitle % 英文封面
\normalsize % 恢复正文字体
\frontmatter
\begin{cabstract}

\ifshowtext
福斯特共振能量转移(Förster Resonance Energy Transfer, FRET)技术被广泛用于探究活细胞中生物大分子的动态相互作用过程,在研究生命科学基础问题、疾病的发生发展和药物研发等方面具有广阔的应用前景。
FRET双杂交分析是目前唯一可以在活细胞内进行生物大分子“滴定实验”的技术,能够定量获得供受体结合的化学计量比和相对亲和力。

FRET双杂交分析的数据处理过程繁琐费时,限制了其推广和应用。
数据处理过程需要借助Zeiss ZEN、HCImage、ImageJ、Excel和MATLAB等多个专业软件进行处理,数据需要在不同软件间进行导入和导出。
数据处理过程包括专家标注ROI(Region of Interest)、背景扣除、异常数据过滤、FRET效率计算、双杂交拟合计算等过程,共28个步骤,单次实验过程需要3.5至7小时。
FRET图像处理过程依赖人工手动标注ROI,无法满足大规模数据处理的需求,限制了FRET双杂交分析技术在高通量等大规模数据场景的应用。
为了解决现有FRET双杂交分析数据处理流程繁琐、人工操作依赖性强等问题,需要设计一款专门研发的数据处理软件,并开发对应的FRET图像处理算法,以实现FRET双杂交分析数据处理的规范化和自动化。

针对活细胞 FRET 双杂交分析数据处理的需求和问题,本文设计并开发了一款FRET双杂交分析数据处理软件 Fretha,支持从原始图像到化学计量比结果的规范化数据处理,并研发了基于明度和均匀度的自动ROI选取算法(LURS),实现了数据处理过程的自动化。
本文的主要工作和创新点如下:

(1)首次设计并实现了FRET双杂交分析数据处理软件Fretha。
该软件基于分层架构(表现层、业务层、数据访问层和数据层)构建,集成了成像参数设置、数据校验、自动ROI选取、FRET定量计算及双杂交拟合等核心功能。
通过封装FRET计算器、图像分析库和双杂交求解器,实现了E-FRET、$3^3$-FRET、L-FRET和DC-FRET等多模态分析算法。
软件采用Qt 5.15.2开发,并通过OpenCV和Dlib库实现图像处理与优化计算。
Fretha的设计体现了针对FRET双杂交分析数据处理的专业性和实用性,实现了从原始图像到化学计量比结果的全流程自动化处理,提高了数据处理的规范化和自动化水平。

(2)对Fretha软件进行了系统性测试与评估。
应用Fretha软件对标准质粒C4Y/C10Y/C40Y/C80Y进行手动$3^3$-FRET分析和E-FRET分析,测量得到的$E_{A}$值分别为$0.291\pm0.020$、$0.243\pm0.031$、$0.159\pm0.018$和$0.118\pm0.019$,$E_{D}$值分别为$0.307\pm0.040$、$0.230\pm0.022$、$0.155\pm0.011$和$0.117\pm0.012$,与文献报道的结果一致,误差小于5\%,$R_{C}$偏差不超过0.01。
手动处理的C32V和CVC质粒的模型质粒,通过Fretha测量得到的C32V中C与V的结合化学计量比为1.071,在CVC中则为2.023,均与文献值一致。
模块测试表明,软件支持多参数动态更新、异常数据隔离和结果可视化,数据处理精度与文献值高度吻合。
稳定性测试显示,软件在48小时连续运行和高频操作下保持稳定,资源占用无显著波动,适用于高通量药物筛选等场景。

(3)首次研发了基于明度和均匀度的自动ROI选取算法(LURS),实现了数据处理过程的自动化。
LURS算法通过多通道自适应阈值分割、变异系数均匀性评估和连通区域分析,能够识别荧光图像中高信噪比且低变异性的区域。
在标准质粒验证实验中,LURS方法测量的$E_{A}$与$E_{D}$值与文献报道值误差小于5\%,$R_{C}$偏差不超过0.05。
模型质粒C32V和CVC的FRET双杂交验证实验中,LURS测量得到的C32V中C与V的结合计量比为1.063,在CVC中则为1.900,与文献值误差不超过6\%。
应用LURS算法检测活MCF-7细胞中Bcl-xL-Bak相互作用的化学计量比,发现加药A1331852处理后,Bcl-xL-Bak之间结合的化学计量比由1.87降为1.12,与手动分析高度一致。
对比基于SAM-Med2D和ilastik,LURS表现出良好的稳定性和鲁棒性,适用于高通量药物筛选等大数据量场景。
LURS在1.4GB数据集上单ROI处理时间仅6.6 ms,内存占用800 MB,显著优于ilastik(35.2 ms / 1.8 GB)和SAM-Med2D(50.7 ms / 14 GB)。

Fretha 的推出为活细胞 FRET 双杂交分析提供了标准化、自动化的解决方案,有望推动该技术在精准医疗和药物研发中的大规模应用。
\fi
\end{cabstract}

\ckeywords{ \ifshowtext FRET;FRET定量分析;荧光图像;FRET双杂交分析;自动数据处理 \fi}

\begin{eabstract}

Förster resonance energy transfer (FRET) technology is widely used to explore the dynamic interactions of biomacromolecules in living cells, and has broad application prospects in basic life science research, disease development, and drug development.
FRET two-hybrid assay is currently the only method that can quantitatively obtain the stoichiometry and relative affinity of donor-acceptor binding through "titration experiments" in living cells.
However, the data processing process relies on multiple professional software and manual processing, which is difficult to meet the needs of large-scale data processing and limits the application of FRET two-hybrid assay technology.
To solve the problems of cumbersome existing FRET two-hybrid assay data processing process and strong dependence on manual operation, it is necessary to design a specially developed data processing software and develop corresponding FRET image processing algorithms to achieve standardized and automated FRET two-hybrid assay data processing.
To address the needs and problems of live cell FRET two-hybrid assay data processing, this thesis designs and develops a fully automated FRET two-hybrid assay data processing software Fretha, which achieves automated processing from raw images to stoichiometry results.
The main work and innovations of this thesis are as follows:

(1) The FRET two-hybrid assay data processing software Fretha is designed and implemented for the first time.
This software is built based on a layered architecture (presentation layer, business layer, data access layer, and data layer), integrating core functions such as imaging parameter setting, data verification, automatic ROI selection, FRET quantitative calculation, and two-hybrid fitting.
By encapsulating the FRET calculator, image analysis library, and two-hybrid solver, multi-modal analysis algorithms such as E-FRET, $3^3$-FRET, L-FRET, and DC-FRET are realized.
The software is developed using Qt 5.15.2 and implements image processing and optimization calculations through the OpenCV and Dlib libraries.
The design of Fretha reflects the professionalism and practicality of FRET two-hybrid assay data processing, achieving full-process automated processing from raw images to stoichiometry results, and improving the standardization and automation level of data processing.

(2) The automatic ROI selection algorithm (LURS) based on brightness and uniformity is developed for the first time, achieving the automation of the data processing process.
LURS algorithm can identify high signal-to-noise ratio and low variability regions in fluorescence images through multi-channel adaptive threshold segmentation, coefficient of variation uniformity evaluation, and connected region analysis.
In the E-FRET measurement of standard plasmids C4Y/C10Y/C40Y/C80Y, the $E_{A}$ and $E_{D}$ values have less than 5\% error compared to the literature, and the $R_{C}$ deviation does not exceed 0.05, consistent with the literature values.
In the FRET two-hybrid verification experiments of model plasmids C32V and CVC, the stoichiometry of C and V in C32V measured by LURS is 1.06, and in CVC it is 1.90, with an error of less than 6\% compared to the literature value.
LURS algorithm successfully detected that the stoichiometry of Bcl-xL-Bak binding in A1331852-treated MCF-7 cells decreased from 1.87 to 1.12, which is highly consistent with manual analysis.
Compared with algorithms based on SAM-Med2D and ilastik, LURS shows good stability and robustness, making it suitable for large data volume scenarios such as high-throughput drug screening.

(3) A systematic test and evaluation of the Fretha software is conducted, and functional testing shows that the software supports multi-parameter dynamic updates, abnormal data isolation, and result visualization, with high consistency in data processing accuracy with literature values. Performance comparison of algorithms shows that LURS has a single ROI processing time of only 6.6 ms and a memory usage of 800 MB on a 1.4GB dataset, significantly better than ilastik (35.2 ms / 1.8 GB) and SAM-Med2D (50.7 ms / 14 GB). Stability testing shows that the software remains stable under continuous operation for 48 hours and high-frequency operations, with no significant fluctuations in resource usage, making it suitable for high-throughput drug screening scenarios.
The launch of Fretha provides a standardized and automated solution for live cell FRET two-hybrid assay, which is expected to promote the large-scale application of this technology in precision medicine and drug development.

The launch of Fretha provides a standardized and automated solution for live cell FRET two-hybrid assay, which is expected to promote the large-scale application of this technology in precision medicine and drug development.

\end{eabstract}

\ekeywords{ \ifshowtext FRET, FRET quantitative analysis, FRET two-hybrid assay, Automated data processing \fi}



% 生成目录
\tableofcontents
% \newpage
% \thispagestyle{plain}
% \mbox{}
\listoftables           % 如果要生成表目录
% % \newpage
% % \thispagestyle{plain}
% % \mbox{}
\listoffigures          % 如果要生成图目录
% \newpage
% \thispagestyle{plain}
% \mbox{}
% \renewcommand{\chapterlabel}{\denotationname} %设置页眉

% \input{data/denotation} % 如果要生成符号列表

% 书写正文,可以根据需要增添章节。
\mainmatter
\chapter{绪论}

\section{福斯特共振能量转移(FRET)}

\subsection{FRET原理概述}
1948年,Förster首次阐述了福斯特共振能量转移(Förster Resonance Energy Transfer,FRET)理论\upcite{forster1948zwischenmolekulare}。
FRET是指处于激发态的供体分子(Donor)通过偶极子间相互作用将一部分能量以非辐射的形式转移给邻近处于基态的受体(Acceptor)分子(供受体之间的距离$R$在0至10 nm)\upcite{lakowicz2006principles}。
FRET 的发生会使得供体的荧光淬灭和受体的荧光增强,其中由于发生FRET而增强受体荧光称为敏化发射荧光。
FRET技术突破传统观测局限,精准解析分子间相互作用与距离\upcite{yung1973relationship}。
在细胞生物学微观环境里,FRET技术可被用作一种“分子尺”,可以检测纳米尺度上的生物分子动态,从而进一步研究其结构与功能。
因而,FRET技术在细胞生物学多领域,如细胞信号转导、蛋白质相互作用探测等广泛应用。
\begin{figure}[htbp]
    \centering
    \includegraphics[width=0.5\linewidth]{../figures/1/1_FRET过程示意图.png}
    \caption{FRET过程示意图}
    \label{fig:fret}
\end{figure}

理论和实验证明,当供受体荧光分子的距离为$r$时,它们之间的能量转移速率由下式给出\upcite{yekta1995dipole}:
\begin{equation}
    k_T(r)=\frac{1}{\tau_D}(\frac{R_0}{r})^6, \label{eq:1-1}
\end{equation}
其中$\tau_D$是供体荧光寿命,$R_0$为Förster半径,由下式表示:
\begin{equation}
    R_0^6=8.79\times{10^{-5}}(n^{-4}Q_DJ(\lambda)\kappa^2)(\text{in~} \mbox{\AA}^6),
\end{equation}
上式中,$n$表示介质的折射率,$Q_D$表示供体的量子产率,$J(\lambda)$是光谱重叠积分,$\kappa^2$是方向因子,表示供、受体偶极子的相对方向,一般在自由状态下取$\kappa^2=2/3$。FRET发生需要满足三个条件:(1)$r$的数值在$R_0$附近,约0-10nm;(2)供体的发射谱与受体的吸收谱有超过30\%的重叠;(3)供、受体偶极子方向不互相垂直。

FRET效率($E$)定义为供体转移给受体的能量与供体发射的总能量的比例,是用来衡量FRET程度的指标,其主要和分子间距和荧光团光谱的重叠度有关。光谱有部分重叠的供受体分子间距越小,能量转移就越高效。其计算式为:
\begin{equation}
    {E}=\frac{k_T(r)}{k_T(r)+\tau^{-1}_{D}},
\end{equation}
将公式\ref{eq:1-1}代入,可以看出$E$与$r$的六次方成反比的关系:
\begin{equation}
    E=\frac{R_0^6}{R_0^6+r^6}=\frac{1}{1+(r/R_0)^6}.
\end{equation}

\subsection{FRET在生物学中的应用}

FRET发生的条件是供、受体分子之间的距离$r$在0 - 10nm之间,这使得FRET技术成为一种“纳米尺”,能够有效地测量纳米级的分子间距。
FRET技术在研究生物蛋白相互作用、大分子构象变化、信号通路中的蛋白质调节机制等方面得到了广泛的应用\upcite{2020Single, wu2018translocation, shrestha2015understanding}。

FRET技术应用到各种生物研究中的重要前提是荧光蛋白的发展\upcite{chudakov2010fluorescent}。
1962 年,第一种绿色荧光蛋白(Green Fluorescent Protein, GFP)在维多利亚水母中被发现,由于荧光蛋白的无毒性以及稳定遗传性,可以利用基因转导技术将荧光蛋白接到感兴趣的蛋白质分子上,借助显微成像技术实时观察活细胞中目标蛋白的转位以及信号传递等生物问题\upcite{shimomura2009discovery}。
随着基因技术的发展,最先发现的GFP 蛋白被改造出了多种GFP突变体,多种荧光蛋白的出现为同时追踪多个蛋白质分子间的相互作用和多种蛋白质的共定位等复杂的生物问题提供了必要条件,使得基于FRET显微成像技术在分子生物学和以及生物物理学的活细胞在体研究得到了广泛的应用\upcite{sekar2003fluorescence}。

FRET定量分析方法的发展,如$3^3$-FRET(Three-cube-FRET)、E-FRET(Emission-FRET)和FRET双杂交分析等,帮助研究人员从独特的角度研究细胞内蛋白质等大分子的相互作用。
2016年,Ben-Johny等人利用FRET双杂交分析技术定量研究了钙离子通道与钙调蛋白之间的结合,发现当细胞内钙离子浓度较低时,一个钙调蛋白与一个钙离子通道结合;当细胞内钙离子浓度较高时,一个钙离子通道可以同时结合两个钙调蛋白\upcite{ben-johny2016}。
杨方方等人利用FRET双杂交实验,研究了Bcl-2家族的四种抗凋亡或促凋亡蛋白(即Bcl-xL、Bad、Bax和tBid)在健康细胞和凋亡细胞中的相互作用。\upcite{yang2020stoichiometry}。
李小梅等人使用双杂交FRET成像等技术对L型钙通道C末端编码肽进行了定量研究,证明了Calmudolin的多种变体均通过与钙调素竞争,抑制钙通道的基本功能\upcite{yang2022cytosolic}。

基于FRET技术在活细胞内实时监测钙离子的动态变化操作简单,灵敏度更高。
Cornea 等在2009利用荧光共振能量转移技术研究了钙调蛋白(CaM)与骨骼肌 Ryanodine 受体(RyR1)的结合模式\upcite{cornea2009fret}。
2020年Yoon等开发了一种基于 EGFP 和 Fusion Red 的新型 FRET 钙传感器(FRET-GF-PRed),并结合高频超声(HFU)技术实现了单细胞水平的精准刺激与成像\upcite{yoon2020fret}。
Yi等人开发红色电压指示剂 Cepheid1,利用 eFRET 技术实现单细胞电活动与钙信号同步监测,揭示了胰腺细胞电振荡与钙波动的时空关联\upcite{han2023bright}。

FRET还可用于检测酶的活性\upcite{carmona2009use, walter2021coumarin}。
酶支配着生物体内的新陈代谢、营养和能量转换等许多催化过程,影响着生物体内许多信号转导过程。
Cheppali等利用单分子三色 FRET 监测 NSF 酶解聚 SNARE 复合体的过程,揭示其解聚存在两条路径,澄清了酶作用的中间状态机制\upcite{cheppali2025single}。
樊林玮等通过 FRET 传感器发现机械牵张激活 cPLA2,抑制脂肪酸氧化酶活性,通过调控转录因子 YY1 影响血管平滑肌细胞功能\upcite{fan2025mechanical}。

在疾病诊断方面,FRET 技术在药物疗效评估等方面同样具有重要意义\upcite{song2011development, sun2021recent,双杂交药筛}。
2014年,Bozza等人通过设计特定的 FRET 生物传感器,能够直接反映癌症药物诱导癌细胞凋亡的效果,为药物的研发和筛选提供了有力的工具,避免了因无法区分细胞死亡和生长抑制而导致的结果偏差\upcite{bozza2014use}。

\subsection{基于受体敏化发射的FRET定量测量方法(E-FRET)}

受限于实验仪器和样本的准备,早期的FRET测量方式都只能基于定性或半定量测量\upcite{awais2004A,Aye2009Fluorescent}。
复杂实验条件的校正,需要使用表达特定荧光蛋白的细胞进行校正,而蛋白质在细胞中的表达又受到表达的成熟度、胞内的微环境等多种因素制约\upcite{liu2018influence}。

基于受体敏化发射的通道方法(E-FRET)具有测量速度快、无损伤的特性,被公认是最适合用于活细胞动态监测的FRET半定量检测技术\upcite{erickson2001preassociation}。
E-FRET方法需要对实验系统响应和荧光团的光学性质进行严格的校准,一般通过提前测量多个参照样本,然后保持在整个实验过程中系统的稳定和实验条件的一致。
E-FRET方法需要3个Cube组成的3个通道,分别实现供体激发供体收集(DD通道)、供体激发受体收集(DA通道)和受体激发受体收集(AA通道)。
受体的敏化荧光从DA通道测得($I_{DA}$图像),但实际上$I_{DA}$图像包含有供体转移到受体的敏化荧光、供体激发光直接激发受体的荧光和供体受激发的发射荧光这三个部分。
为了消除后两部分串扰,需要收集额外的DD通道和AA通道的荧光图像$I_{DD}$和 $I_{AA}$。通过减掉串扰得到敏化荧光$F_C$,由如下公式得到:
\begin{equation}
F_C=I_{DA}-a(I_{AA}-cI_{DD})-d(I_{DD}-bI_{AA}) ,
\label{eq:fc}
\end{equation}
其中$a, b, c, d$是串扰系数,由单转的供体样本和单转的受体样本得到,其计算公式如下:
\begin{align}
a&=\frac{I_{DA}(A)}{I_{AA}(A)}, \label{eq:a}\\
b&=\frac{I_{DD}(A)}{I_{AA}(A)}, \label{eq:b}\\ 
c&=\frac{I_{AA}(D)}{I_{DD}(D)}, \label{eq:c}\\ 
d&=\frac{I_{DA}(D)}{I_{DD}(D)}, \label{eq:d}
\end{align}
其中,$I_{DA}(A)$表示单转受体(A)样本在DA通道测得的荧光强度,其他参量意义类似。

E-FRET还需要测量得到荧光强度由DD通道转换为DA通道的转换因子$G$,在仪器系统参数保持不变时,$G$因子可以通过敏化发射荧光$F_C$在受体光漂白后荧光减少与受体光漂白后供体在DD通道的荧光恢复的比值确定,其定义如下:
\begin{equation}
    G=\frac{F_C-F_C^{post}}{I_{DD}^{post}-I_{DD}}, 
    \label{eq:g}
\end{equation}
其中$I_{DD}^{post}$是受体光漂白后受体敏化发射的荧光强度,$I_{DD}^{post}$是受体光漂白后供体的荧光强度。
获得敏化淬灭转化因子$G$和敏化发射荧光强度$F_C$后,供体角度的FRET效率$E_D$可以通过如下公式计算:
\begin{equation}
    E_D=\frac{F_C}{F_C+G \cdot I_{DD}}.
    \label{eq:ed}
\end{equation}
% \vspace{-2.5em} % 减少与后续文字的垂直间距

为了确定待测样本中受体与供体的浓度比例,需要首先通过测量受体与供体比例为1:1的FRET固定质粒样本来确定发生FRET时相等浓度的供体荧光和受体荧光的比例\upcite{chen2006measurement}:
\begin{equation}
    k=\frac{I_{DD}+F_C/G}{I_{AA}}.
    \label{eq:k}
\end{equation}
然后使用$k$测量待测样本中的受供体浓度比$R_C$,计算方式为:
\begin{equation}
    R_C = \frac{[A]}{[D]} = \frac{k \cdot I_{AA}}{I_{DD} + F_C/G}.
    \label{eq:rc}
\end{equation}

\section{FRET双杂交分析技术}

\subsection{基于Langmiur模型曲线拟合的FRET双杂交分析(L-FRET)}
蛋白质之间相互作用的化学计量比是阐明蛋白质间相互作用机制的重要参数,确定化学计量比能够进一步评估蛋白质间的生物相关性,并且能够确定其病理角色\upcite{zhao2020quantitative, clark2023single}。
传统的体外生化方法往往都需要从细胞中分离并且提纯大分子复合物才能进行测量,这类体外实验方法无法在活细胞中获得结果,而且一些大分子的复合物不容易进行分离提纯或者体外重建,这就限制了这些体外方法的应用\upcite{cui2019techniques}。

FRET过程改变了供受体复合物荧光发射谱的两个方面:(1)供体荧光淬灭;(2)受体荧光增强。
从这两个方面出发,FRET效率也可以从两种方式进行测量:(1)通过E-FRET方法从供体角度测量的FRET效率$E_D$,即供体转移给受体的能量占供体总能量的比例;(2)通过$3^3$-FRET方法从受体角度测量的FRET效率$E_A$,即受体敏化发射的荧光量占所有供体能量转移给受体时受体的荧光发射总量的比例。
$3^3$-FRET方法中,$E_A$可由如下公式给出:
\begin{equation}
    E_A = \frac{F_C}{a \cdot I_{AA}} \frac{\varepsilon_A}{\varepsilon_D},
    \label{eq:ea}
\end{equation}
其中,$\varepsilon_A / \varepsilon_D$是受体和供体的摩尔消光系数之比,$a$是光谱串扰系数,由公式 \ref{eq:a} 确定。

FRET双杂交分析是目前唯一可以在活细胞内进行生物大分子结合“滴定”实验的技术。
FRET双杂交分析实验中,通过不断增加受体的浓度使得每个供体都结合有受体,从而测出饱和结合时供体角度探测的最大$E_D$($E_{D,max}$)。
同样的方法,通过不断增加供体的浓度使得每个受体都结合有供体,从而测出饱和结合时受体角度探测的最大$E_A$($E_{A,max}$)。
在存在自由供体、自由受体和以$n_D:n_A$比例结合的受供体复合物($n_DD$-$n_AA$,$n_D$和$n_A$是供体和受体在复合物中的数量),当所有供体都被受体结合时,每个供体预期的能量转移效率为
\begin{equation}
    E_{D,max}=\frac{1}{n_D} \sum_{i=1}^{n_D} \sum_{j=1}^{n_A} E_{ij}.
\end{equation}
当所有受体都被供体结合时,每个受体预期的能量转移效率为
\begin{equation}
    E_{A,max}=\frac{1}{n_A} \sum_{i=1}^{n_D} \sum_{j=1}^{n_A} E_{ij}.
\end{equation}
于是,$E_{A,max}$与$E_{D,max}$的比值即为供受体的化学计量比:
\begin{equation}
    \upsilon = \frac{n_D}{n_A} = \frac{E_{A,max}}{E_{D,max}}. \label{eq:stoic}
\end{equation}

2016年,Butz等人提出将FRET效率和自由受供体浓度按照Langmiur模型进行拟合的FRET双杂交分析方法\upcite{butz2016}。
对于包含自由供体、自由受体和供受体复合物中,$E_D$和$E_A$可分别与自由受体浓度、自由供体浓度相关联:
\begin{align}
    E_A = E_{A,max} \frac{D_{free}}{D_{free}+K_{d,EFF}}, \label{eq:eadfree} \\
    E_D = E_{D,max} \frac{A_{free}}{A_{free}+K_{d,EFF}}, \label{eq:edafree}
\end{align}
其中,$K_{d,EFF}$为相对解离常数,$D_{free}$是自由供体的浓度,$A_{free}$是自由受体的浓度。
从公式 \ref{eq:eadfree} 和公式 \ref{eq:edafree} 可以看出,与体外滴定实验类似,用$3^3$-FRET法可得到$E_A$随自由供体浓度变化的动态滴定曲线,并得到$E_{A,max}$,用E-FRET方法也可以得到$E_D$随自由受体浓度的动态滴定曲线,并得到$E_{D,max}$。供受体复合物的化学计量比计算与公式 \ref{eq:stoic} 相同。

\subsection{\texorpdfstring{基于FRET效率$E_D$和受供体浓度比$R_C$线性分离的FRET双杂交分析}{基于FRET效率Ed和受供体浓度比Rc线性分离的FRET双杂交分析(DC-FRET)}}

L-FRET方法存在一定的局限性。从实验条件上来看,L-FRET需要得到FRET效率与自由供体/自由受体间的饱和滴定曲线,这意味着实验人员需要精心准备不同供受体浓度比的样本,并且这些样本的供受体浓度分布要尽可能均匀。
因此L-FRET往往需要配备一系列梯度比例的供体和受体溶液,然后分别对其进行混合和检测,这增加了实验样本的数量。
在实验数据处理时,大量浓度梯度都需要进行测量和数据处理,这极大地增加了实验人员的工作量和操作难度。
另一方面,从公式 \ref{eq:eadfree} 和 \ref{eq:edafree} 来看,$A_{free}$和$D_{free}$是拟合过程中更新的中间量,在实际的计算分析过程中,由于这些中间量的不确定性以及数据的复杂性,很容易出现拟合失败的情况。
一旦拟合失败,就需要重新进行实验或者调整参数,进一步增加了实验成本。

为了解决这些问题,在2018年,Du等人创新性地发展了基于FRET效率($E_D$)和受供体浓度比($R_C$)线性分离的FRET双杂交分析方法,简称为 Du-Chen-FRET(DC-FRET)\upcite{Du2018}。
DC-FRET聚焦关注分析供体饱和结合和受体饱和结合的数据。
当供体完全饱和时,即$D_{free}>>K_d$,此时受体被完全结合,以下公式成立:
\begin{align} 
    E_A &= E_{A,max}, \label{eq:ea_appro} \\
    E_D &= {E_{A,max}}{\cdot}{R_C}. \label{eq:ea_slope}
\end{align}
此时$E_D$与$R_C$成正比,且斜率为$E_{A,max}$。
当受体饱和时,即$A_{free}>>K_d$,此时供体被完全结合,以下公式成立:
\begin{align}
    E_D &= E_{D,max}, \label{eq:ed_appro} \\
    E_A &= E_{D,max}{\cdot}{1/R_C}. \label{eq:ed_slope}
\end{align}
此时$E_D$与$R_C$成正比,且斜率为$E_{A,max}$。
求得$E_{A,max}$和$E_{A,max}$后,供受体复合物中的化学计量比计算同公式 \ref{eq:stoic} 。

DC-FRET方法相比L-FRET方法有着一定的优势。
首先,在实验准备阶段,L-FRET由于需要得到FRET效率与自由供体/自由受体间的饱和滴定曲线,因此需要准备大量不同浓度的供体和受体样本,才能保证滴定曲线拟合的准确性。
DC-FRET方法只需要准备$R_C$比较大的样本和$R_C$比较小的样本,省去了不饱和结合的样本准备工作。
其次,L-FRET方法因为采用参数迭代拟合计算,容易放大异常数据带来的影响,从而出现结果超出物理意义范围或者出现不合理的结果,鲁棒性较差;DC-FRET方法线性拟合需要的$E_A$、$E_D$和$R_C$数据是$3^3$-FRET和E-FRET方法直接测量得到的,其数据可靠且方便筛选,通过这种方式得到的数据进行线性拟合,其结果也更加稳定可靠。

\section{FRET分析数据处理方法现状}

\subsection{FRET双杂交分析数据处理现状}
FRET 双杂交数据处理流程复杂耗时,人工依赖程度高,严重制约了实验效率与数据可重复性。
Butz 等指出,数据处理包括图像分析、FRET 定量计算及双杂交分析等多个环节,共28个步骤,单次实验数据处理过程需要 3.5 - 7 小时不等\upcite{butz2016}。
在图像分析环节,科研人员需手动标注上百个典型荧光区域作为 ROI,逐一定量灰度值并计算荧光信号与背景值,这一过程通常耗时 1 - 3 小时。
在 FRET 定量计算阶段,研究人员需将标注三通道荧光强度数据,再通过 Excel 设定公式完成批量计算,包括敏化发射荧光和 FRET 效率等参数。
这一过程虽部分实现自动化,但仍需人工输入数据并验证公式逻辑,耗时约 30 分钟至 1 小时。
在双杂交拟合计算中,使用 Excel 规划求解拟合 Langmuir 模型需经历 14 个步骤,耗时约 30 分钟。

通用软件缺乏针对 FRET 双杂交的优化。
Zen和ImageJ等通用软件可以手动标注ROI,但是缺少直观的辅助信息如信号背景比等,更无法展示FRET效率等直观的结果。
Excel 的规划求解功能在处理非线性拟合时灵活性不足,常需结合编程工具进行二次开发,频繁在 Excel 与 Matlab / Python 等工具间转移数据。

传统数据处理流程对科研人员的技术要求较高,不利于实验结果的复现和对比。
FRET图像处理中,人工标记 ROI 是一种盲处理,要求实验人员拥有丰富的处理经验和专业知识。
此类人工操作不仅效率低下,还因缺乏操作规范,易引入主观偏差,影响数据的可复现性。
在面对动态荧光事件或低信噪比数据时,人工分析的主观性与低效性尤为突出\upcite{zhou2024deep}。

这些挑战促使科研人员探索更高效的自动化解决方案,以推动 FRET 技术在生物医学研究中的广泛应用。

\subsection{基于机器学习的FRET数据处理}
近年来,随着机器学习不断发展,越来越多的研究者着手将此类方法应用于 FRET 数据分析。

深度学习技术为单分子FRET技术中高效解析动态轨迹数据提供了新方法。
Li等基于长短期记忆(Long Short-Term Momery, LSTM)和卷积神经网络(Convolutional Neural Network, CNN)的 AutoSiM 框架,自动分类和分割单分子荧光轨迹,提升 SiMREPS 检测灵敏度与 smFRET 分析效率,支持迁移学习适应新数据\upcite{li2020automatic}。
Zhang等人提出基于LSTM的Kin-SiM框架通过预训练模拟数据自动识别 smFRET 轨迹中的分子状态及动力学参数,实现轨迹理想化,性能与传统 HMM 相当但效率更高,支持多任务学习适应不同状态数\upcite{zhang2025pretrained}。
Miao等提出基于深度学习的局部特征分类框架DEBRIS和多帧双通道融合去噪网络(MUFFLE),通过滑动窗口技术和用户自定义标准,自动识别双色/单色单分子荧光事件。

多分子FRET技术中,应用深度学习技术可以提高选取FRET荧光信号的效率,避免了重复的人工数据处理。
Ge等人研发出一种基于 U-Net 模型的深度学习方法进行高效FRET分析,该模型基于一个包含230个手动标注ROI的数据集进行训练,随后通过旋转操作将数据集扩充,最终得到 2760 个样本\upcite{ge2020}。
U-Net 模型是图像分割领域广泛运用的卷积神经网络架构,它能够有效捕捉荧光图像的特征,精准地分割并提取 ROI 的荧光强度信息\upcite{ronneberger2015u}。
Thomsen 等人开发的 DeepFRET 软件基于深度学习技术,实现了从原始显微镜图像到 FRET 数据分类的全自动化流程\upcite{thomsen2020deepfret} 。
该软件通过预训练的深度神经网络对荧光轨迹进行逐帧分类,仅需用户设定质量阈值,即可快速生成高质量的 FRET 直方图。
Feldmann 等在进行FRET双杂交分析数据处理时采用了ilastik这一用于生物医学图像分析的开源交互式工具,将机器学习算法与用户交互相结合,能减轻手动标注的工作量\upcite{feldmann2023, berg2019ilastik}。

然而,机器学习驱动的FRET图像分析算法仍存在一定局限性。
首先,模型训练和泛化的效果依赖优质的数据集,而FRET图像的手动标注需专业知识且耗时费力,导致优质数据稀缺\upcite{kromp2020annotated}。
其次,深度人工神经网络模型的训练需要高性能图形处理器,伴随较长的训练时间,数据集越大,训练占用的时间可能会越久\upcite{9120226, huang2018gpipe}。
此外,深度学习模型的 “黑箱” 特性使其缺乏可解释性\upcite{zhang2021survey, fan2021interpretability}。
就基于强度的 FRET 定量分析而言,神经网络虽然能够捕捉图像的细微特征,但却无法用清晰的数学框架来解释决策逻辑,而且可能会引入未知的偏差。

\section{本文的工作内容和意义}

针对活细胞 FRET 双杂交分析数据处理存在的数据孤岛、流程繁琐、强人工依赖等核心问题,本文从以下方面开展了工作:

(1)首次设计并实现了针对FRET双杂交分析技术的数据处理软件 Fretha。
基于FRET双杂交分析技术的数据特点和处理流程,本文划分了成像参数设置、数据校验、FRET图像处理、数据管理、结果可视化等功能模块,并设计了用户友好的操作界面。
基于分层架构构建,Fretha由底层向上分为数据层、数据接口层、业务层和表现层。
通过封装 FRET 计算器、图像处理器和双杂交求解器,Fretha实现了E-FRET、$3^3$-FRET、DC-FRET、L-FRET等多种算法,为FRET定量分析计算提供了后台计算的能力。
软件采用 Qt 5.15.2 开发,支持跨平台部署,并通过 OpenCV 和 Dlib 库实现图像处理与优化计算\upcite{陆文周2015Qt5, bradski2000opencv, bradski2008learning}。
Fretha实现了从原始图像到化学计量比结果的一键式分析,有效替换了5款专业软件,避免了软件工具的切换,使得数据处理更加简单高效。
Fretha的推出,将有效降低FRET双杂交分析技术的使用门槛,提高实验结果的可对比性和可重复性,推动FRET双杂交技术在生物大分子相互作用研究中的应用。

(2)对 Fretha 软件进行了系统的验证和测试。
本文通过对标准质粒进行$3^3$-FRET和E-FRET分析,测得C4Y、C10Y、C40Y和C80Y的$E_A$、$E_D$和$R_C$值,测量得到的$E_{A}$值分别为$0.291\pm0.020$、$0.243\pm0.031$、$0.159\pm0.018$和$0.118\pm0.019$,$E_{D}$值分别为$0.307\pm0.040$、$0.230\pm0.022$、$0.155\pm0.011$和$0.117\pm0.012$,与文献值的相对误差分别为3.9\%、2.2\%,验证了Fretha计算FRET效率的准确性。
在FRET双杂交验证实验中,本文测量得Cerulean和Venus在C32V中的结合化学计量比为1.071,在CVC中为2.023,接近文献报道的结果,进一步验证了Fretha双杂交计算确定化学计量比的准确性。
本文针对每个功能模块单独进行了测试,以确认每个模块的功能效果,保证了软件运行时符合设计预期。
可靠性测试显示,软件在 48 小时连续运行和高频操作下保持稳定,资源占用无显著波动。
综合测试结果表明,Fretha 软件在数据处理准确性和可靠性方面均达到预期,支持连续稳定的数据处理。

(3)首次提出了基于明度和均匀度的自动ROI选取(Luminance-Uniformity-based ROI Selection, LURS)算法,实现了自动FRET双杂交分析。
LURS算法通过多通道自适应阈值分割、变异系数均匀性评估和连通区域分析,实现了荧光图像中高信噪比区域的精准识别,显著减少了人工标注 ROI 的时间成本。
结合LURS算法和鲁棒性较好的 DC-FRET 方法,Fretha软件提供了FRET 双杂交分析的数据处理全自动化,提高了大规模数据处理能力。
通过对比深度学习的方法,本文验证了LURS自动处理在性能上优于ilastik和SAM-Med2D,更适用于大数据量场景。
基于LURS算法的自动FRET双杂交分析为高通量药物筛选等大规模数据处理需求提供了高效可靠的算法工具,进一步推动了FRET双杂交分析技术的应用。

\section{本文的章节安排}
第一章,绪论。本章简要概述了定量FRET技术和FRET双杂交分析技术的理论基础和发展,然后介绍了FRET数据分析处理方法的研究现状。同时介绍了本文的工作内容和结构安排。

第二章,FRET双杂交分析数据处理软件的设计和开发。
本章基于FRET双杂交分析数据处理流程和FRET 多模态显微成像系统的特点进行需求分析,然后设计了Fretha软件的总体架构和模块划分,并详细介绍了各个模块的开发实现。

第三章,Fretha的验证和测试。
本章验证了使用Fretha进行$3^3$-FRET和E-FRET分析测量结果的准确性和稳定性,以及进行FRET双杂交分析数据处理的准确性。
针对每个模块,本章进行了单独的功能测试和性能测试,验证了Fretha软件的功能和性能符合预期。
最后,本章实施了Fretha稳定性测试,以测试Fretha长时间运行和高频操作下的稳定性。

第四章,基于LURS的自动FRET双杂交分析。
本章分析了FRET双杂交分析时的痛点,提出了基于明度和均匀度的自动ROI选择算法,并结合DC-FRET方法实现了全自动的FRET双杂交分析数据处理算法,在标准质粒和模型质粒上进行了验证实验。然后,本文还应用自动化数据处理算法分析了加药处理前后的Bcl-xL-Bak在活细胞内的相互作用,对比了深度学习方法的计算性能指标。

第五章,总结与展望。
本章总结了本文的工作内容和在相关领域的意义,对后续研究内容和方法进行展望。 


\chapter{Fretha的设计和开发}

\section{引言}

\ifshowtext
在现代生物医学研究中,FRET双杂交分析作为一种关键技术,用于检测生物分子间的相互作用,为揭示生物过程的分子机制提供了重要手段。
然而,传统的 FRET 双杂交分析数据处理过程繁琐且效率低下,严重依赖人工操作,容易引入误差。
开发专门针对 FRET 双杂交分析数据处理的配套软件,能够实现数据的自动化采集、标准化处理以及高效的分析统计,对于提升研究效率和准确性具有重要意义。
本章节首先根据我们组自研的用于高通量药物筛选的 FRET 显微成像系统(FRETscopeII)的硬件参数和 FRET 数据处理及 FRET 双杂交分析数据处理功能进行深入的需求分析。基于此需求分析,对软件按照功能进行科学合理的模块划分和总体框架设计。
接着,详细介绍软件的开发技术选型以及每个模块的具体实现方式,旨在全面展示 Fretha 软件从设计构思到开发实现的全过程。
\fi

\section{需求分析和总体设计}

\subsection{需求分析}

\ifshowtext
% FRET多模态显微成像系统(FRETscopeII)的硬件和数据特点;
FRETscopeII是本课题组自研的适用于$3^3$-FRET、E-FRET、Pb-FRET等多种定量FRET分析方法的FRET多模态显微成像系统。该系统具备高分辨率成像、多通道同步采集等先进特性,能够获取高质量的 FRET 三通道图像数据。

% FRET双杂交分析数据处理流程。
使用FRETscopeII对制备好的样本进行图像采集得到若干视野的FRET三通道图像后,通过FRET双杂交分析方法定量计算细胞中的生物大分子结合作用的FRET饱和效率、化学计量比、相对亲和力等信息。FRET双杂交分析的数据处理需要如下步骤:首先需要对FRET数据进行数据完整性校验,确保每个视野下存在完整的三通道图像和文件完整可读,然后对每个视野进行荧光信号提取,通过在图像上绘制ROI并计算其灰度均值作为FRET分析计算重要的荧光信号;根据E-FRET和$3^3$-FRET方法将上述荧光信号代入对应的计算公式求取$E_A$、$E_D$、$R_C$等FRET数据;对这些数据进一步依据物理含义或者数据分析进行异常值去除等数据预处理;最后是通过优化算法拟合Langmiur模型或者线性模型计算相关的参数。具体流程如图 \ref{fig:tha_data_process} 所示。

\begin{figure}[hbtp]
    \centering
    \includegraphics[width=0.4\linewidth]{../figures/2/2_FRET双杂交分析数据处理流程.png}
    \caption{FRET双杂交分析数据处理流程}
    \label{fig:tha_data_process}
\end{figure}

\fi

\subsection{模块划分}

结合 FRET 双杂交分析的数据处理流程与功能逻辑,Fretha 软件主要划分为以下核心模块:

\begin{enumerate}
  \item \textbf{成像参数设置模块:}负责设置 FRET 成像过程中的关键参数,确保成像数据的准确性和一致性。
  该模块允许用户输入和保存成像参数,如曝光时间、光学参数等,并提供参数校验功能,确保输入的参数在合理范围内,避免因参数设置错误导致的数据分析问题。
  成像参数一般比较稳定,通常2到3个月才需要重新测量,因此需要持久化到本地,以供多次处理数据时使用。
  \item \textbf{数据检验模块:}对输入数据进行严格的完整性和合法性检验,避免异常数据导致的分析错误。
  该模块会检查每个视野的三通道图像文件是否完整,并对文件格式和内容进行验证,确保数据的可靠性。
  通过数据完备性检验,可以避免异常数据导致的分析错误,提高数据处理的准确性。
  \item \textbf{FRET 图像处理模块:}支持手动图像处理和 ROI 选取,满足用户对数据的精细化处理需求。
  用户可以在图像上手动圈选感兴趣区域(ROI),并进行图像增强、滤波等处理操作。
  该模块提供多种图像处理工具,帮助用户提高图像质量和分析精度。
  手动选取感兴趣区域ROI是一个基础且关键的操作,它能帮助分析人员聚焦于细胞中荧光质量较高的区域,进行更精确的数据处理与分析。
  \item \textbf{自动 ROI 圈点模块:}结合LURS(Luminance-Uniformity-based ROI Selection)算法实现自动 ROI 选取,显著提高数据处理效率。
  该模块利用机器学习算法自动识别和圈选图像中的感兴趣区域,减少人工操作,提高处理速度和一致性。
  自动圈点功能通过FRET双杂交求解器中封装的多线程服务,在不影响Fretha前台界面显示和操作的前提下,实现了一键式线程分离的自动数据处理。
  \item \textbf{结果可视化模块:}将分析结果以直观的图表形式展示,并提供数据保存功能,方便用户进一步分析和应用。
  该模块支持多种图表类型,如趋势线图、散点图等,用户可以根据需要选择合适的图表进行数据展示和分析。
  通过结果可视化,用户可以更直观地理解分析结果,比如分析FRET双杂交分析结果的相关性、拟合程度等,从而确保数据处理的准确性和可靠性。
  结果可视化模块还可以将分析结果保存为图片或数据文件,方便用户进行后续数据处理和报告撰写。
  
\end{enumerate}

\subsection{软件总体框架}

Fretha架构采用分层设计,由顶层向下依次分别为:表现层、业务层、数据访问层和数据层,如图\ref{fig:fretha_arch}所示。

\begin{figure}[hbtp]
    \centering
    \includegraphics[width=1\linewidth]{../figures/2/2_Fretha软件架构图.png}
    \caption{Fretha软件总体架构}
    \label{fig:fretha_arch}
\end{figure}

表现层(Presentation Layer)处于整个架构的最外层,直接面向用户,是用户与系统进行交互的主要界面。
它负责接收用户输入的各种指令和数据,并以直观、友好的方式展示系统的处理结果。
Fretha的表现层主要包括参数设置界面、图像处理界面、ROI圈点界面、数据记录界面、可视化界面等,用户能够轻松地与系统进行交互,完成FRET双杂交分析的数据处理操作。

业务层(Business Logic Layer)是整个架构的核心逻辑处理部分,承担着对系统业务规则和流程的实现。
它接收来自表现层的请求,根据预设的业务逻辑对数据进行处理和转换。
Fretha的业务层封装了包括参数读/写业务、数据检验业务、自动ROI选取业务、数据导入/导出业务、FRET双杂交分析业务、数据绘图/可视化业务等。
FRET定量计算器和FRET图像分析业务是业务层中进行FRET定量计算的业务,分别可以对FRET原始灰度数据和FRET图像数据进行定量FRET计算分析。Fretha中这两个业务为其他业务提供了关于FRET定量计算的数据处理能力,使得FRET定量分析及FRET双杂交分析数据处理过程中的科学计算逻辑进一步独立,减少了软件架构的冗余度,提高了编码以及程序运行的效率。

数据访问层(Data Access Layer)实现对数据的访问和操作,它将业务层与数据层进行隔离。
数据访问层提供了统一的数据访问接口,业务层通过调用这些接口来获取和存储数据,而无需关心数据的具体存储方式和位置。
通过设置数据访问层,能够使得在复杂的业务处理时避免对数据的直接操作和影响,从而提高了数据存储的安全性。
Fretha的数据访问包括FRET图像数据访问、FRET数值数据访问和FRET双杂交数据访问的接口。

最后是数据层(Data Layer),作为架构的最底层,数据层负责存储系统的所有数据。
Fretha数据层包括系统静态数据、FretImage类、FretDataPiece类和FretRecord类。
系统静态数据是在软件运行时的环境参数,只需要在指定步骤运行前提前设置好即可,如成像参数、文件目录等。
FretImage类、FretDataPiece类和FretRecord类用来表示数据处理时的各种动态数据。
在FRET双杂交分析数据处理中,一组FRET三通道图像中可以提取并计算出若干条FRET数据,由若干条FRET数据作为一个批次只能解析出一条FRET双杂交分析结果。
因此,在设计上三种数据实体类型存在关联关系,FretImage和FretDataPiece之间存在一对多的关系,FretDataPiece和FretRecord之间存在一对多关系,如图 \ref{fig:fretha_data_relations} 所示。

\begin{figure}[hbtp]
    \centering
    \includegraphics[width=1\linewidth]{../figures/2/2_Fretha数据层对应关系.png}
    \caption{Fretha数据层实体关联关系图}
    \label{fig:fretha_data_relations}
\end{figure}

\subsection{开发技术选型}
\ifshowtext
Qt 5.15.2是Qt官方发布的长期支持(LTS)版本,具有卓越的跨平台性能。该版本能够在Windows、Linux、macOS以及嵌入式系统等多种平台上稳定运行。
开发者只需对一套代码进行简单配置,即可实现不同平台的适配,这种特性大幅降低了软件开发与维护的成本。FRETscopeII的控制软件系统选择了Qt 5.15.2版本作为技术选型。
为了确保Fretha和FRETscopeII在开发过程中的一致性和可融合性,Fretha同样选用该版本作为软件开发的基础技术。
这一选择不仅有利于代码的共享、复用和维护,还能显著降低二次开发以及未来合并开发过程中的成本和风险。

在计算机视觉处理方面,Fretha选用了OpenCV(Open Source Computer Vision Library)。
OpenCV是一个开源的计算机视觉与机器学习软件库,具备强大的跨平台能力,可在Windows、Linux、macOS、Android、iOS等多种操作系统上运行。
其功能丰富多样,广泛涵盖了图像和视频处理的各个方面,例如图像的读取、滤波、边缘检测等。
在性能上,OpenCV经过SIMD指令集优化以及多线程并行计算等技术的优化,处理效率得到了显著提升,能够充分满足实时性的需求。
此外,OpenCV拥有一个活跃的社区,汇聚了大量的开发者。
社区中提供了丰富的算法、示例代码和实际案例,为开发者提供了有力的技术支持和学习借鉴的资源。
基于以上诸多优势,OpenCV完全契合本项目中计算机视觉任务的需求,因此被选定为本项目的重要技术工具。

在最优化计算方面,Fretha选择了C++库Dlib。
Dlib库集成了基于梯度的优化算法,并采用了自适应学习率机制,能够实现快速且稳定的收敛,有效降低了陷入局部最优解的概率。
该库还采用了牛顿法与拟牛顿法,在保证计算精度的同时兼顾了计算效率。
此外,Dlib在处理约束优化问题方面表现出色,例如可以运用内点法来解决资源分配中的约束难题,为研究工作提供了可靠的支持。  
\fi

\section{FRET算法和后台接口}

\subsection{FRET定量计算器}
FRET定量计算器(FretCalculator)用于处理E-FRET和$3^3$-FRET定量计算。FRET定量计算器定义了参数设置、数据加载、数据校正、数据计算和结果获取等计算步骤,每个步骤需要按照顺序执行,并且在执行时记录运行状态,在获取结果数据之前进行检查来保证数据安全。

\begin{figure}[hbtp]
  \centering
  \includegraphics[width=0.25\linewidth]{../figures/2/2_FRET定量计算器流程.png}
  \caption{FretCalculator计算步骤}
  \label{fig:fretha_calculator}
\end{figure}

\subsection{FRET图像分析库}
FRET图像分析库(FretImageProcesser)封装了对FRET图像进行计算分析的计算器类,还静态方法的形式提供了图像处理算法的一系列接口。
首先按照FRET计算器的计算步骤对FRET图像数据进行处理,然后根据FRET图像数据的特点进行分析,完成逐像素的FRET计算,最后将计算结果保存到数据层中。
与FRET定量计算器不同,FRET图像分析时需要对每个像素点的荧光强度进行计算,因此需要对图像数据进行遍历处理。
FRET图像分析库采用OpenCV的Mat数据结构来存储图像数据。OpenCV提供的数据结构Mat是一个多维数组,可以方便地存储和处理图像数据,很适合存储逐像素的FRET图像数据。
FRET图像分析库的计算类FretImageProcesser的计算步骤和FRET定量计算器类似,如图 \ref{fig:fretha_calculator} 所示。

FRET图像分析库以静态方法的形式提供了FRET图像处理时涉及的一系列算法接口,包括图像预处理、图像分割、特征提取、图像增强等。所有的算法接口如表 \ref{tab:算法接口} 所示。
运用这些算法,FRET图像分析库支持了对16位原始数据的计算处理能力,以及可视化输出为伪彩图等功能。

\begin{table}[htb]
  \centering
  \caption{FRET图像处理库算法接口}
  \label{tab:算法接口}
    \begin{tabular*}{\textwidth}{p{0.3\textwidth}p{0.3\textwidth}p{0.4\textwidth}}
      \toprule[1.5pt]
      {\hei 接口} & {\hei 参数} & {\hei 说明} \\
      \hline

      morphologyClose & 
      \begin{tabular}[t]{@{}l@{}}
        Mat: 二值化图像 \\ 
        int: 迭代次数
      \end{tabular} & 
      形态学闭运算 \\

      morphologyOpen & 
      \begin{tabular}[t]{@{}l@{}}
        Mat: 二值化图像 \\ 
        int: 迭代次数
      \end{tabular} & 
      形态学开运算 \\
      
      medianFilter & 
      \begin{tabular}[t]{@{}l@{}}
        Mat: 单通道图像 \\ 
        int: 核大小
      \end{tabular} & 
      图像中值滤波 \\

      meanFilter & 
      \begin{tabular}[t]{@{}l@{}}
        Mat: 单通道图像 \\ 
        int: 核大小
      \end{tabular} & 
      图像均值滤波 \\
      
      gaussianFilter & 
      \begin{tabular}[t]{@{}l@{}}
        Mat: 单通道图像 \\ 
        int: 核大小 \\
        double: 高斯标准差
      \end{tabular} & 
      图像高斯平滑 \\
      
      getBackgroundValue & 
      Mat: 单通道图像 & 
      基于直方图的背景值估计 \\
      
      otsuThreshold & 
      \begin{tabular}[t]{@{}l@{}}
        Mat: 输入图像 \\ 
      \end{tabular} & 
      Otsu自动阈值分割 \\
      
      adaptiveThreshold & 
      \begin{tabular}[t]{@{}l@{}}
        Mat: 输入图像 \\ 
        int: 邻域大小(奇数) \\ 
        double: 阈值偏移量
      \end{tabular} & 
      自适应局部阈值分割 \\
      
      applyPseduoColor & 
      Mat: 单通道图像(8位) & 
      伪彩色映射(Jet颜色表) \\
      
      applyMask & 
      \begin{tabular}[t]{@{}l@{}}
        Mat: 输入图像 \\ 
        Mat: 掩膜(二值/同尺寸)
      \end{tabular} & 
      图像掩膜操作 \\
      
      minMaxNormalization & 
      Mat: 输入图像 & 
      全局线性归一化 \\
      
      mergeChannels & 
      \begin{tabular}[t]{@{}l@{}}
        Mat: R通道(8位) \\ 
        Mat: G通道(8位) \\ 
        Mat: B通道(8位)
      \end{tabular} & 
      多通道图像合并 \\
      \bottomrule[1.5pt]
    \end{tabular*}
\end{table}


\subsection{FRET双杂交求解器}
FRET 双杂交求解器对采集到的 FRET 批数据 $E_D$、$E_A$、$R_C$、$A_{est}$ 和 $D_{est}$ 进行最优化计算,以获取使预测结果与测量结果之间误差最小的$E_{A,max}$、$E_{D,max}$、$N_D / N_A$、$K_{d,EFF}$等参数。求解器从数据模型 FretRecord 中获取批量数据作为数据集,然后分别按照 DC-FRET 方法或 L-FRET 方法进行 FRET 双杂交分析求解。
具体算法如下:
\begin{enumerate}
  \item \textbf{DC-FRET线性拟合算法的封装}。根据公式\ref{eq:ea_appro}和\ref{eq:ed_appro},DC-FRET拟合斜率时截距项为0。以参数$E_{A,max}$的拟合过程为例,其线性方程形式为
  \begin{equation}
      E_D = E_{A,max}\cdot R_C,
  \end{equation}
  其中,$E_D$是自变量,$R_C$是自变量,$E_{A,max}$是斜率。线性拟合的目标是找到合适的参数$E_{A,max}$,使得方程预测的$E_D$值与实际观测到的$E_D$值之间的误差尽可能小。
  通常使用最小二乘法,其原理是最小化观测值与预测值之间的误差平方和,即
  \begin{equation}
      S=\sum^{n}_{i=1}(E_{D_i}-(E_{A,max}R_{C_i}))^2
  \end{equation}
  其中,$n$是数据点的数量,$R_{C_i}$和$E_{D_i}$分别是第i个数据点的自变量和因变量的值。
  为了找到$S$最小的$E_{A,max}$值,对$S$关于$E_{A,max}$求偏导,并令其等于0。
  首先,展开误差$S$:
  \begin{align}
       S=\sum_{i = 1}^{n}(E_{D_i}^{2}-2E_{A,max}R_{C_i}E_{D_i} + E_{A,max}^{2}R_{C_i}^{2}), \\
       \frac{\partial S}{\partial E_{A,max}}=\sum_{i = 1}^{n}(-2R_{C_i}E_{D_i} + 2E_{A,max}R_{C_i}^{2}),
  \end{align}
  令 \(\frac{\partial S}{\partial E_{A,max}}=0\):
  \begin{align}
       \sum_{i = 1}^{n}(-2R_{C_i}E_{D_i} + 2E_{A,max}R_{C_i}^{2}) = 0, \\
      -2\sum_{i = 1}^{n}R_{C_i}E_{D_i}+2E_{A,max}\sum_{i = 1}^{n}R_{C_i}^{2}=0,
  \end{align}
  最后求解$E_{A,max}$:
  \begin{equation}
         E_{A,max}=\frac{\sum_{i = 1}^{n}R_{C_i}E_{D_i}}{\sum_{i = 1}^{n}R_{C_i}^{2}}. \label{eq:linear_fit_quick}
  \end{equation}
  公式 \ref{eq:linear_fit_quick} 给出了线性拟合求解的解析公式,可用于直接计算斜率$E_{A,max}$和$E_{D,max}$,从而避免了基于迭代的线性拟合求解算法的消耗。

  \item \textbf{L-FRET Langmiur模型拟合算法的封装。}Langmiur模型具有模型的非线性和参数的复杂性,无法通过简单的解析式解析其求解公式。因此,我们引入dlib计算库进行复杂的参数拟合。

  为完成拟合计算,需要设计专门的数据类型ColumnVector和ExperimentalData,如代码 \ref{code:data_type}所示。
  ColumnVector类型用于存储双精度浮点数的列向量,在后续的计算和优化过程中承载参数向量和中间计算结果。
  ExperimentalData结构体用于存储来自实验采集到的数据集,该结构体包含四个std::vector<double>类型的成员变量:aest存储$A_{est}$相关的数据数组,dest存储$D_{est}$相关的数据数组,ea\_corr存储$E_A$预测值($E_{A,p}$)与测量值($E_{A,o}$)之间的误差,ed\_corr存储$E_D$预测值($E_{D,p}$)与测量值($E_{D,o}$)之间的误差。
  \begin{lstlisting}[language=C++, caption={数据类型}, label={code:data_type}]  
  // 定义列向量类型
  typedef dlib::matrix<double, 0, 1> ColumnVector;
  
  // 定义数据结构体,用于存储实验数据
  struct ExperimentalData {
      std::vector<double> aest;
      std::vector<double> dest;
      std::vector<double> ea_corr;
      std::vector<double> ed_corr;
  };
  \end{lstlisting}
  如代码 \ref{code:error_calculate} 所示,CalculateLoss函数负责计算模型在整个数据集上的整体损失。
  对于给定的ExperimentalData实例data和参数向量parameters,CalculateLoss函数首先遍历数据集中的每个数据点,计算出$E_A$或$E_D$预测值和与实际值之间的误差,最后将误差累加到total\_error中。
  \begin{lstlisting}[language=C++, caption={误差计算函数}, label={code:error_calculate}]
  // 计算整体损失
  double CalculateLoss(const ExperimentalData& data, const ColumnVector& parameters) {
      double total_error = 0.0;
      for (size_t i = 0; i < data.aest.size(); ++i) {
          double d_free = ((data.dest[i] - parameters(0) - data.aest[i] * parameters(1)) + std::sqrt(std::pow(data.dest[i] - parameters(0) - data.aest[i] * parameters(1), 2) + 4 * parameters(0) * data.dest[i])) / 2;
          double a_free = data.aest[i] - (data.dest[i] - d_free) / parameters(1);
          double ea_pred = parameters(2) * d_free / (d_free + parameters(0));
          double ed_pred = parameters(3) * a_free / (a_free + parameters(0) / parameters(1));
  
          total_error += CalculateError(data.ea_corr[i], ea_pred) + CalculateError(data.ed_corr[i], ed_pred);
      }
      return total_error;
  }
  \end{lstlisting}
  TwoHybridSolver是用来调用整个拟合过程的函数,并向外提供调用接口,如代码 \ref{code:two_hybrid_solver} 所示。
  {TwoHybridSolver} 函数接受四个 std::vector<double> 类型的参数,分别为 {aest\_data}、{dest\_data}、{ea\_corr\_data} 和 {ed\_corr\_data},返回一个 {ColumnVector} 类型的对象,其中存储了经过拟合的结果参数。整个拟合过程如下:
  首先设置$K_{d,EFF}$、$n_D/n_A$、$E_{A,max}$和$E_{D,max}$的拟合初值为 1、1、0.5 和 0.5, 存储到{ColumnVector} 对象 {starting\_point},作为参数拟合的起始点。
  创建一个 {ExperimentalData} 类型的对象 {data},将传入的四个实验数据向量存储在该结构体中,以便后续目标函数使用。
  使用 {dlib} 库中的 {find\_min\_using\_approximate\_derivatives} 函数进行优化。
  在搜索策略算法上,求解器采用一种拟牛顿法 {dlib::bfgs\_search\_strategy()} ,用于在参数空间中寻找目标函数的最小值,并使用 {dlib::objective\_delta\_stop\_strategy(1e - 7)} 作为停止策略,当两次拟合后参数的变化小于 $10^{-7}$ 时,认为此时的结果已收敛,停止优化过程。
  最后,函数返回 {starting\_point},此时 {starting\_point} 存储的是经过优化后得到的拟合参数。
  \begin{lstlisting}[language=C++, caption={双杂交求解器}, label={code:two_hybrid_solver}]
    
    // 双杂交求解器,进行参数拟合
    ColumnVector TwoHybridSolver(const std::vector<double>& aest_data, const std::vector<double>& dest_data, const std::vector<double>& ea_corr_data,const std::vector<double>& ed_corr_data) {
        // 初始化起始点
        ColumnVector starting_point(4);
        starting_point = 1, 1, 0.5, 0.5;
    
        // 创建数据结构体
        ExperimentalData data = {aest_data, dest_data, ea_corr_data, ed_corr_data};
    
        // 定义目标函数包装器
        auto objective_wrapper = [&data](const ColumnVector& parameters) {
            return ObjectiveFunction(parameters, &data);
        };
    
        // 使用 dlib 进行优化
        dlib::find_min_using_approximate_derivatives(dlib::bfgs_search_strategy(), dlib::objective_delta_stop_strategy(1e-7), objective_wrapper, starting_point, -1, 0.01);
  
        return starting_point;
    }
    \end{lstlisting}  
\end{enumerate}

\section{功能模块的实现}

\subsection{成像参数设置模块}
\ifshowtext
FRET定量分析中,在数据处理前需要设置好FRET定量计算过程中必须的参数,设置成像过程时的成像参数至关重要。
成像参数设置模块的界面如图 \ref{fig:fretha_param_module} 所示,包括了FRET成像参数的设置和保存功能。
\begin{figure}[hbtp]
  \centering
  \includegraphics[width=0.9\linewidth]{../figures/2/2_参数设置界面.png}
  \caption{成像参数设置模块界面}
  \label{fig:fretha_param_module}
\end{figure}

FRET成像参数在Fretha中以静态参数保存在软件内存中,是数据处理时的环境参数。其中,a、b、c、d、G、k和Y是FRET成像系统的光学参数,在前文中已介绍;ExpTimeAA、ExpTimeDD和ExpTimeDA是成像时三个探测通道的曝光时间,在FRET定量计算时需要根据曝光时间参数在各个通道归一化,然后才能进行计算。
Fretha中包括的所有成像参数如表 \ref{tab:fretha_param_list} 所示。

\begin{table}[htb]
  \centering
  \caption[FRET成像参数]{FRET成像参数}
  \label{tab:fretha_param_list}
    \begin{tabular*}{\textwidth}{cp{8cm}lc}
      \toprule[1.5pt]
      {\hei 参数} & {\hei 说明} & {\hei 意义范围} & {\hei 单位} \\
      \hline
      \text{a} & 供体激发时在DA通道的串扰系数 & $(0,1)$ & 无\\
      \text{b} & 所有的一切都在这里面。 & $(0,1)$ & 无\\
      \text{c} & 模板类文件。 & $(0,1)$ & 无\\
      \text{d} & 受体激发时在DA通道的串扰系数 & $(0,1)$ & 无\\
      \text{G} & 供体猝灭和受体荧光增强的比值         & $(0,+\infty)$ & 无\\
      \text{k} & 受体浓度和供体浓度相同时的荧光比值 & $(0,+\infty)$ & 无\\
      \text{Y} & 供受体在激发光条件下的消光系数之比,也记作$\varepsilon_{YFP}(\lambda)/\varepsilon_{CFP}(\lambda)$   & $(0,+\infty)$ & 无\\
      \text{ExpTimeDD} & DD通道下的成像曝光时间 & $(0,+\infty)$ & 毫秒(ms)\\
      \text{ExpTimeDA} & DA通道下的成像曝光时间 & $(0,+\infty)$ & 毫秒(ms)\\
      \text{ExpTimeAA} & AA通道下的成像曝光时间 & $(0,+\infty)$ & 毫秒(ms)\\
      \bottomrule[1.5pt]
    \end{tabular*}
\end{table}

业务设计上遵循一定的原则。首先,设置时所有参数应一同更新,避免因参数不匹配导致的数据处理错误。其次,每个参数需要在其有意义的范围内,避免无意义的值。
因此,在点击“更新参数”按钮时,若无法从界面中的每个参数输入框都解析到合法的数字,那么本次更新参数就会失败。
这是因为FRET成像参数是一批参数,参数间存在依赖关系,如测量参数$G$、$k$、$Y$就依赖参数$a$、$b$、$c$、$d$,这是因为测定参数$G$、$k$、$Y$时需要计算敏化发射荧光$F_C$,根据公式 \ref{eq:fc} 所示,$F_C$的确定与$a$、$b$、$c$、$d$密切相关。
强制所有参数一同更新可以避免用户单独设置某一参数而导致参数之间不匹配等问题。

FRET成像参数一般比较稳定,一般2到3个月才需要重新测量,因此需要持久化到本地,以供多次处理数据时使用。
Fretha的本地参数文件保存为可执行程序同级目录下的“config.ini”中。在软件初始化阶段,会自动检测并应用本地配置文件中的参数。
用户可通过保存多套配置文件,在使用时替换目标配置文件,快速进行参数配置的切换。

参数设置的主要业务流程如图 \ref{fig:fretha_param_module_flow} 所示。

\begin{figure}[hbtp]
    \centering
    \includegraphics[height=1\linewidth]{../figures/2/2_成像参数设置模块业务流程.png}
    \caption{参数设置模块业务流程}
    \label{fig:fretha_param_module_flow}
\end{figure}
\fi

\subsection{数据检验模块}
\label{sec:数据检验模块}

FRET双杂交分析需要处理一批FRET图像文件,因此需要对输入数据的完备性进行检验识别。
该模块的作用有以下两个方面:一方面通过模式识别FRET合法数据,避免了异常输入导致的运行错误;
另一方面,在这一模块会将FRET批数据的视野子文件夹进行解析和类型识别,为后续数据处理提供对子文件夹的不同操作。

Fretha的数据识别检验模块匹配识别FRETscopeII的数据格式,从而保证数据处理能够正常开始。
如图 \ref{fig:fretscope_data_struct} 所示,FRETscopeII的数据结构由上层到下层依次为:
\begin{enumerate}
  \item 批数据根目录:存放所有视野子目录和数据文件;
  \item 视野子目录:存放一个视野的三通道图片和参数文件;
  \item 图片数据和参数文件。
\end{enumerate}

\begin{figure}[htbp]
    \centering
    \includegraphics[height=0.5\linewidth]{../figures/2/2_FRETscopeII数据格式.drawio.png}
    \caption{FRETscopeII数据文件结构}
    \label{fig:fretscope_data_struct}
\end{figure}

数据检验业务的流程如图 \ref{fig:fretha_data_check_flow} 所示。
其中,检查子文件夹类型是通过图 \ref{fig:fretscope_data_struct} 进行匹配的,当且仅当子文件夹中同时存在“DA.tif”、“DD.tif”和“AA.tif”图片文件时,当前子文件夹会被识别为FRET视野,并在视野表格模型中记录。
其他情况的子文件夹会被记作“Unknown”文件夹,在后续FRET图像处理或者自动处理中被跳过。
这种检查还会对图像数据是否可读进行检查。
\begin{figure}[htbp]
    \centering
    \includegraphics[width=0.6\linewidth]{../figures/2/2_数据完备性检验业务.drawio.png}
    \caption{Fretha数据检验业务主流程图}
    \label{fig:fretha_data_check_flow}
\end{figure}
\subsection{FRET图像处理模块}
\label{sec:FRET图像处理模块}
FRET图像处理模块是对FRET三通道图像进行ROI提取等图像处理分析的模块,包括手动ROI圈点和自动ROI圈点等。其界面如图 \ref{fig:fretha_imageprocess_ui} 所示:
\begin{figure}[htbp]
    \centering
    \includegraphics[width=0.75\linewidth]{../figures/2/2_图像处理模块界面.png}
    \caption{Fretha图像处理模块界面}
    \label{fig:fretha_imageprocess_ui}
\end{figure}
在FRET图像处理工作里,手动选取感兴趣区域ROI是一个基础且关键的操作,它能帮助分析人员聚焦于特定区域,进行更精确的数据处理与分析。
为实现图像处理过程中 ROI 的手动选取功能,Fretha 借助 Qt 框架提供的 QGraphicsView 类,开发了自定义的 FretGraphicsView 类。
QGraphicsView 是 Qt 用于可视化和交互处理二维图形场景的重要类,具备丰富的功能和良好的可扩展性,为 FretGraphicsView 的实现提供了有力支撑。

FretGraphicsView 类中设计了一个基于QRectItem的ROI 成员,它能够在视图中直观呈现 ROI 的大小和位置,方便用户确认所选区域。
设定ROI的绘制模式为“Crop Mode”截取模式时,此时会根据鼠标和ROI边框的位置,变换ROI交互的功能,包括ROI的创建、移动、缩放等操作。
切换绘制模式为“Stamp”邮戳模式,此时可以固定ROI的大小,从而快速圈选多个大小一致的ROI。

Fretha的状态栏中提供了视图类型切换选项,在数据处理中可以辅助圈点,支持的视图类型如表\ref{tab:fretha_viewtype_list}所示。
\begin{table}[htbp]
  \centering
  \caption[FRET图像视图类型]{FRET图像视图类型表}
  \label{tab:fretha_viewtype_list}
    \begin{tabularx}{\linewidth}{
    >{\centering\arraybackslash}X
    >{\centering\arraybackslash}X
    >{\centering\arraybackslash}X
    >{\centering\arraybackslash}X
    >{\centering\arraybackslash}X
    >{\centering\arraybackslash}X} % 修改第二列格式为p{6cm},可根据实际情况调整宽度
      \toprule[1.5pt]
      {\hei 信息} & {\hei 说明} \\
      \hline
      Channel Merged & 三通道归一合成图 \\
      DD Normalized & DD通道的归一化增强图 \\
      DA Normalized & DA通道的归一化增强图 \\
      AA Normalized & AA通道的归一化增强图 \\
      $R_C$ Pseudo & $R_C$逐像素数值伪彩图 \\
      $E_D$ Pseudo & $E_D$逐像素数值伪彩图 \\
      \bottomrule[1.5pt]
    \end{tabularx}
\end{table}
归一化增强图包括DD、DA、AA通道分别归一化的增强视图以及三通道归一化合成图。
由于FRET成像系统的相机的量程为$[0,65535]$,然而计算显示机制中,显示器的量程为$[0,255]$,因此需要对图像进行归一化处理。
通过表 \ref{tab:算法接口} 中的 minMaxNormalization 算法,可以对图像进行全局线性归一化,从而增强图像的对比度。
三通道归一化合并图则是将三个通道的归一化增强图分别作为RGB色彩通道,合并成一幅彩色图像,以便于用户直观地观察三通道的信号分布情况。
增强视图的算法首先通过FretImageProcessor进行逐像素FRET计算得到$E_D$和$R_C$的逐像素矩阵,然后归一化后赋予伪彩图。

Fretha状态栏能够清晰展示当前视野及ROI的状态信息,如当前视野的三通道背景灰度值、ROI信号的三通道信号背景比等,还可以显示使用当前ROI提供的扣除背景灰度值后的$I_{DD}$、$I_{DA}$和$I_{DD}$,其中所有显示内容如表\ref{tab:fret_statusbar_list}所示。
FretGraphicsView 通过自定义鼠标释放事件的信号,实现了与数据访问层的数据交互。
当用户完成 ROI 选取并释放鼠标时,FretGraphicsView 会将 ROI 的坐标和大小信息传递给数据访问层。
数据访问层依据当前视野索引,从数据模型中读取相应的三通道图像文件,并且从ROI信息提取FRET信号。
在获得了$I_{DD}$、$I_{DA}$和$I_{DD}$后,Fretha后台会自动扣除每个通道的背景灰度,然后根据公式\ref{eq:fc},调用后台的FretCalculator计算出敏化发射的荧光强度$F_C$、$E_D$,$R_C$,并显示在状态栏中,方便用户获取处理结果,为ROI的保留和移除提供判断依据。
这种基于 QGraphicsView 扩展的设计,有效提升了 ROI 选取的灵活性和准确性。 

\begin{table}[htbp]
  \centering
  \caption[FRET圈点状态栏显示内容]{FRET圈点状态栏显示内容}
  \label{tab:fret_statusbar_list}
      \begin{tabular}{cc}
      \toprule
      {信息} & {说明} \\
      \hline
      DD通道信号 & DD通道的ROI内灰度均值 \\
      DA通道信号 & DA通道的ROI内灰度均值 \\
      AA通道信号 & AA通道的ROI内灰度均值 \\
      DD通道背景 & DD通道的视野背景灰度值 \\
      DA通道背景 & DA通道的视野背景灰度值 \\
      AA通道背景 & AA通道的视野背景灰度值 \\
      DD通道SBR & DD通道的信号背景比 \\
      DA通道SBR & DA通道的信号背景比 \\
      AA通道SBR & AA通道的信号背景比 \\
      $F_C$ & 敏化发射的荧光强度 \\
      $E_D$ & 供体视角的表观FRET效率 \\
      $R_C$ & 受体与供体的浓度比 \\
      \bottomrule
    \end{tabular}
\end{table}
点击左侧“自动圈点”按钮,可以使用LURS算法进行自动ROI圈点生成,并记录到右侧的数据区中。
计算过程中,Fretha界面通过进度条显示计算进度,方便用户查看。
自动圈点功能通过FRET双杂交求解器中封装的多线程服务,在不影响Fretha前台界面显示和操作的前提下,实现了一键式线程分离的自动数据处理。
自动ROI圈点模块的业务流程及架构如图\ref{fig:fret_auto_roi_flow}所示。
架构层内的流程流转以实线表示,不同架构层之间的流程流转以虚线表示。
在实现自动圈点功能的流程转移中,发生在架构不同层之间的转移占更多数,而层内的流程转移相对较少,分层架构设计使得每个层内实现较好的封装,因此能够在实现复杂业务功能时,只需要专注于层和层之间的数据和流程切换即可。
LURS算法将会在第三章具体说明。
\begin{figure}[htbp]
    \centering
    \includegraphics[width=1\linewidth]{../figures/2/2_自动ROI圈点模块业务流程.png}
    \caption{Fretha自动圈点模块流程}
    \label{fig:fret_auto_roi_flow}
\end{figure}

\subsection{数据管理模块}
\label{sec:数据管理模块}

数据管理模块支持对数据的实时操作,可以对于图像处理模块获得的数据,进行异常数据筛选、数据追踪、数据导入、数据导出、计算入口等数据管理控制功能。

一条数据类型FretDataPiece中包含数据如表 \ref{tab:数据项内容定义} 所示。
可以看出,FretDataPiece包含了一个ROI对应的所有原始信息,如ROI的位置、大小、信号强度、视野等;
同时,还包含了计算后的$E_D$、$E_A$、$R_C$、$1/R_C$等中间参数,可以直接被用于后续的数据处理。

\begin{table}[htbp]
  \centering
  \caption{FretDataPiece数据类型}
  \label{tab:数据项内容定义}
    \begin{tabular}{cc}
      \toprule
      {\hei 信息} & {\hei 说明} \\
      \hline
      $I_{DD}$ & ROI在DD通道扣除背景后的信号强度 \\
      $I_{DA}$ & ROI在DA通道扣除背景后的信号强度 \\
      $I_{AA}$ & ROI在DD通道扣除背景后的信号强度 \\
      $E_D$ & 根据$I_{DD}$、$I_{DA}$、$I_{AA}$计算的$E_D$ \\
      $R_C$ & 根据$I_{DD}$、$I_{DA}$、$I_{AA}$计算的$R_C$ \\
      $E_A$ & 根据$I_{DD}$、$I_{DA}$、$I_{AA}$计算的$E_A$ \\
      $1/R_C$ & $R_C$的倒数 \\
      $A_{est}$ & 供体相对分子数的估计值 \\
      $D_{est}$ & 受体相对分子数的估计值 \\
      $x$ & ROI的横坐标 \\
      $y$ & ROI的纵坐标 \\
      $w$ & ROI的宽度 \\
      $h$ & ROI的高度 \\
      $view$ & ROI隶属的视野\\
      \bottomrule
    \end{tabular}
\end{table}

为保证数据符合实际物理意义,避免中间变量的异常值对后续计算产生影响,数据管理模块提供了数据筛选功能。
点击“筛选数据”,数据管理模块会对数据根据物理定义和统计学准则进行筛选,剔除异常数据。
具体数据筛选流程实施如下:
\begin{enumerate}
  \item 对$I_DD$、$I_DA$、$I_AA$基于物理约束的初步数据清洗:
    针对各数值型变量 \( X_i \),依据其预设的物理合理区间 \( [L_i, U_i] \) 执行有效性校验。构建如下二元判别函数:
    \begin{equation}
      \delta(x_i) = 
      \begin{cases} 
        1, & x_i \notin [L_i, U_i] \\
        0, & x_i \in [L_i, U_i]
      \end{cases}
    \end{equation}
    当 \( \delta(x_i) = 1 \) 时,判定该数据点为物理意义上的异常值并予以剔除。
  \item 对中间变量$E_D$、$E_A$、$R_C$统计离群点检测:
    对经初步清洗后的数据子集 \( E_D \) 和 \( E_A \),分别计算变量的样本均值 \( \mu \) 和样本标准差 \( \sigma \),计算公式如下:
    \begin{equation}
      \mu = \frac{1}{n}\sum_{i=1}^n x_i, \quad \sigma = \sqrt{\frac{1}{n - 1}\sum_{i=1}^n (x_i - \mu)^2}
    \end{equation}
    采用3σ准则设定离群阈值,即构建判别区间 \( [\mu - 3\sigma, \mu + 3\sigma] \)。超出该区间的观测值均判定为统计离群点并剔除。
\end{enumerate}

数据导出用于保存数据处理时的ROI信息。
在完成FRET图像处理和ROI绘制选取后,需要保存ROI的结果,以便后续分析或者修改编辑。
在数据管理模块点击“导出数据”按钮,Fretha将会导出数据区记录的所有数据,每一条数据都会按照FretDataPiece定义进行逐列导出,如表 \ref{tab:数据项内容定义} 所示,文件格式为CSV文件。
数据导入功能可以选择CSV文件,然后解析其中的数据按照FretDataPiece的格式。

点击“添加数据”按钮或者使用快捷键“A”,可以添加当前状态栏中的数据到数据表格中;
点击“删除数据”按钮或者使用快捷键“D”,可以删除数据表格中选中的数据;
点击“清空数据”按钮或者使用快捷键“C”,可以清空数据表格中的所有数据;
点击“开始计算”按钮,软件将调用FretTwoHybridSolver等算法对数据表格中的数据进行FRET双杂交分析求解计算,并将结果显示在结果可视化模块中,然后切换软件界面到结果可视化模块。

在数据表格中点击某一条数据,软件会根据这条数据的位置和形状信息,在图像显示上显示该条数据对应的ROI位置,以方便识别数据中潜在的错误。
实现这一功能,主要基于Qt提供的信号与槽机制。
将数据中的数据模型被点击的事件信号与FretGraphicsView控件中的回调槽函数绑定后,FretGraphicView模型可以接收到来自所选数据的信息,并将当前活动的ROI按照所选数据项中的信息$x$、$y$、$w$和$h$绘制到FretGraphicsView控件上。
接下来,FretGraphicsView会将窗口的视角移动到所选数据的ROI位置,以便用户更加直观地查看数据的位置和形状。

\subsection{结果可视化模块}
\label{sec:结果可视化模块}
\ifshowtext
FRET双杂交分析的输出结果包含双杂交计算的参数结果与拟合曲线图。
其中,$n_D/n_A$、$K_{d,EFF}$等参数作为表征生物大分子相互作用的核心量化指标,为生物学研究提供关键数据支撑。
拟合曲线图通过将双杂交理论拟合曲线与实验散点数据同图可视化,实现对拟合效果的直观评估,进而验证FRET双杂交分析结果的可靠性。
\begin{figure}[h]
  \centering
  \includegraphics[width=0.9\linewidth]{../figures/2/2_结果可视化.png}
  \caption{Fretha结果可视化模块界面}
  \label{fig:fretha_result_ui}
\end{figure}

结果可视化模块的界面组成如图 \ref{fig:fretha_result_ui} 所示,主要包含以下三个功能区域:
\begin{enumerate}
  \item 视图选择:切换FRET双杂交分析算法,更新对应的结果视图;
  \item 图表分析:显示FRET双杂交分析的拟合趋势线和散点图;
  \item 操作按钮:结果保存按钮和返回圈点界面按钮。
\end{enumerate}

点击视图按钮可以切换不同FRET双杂交分析的算法方法的计算结果,更新图表分析区域显示的方法。
在可视化结果展示区域,软件将拟合结果与实验数据同步呈现在同一坐标系中,便于用户进行直接比对。
拟合曲线是基于拟合参数计算得到的理论数据点连接而成的平滑曲线,实验数据则以离散点形式展示,用户可以直观检查拟合计算的效果和误差大小。
可视化界面基于QChart组件作图绘制,根据一批数据(FretDataRecord)的数据范围分布,自动优化坐标轴刻度范围,以提升数据可视化的清晰度。

在 L-FRET 视图模式下,软件集成了数据预处理 BIN 功能,允许用户通过设定 BIN 的上下限及间隔参数,对数据合并预处理的参数进行灵活调节,然后点击“更新”按钮即可更新结果,在L-FRET计算效果不理想时可以用来优化数据处理结果。

\begin{figure}[htbp]
  \centering
  \includegraphics[width=0.9\linewidth]{../figures/2/2_DC-FRET结果界面.png}
  \caption{Fretha DC-FRET视图}
  \label{fig:fretha_dc_fret}
\end{figure}

DC-FRET视图的界面如图 \ref{fig:fretha_dc_fret} 所示。
在DC—FRET视图模式下,软件能显示出DC-FRET的拟合结果,包括$E_{A,max}$、$E_{D,max}$、$n_D/n_A$,以及拟合曲线和实验数据的对比图。
界面中设计了调整线性拟合参数的设置栏,用户可以调整$R_C$($1/R_C$)的数据范围,点击“更新”按钮以应用范围参数,用来处理复杂的数据。

点击“保存结果”按钮后,系统将同步存储拟合结果数据、可视化图像及实验数据,保存结果如图表所示。
\begin{table}[htbp]
  \centering
  \caption{结果保存生成文件}
  \label{tab:fretha_result_list}
    \begin{tabular}{cc}
      \toprule
      {文件名} & {说明} \\
      \hline
      Ea-Rda图.png & $E_A$-$1/R_C$散点和趋势线图 \\
      Ed-Rad图.png & $E_D$-$R_C$散点和趋势线图 \\
      Ea-Dfree图.png & $E_A$-$D_{free}$散点和趋势线图 \\
      Ed-Afree图.png & $E_D$-$A_{free}$散点和趋势线图 \\
      FretThaData.csv & FRET双杂交分析结果数据 \\
      FretThaResults.csv & FRET双杂交分析结果拟合参数 \\
      \bottomrule
    \end{tabular}
\end{table}
其中,实验数据与拟合参数的原始数值将以 CSV 文件格式进行保存,该文件包含可直接用于其他科研绘图软件的数据记录,从而支持用户在不同可视化工具中进行后续的图形优化与再处理。
这一功能设计保障了实验结果的完整留存,为科研工作者提供了灵活的数据导出与再分析解决方案。
\fi

\section{本章小结}

\ifshowtext
本章介绍了FRET双杂交分析数据处理软件(Fretha)的设计和开发。
Fretha采用分层架构设计,被分为表示层、业务层、数据访问层和数据层,从设计上尽可能地进行解耦,减少了冗余设计。
基于FRET双杂交分析数据处理的需求,Fretha在功能逻辑上划分为成像参数设置模块、数据校验模块、FRET图像处理模块、数据管理模块和结果可视化模块,然后通过分层架构完成了每个功能模块的开发。
得益于分层架构设计和功能模块化设计,Fretha实现了各种复杂Fretha说明Fretha是一款拥有用户友好界面、简单易用、处理高效的科研数据处理软件。
\fi
\chapter{Fretha的测试}

\section{引言}
软件测试是确保软件质量和性能的重要环节。
结合Fretha的应用实验,我们对软件的功能模块进行了测试,包括成像参数设置模块、数据检验模块、FRET图像处理模块、数据管理模块、结果可视化模块等。
我们对集成在Fretha中的自动圈点算法LURS进行了性能测试,并对比了SAM-Med2D、ilastik等其他自动荧光信号提取的算法。
针对系统的稳定性,测试了软件在长时间运行和高频操作下的表现。
综合测试结果表明,Fretha在功能丰富度、性能表现和稳定性方面均达到了预期目标,能够满足复杂科研和高通量应用需求。

\section{材料与方法}
\subsection{细胞培养与转染}
\label{sec:细胞质粒}
标准质粒实验中,中国科学院细胞库提供的MCF-7细胞株在含有10\%胎牛血清、100 U/mL青霉素和100 $\mu$g/mL链霉素的DMEM培养基中培养。
在Bcl-xL-Bak药效检测实验中,将MCF-7细胞(4000个细胞/孔)接种于含DMEM培养基和10\%胎牛血清的96孔板(中国LABSELECT公司),置于37$^\circ \text{C}$、5\% $\text{CO}_2$培养箱中孵育12小时。每孔转染400 ng质粒,并以3:1或1:3的比例形成转染复合物。

Turbofect\texttrademark{}转染试剂购自美国赛默飞世尔科技公司。所有质粒均由Steven Vogel赠送:C17V(Addgene质粒\# 26395)、C32V(Addgene质粒\# 26396)、mVenus N1(Addgene质粒\# 27793)、mCerulean C1(Addgene质粒\# 27796)、CVC(Addgene质粒\# 27809)\upcite{koushik2006cerulean,thaler2005quantitative}。CFP(青色荧光蛋白)-YFP(黄色荧光蛋白)二聚体质粒包括YFP-G4-CFP(C4Y)、YFP-G10-CFP(C10Y)、YFP-G40-CFP(C40Y)和YFP-G80-CFP(C80Y),由Christian Wahl-Schott赠送 \upcite{butz2016}。CFP-Bcl-xL质粒由A. P. Gilmore提供 \upcite{warren2019bcl},YFP-Bak质粒的构建方法已先前报道过 \upcite{sun2023regorafenib}。药物A1331852购自美国新泽西州MCE公司。

\subsection{FRET成像系统}
\label{sec:成像条件}
\ifshowtext
本研究中,所有实验数据均使用自主研发的多模态FRET自动化成像系统获取 \upcite{sun2022automated}。
对于CY标准质粒,实验选用了20倍的0.45NA物镜(Olympus,日本)和6\%光照强度。
模型质粒实验,选用了20倍的0.45NA物镜(Olympus,日本)和50\%光照强度。
实验过程中,在AA通道寻找视野,然后依次捕获AA、DA和DD通道的荧光图像。

对于CV质粒,串扰因子$a$和$b$通过单转Venus质粒测量,串扰因子$c$和$d$通过单转Cerulean质粒测量,系统校正因子$G$和$k$和$\varepsilon_{YFP}(\lambda)/\varepsilon_{CFP}(\lambda)$是由标准质粒C17V和C32V测量。

对于CY质粒,串扰因子$a$和$b$通过单转YFP质粒测量,串扰因子$c$和$d$通过单转CFP质粒测量,系统校正因子$G$和$k$和$\varepsilon_{YFP}(\lambda)/\varepsilon_{CFP}(\lambda)$是由标准质粒C4Y、C10Y、C40Y和C80Y质粒测量。
所有的FRET成像参数如表 \ref{tab:lurs_imaging_params} 所示。

\begin{table}[htbp]
    \centering
    \caption{FRET成像系统参数}
    \begin{tabular}{ccc}
        \toprule[1.5pt]
        参数名 & CV质粒成像参数 & CY质粒成像参数 \\
        \midrule
        a & 0.206 & 0.160\\
        b & 0.040 & 0.002\\
        c & 0.047 & 0.003\\
        d & 0.789 & 0.784\\
        G & 4.224 & 6.430\\
        k & 0.635 & 0.406\\
        $\varepsilon_{YFP}(\lambda)/\varepsilon_{CFP}(\lambda)$ & 0.077 & 0.064\\
        \bottomrule[1.5pt]
    \end{tabular}
    \label{tab:lurs_imaging_params}
\end{table}
\fi

\subsection{使用Fretha手动处理数据}
\label{sec:Fretha手动处理数据}
使用Fretha手动处理数据的步骤如下:
\begin{enumerate}
  \item 打开Fretha软件,点击“FRET参数”按钮,按照表 \ref{tab:lurs_imaging_params} 中的参数设置成像参数;
  \item 返回开始界面,点击“浏览”按钮,选择数据文件夹,点击“开始”按钮,进入数据处理界面;
  \item 进入FRET图像处理模块,选择三通道信号背景比大于3且区域均匀的ROI,记录到数据区;
  \item 点击“筛选数据”,排除异常的数据和离群值;
  \item 点击“开始计算”,进入结果展示页界面,点击“保存结果”按钮,保存数据处理结果。
\end{enumerate}

\section{结果}

\subsection{E-FRET和$3^3$-FRET处理结果}
Fretha手动处理数据时选取100个ROI,然后统计Fretha数据区自动计算的E-FRET和$3^3$-FRET结果。
通过ZEN在每个质粒中同样选取了100个ROI,并对应标注了每个ROI的背景值,使用Excel扣除每个背景后编写公式进行E-FRET和$3^3$-FRET定量计算,然后统计结果。

如表 \ref{tab:Fretha手动E-FRET结果} 所示,Fretha手动处理的标准质粒C4Y的效率为$E_A=0.291\pm0.020$,$E_D=0.307\pm0.040$,受体与供体的浓度比为$R_C=0.994\pm0.096$。对比C4Y的文献值$E_A=0.296\pm0.001$,$E_D=0.299\pm0.004$,$R_C=1$,发现使用Fretha手动数据处理方法的结果均与文献值一致。
C10Y、C40Y和C80Y的效率和浓度比也与文献值一致,说明Fretha软件的手动数据处理功能达到了预期目标。
Fretha测量的C4Y、C10Y、C40Y和C80Y的$R_C$的误差分别为0.006、0.008、0.009和0.010,均小于0.01,说明Fretha手动数据处理的精度较高。
\begin{table*}[hbtp]
  \centering
  \caption{手动定量$3^3$-FRET和E-FRET分析标准质粒结果}
  \begin{tabularx}{\linewidth}{
    >{\centering\arraybackslash}p{1cm}
    >{\centering\arraybackslash}X
    >{\centering\arraybackslash}X
    >{\centering\arraybackslash}X
    >{\centering\arraybackslash}X
    >{\centering\arraybackslash}X
    >{\centering\arraybackslash}X
  }
  \toprule[1.5pt]
  \multirow{2}{*}{样本} & \multicolumn{3}{c}{Fretha手动处理结果} & \multicolumn{3}{c}{文献结果} \\
   & $E_{A}$ & $E_{D}$ & ${R_C}$ & $E_A$ & $E_{D}$ & $R_C$ \\
  \midrule
  C4Y  & $0.291\pm0.020$ & $0.307\pm0.040$ & $0.994\pm0.096$ & $0.296\pm0.001$ & $0.299\pm0.004$ & $1$ \\
  C10Y & $0.243\pm0.031$ & $0.230\pm0.022$ & $0.992\pm0.083$ & $0.228\pm0.002$ & $0.223\pm0.003$ & $1$ \\
  C40Y & $0.159\pm0.018$ & $0.155\pm0.011$ & $0.991\pm0.078$ & $0.156\pm0.002$ & $0.158\pm0.002$ & $1$ \\
  C80Y & $0.118\pm0.019$ & $0.117\pm0.012$ & $1.041\pm0.109$ & $0.112\pm0.001$ & $0.116\pm0.002$ & $1$ \\
  \bottomrule[1.5pt]
  \end{tabularx}
  \label{tab:Fretha手动E-FRET结果}
\end{table*}

\subsection{模型质粒FRET双杂交验证实验}
\label{sec:模型质粒FRET双杂交验证实验}

为了模拟FRET双杂交实验,本节使用Fretha手动测量了活的MCF-7细胞中存在自由供体(Cerulean,C)和自由受体(Venus,V)时固定质粒C32V和CVC的$E_{A,max}$、$E_{D,max}$和$n_D/n_A$值。图\ref{fig:Fretha手动双杂交}展示了共表达 C32V / CVC 且含有游离的 C(C32V + C,CVC + C)(上半部分)或游离的 V(C32V + V,CVC + V)(下半部分)的活 MCF7 细胞的三张荧光图像(DD、AA 和 DA)(左侧),使用Fretha手动标注的ROI(中间),以及DC-FRET和L-FRET的结果图(右侧)。
\begin{figure}
  \centering
  \includegraphics[width=1\linewidth]{../figures/3/Fretha手动双杂交数据处理.png}
  \caption{Fretha手动处理的C32V和CVC的FRET双杂交分析结果}
  \label{fig:Fretha手动双杂交}
\end{figure}

DC-FRET数据处理过程中的具体参数设置如下:
C32V,选取$R_C$在0-0.7之间的数据用于斜率拟合$E_{A,max}$,选取$1/R_C$在0-0.8之间的数据用于斜率拟合$E_{D,max}$,
测定的$E_{A,max}$为0.318,$E_{D,max}$为0.297,$n_D/n_A$为1.071;
对CVC,选取$R_C$在0-0.2之间的数据用于斜率拟合$E_{A,max}$,选取$1/R_C$在0-1.2之间的数据用于斜率拟合$E_{D,max}$,
测定的$E_{A,max}$为0.795,$E_{D,max}$为0.393,$n_D/n_A$为2.023。
L-FRET测量得到C32V的$E_{A,max}$为0.347,$E_{D,max}$为0.344,$n_D/n_A$为1.033,CVC的$E_{A,max}$为0.827,$E_{D,max}$为0.419,$n_D/n_A$为2.046。
文献报道的C32V的$E_{D,max}$为0.311,$n_D/n_A$为1,CVC的$E_{D,max}$为0.414,$n_D/n_A$为2。

所有的测试结果如表 \ref{tab:Fretha手动双杂交} 所示。
使用Fretha进行DC-FRET和L-FRET的测量结果均与文献值一致,说明Fretha手动数据处理在准确度上满足预期。
\begin{table*}[htbp]
  \centering
  \caption{手动处理C32V、CVC的FRET双杂交分析结果}
  \begin{tabularx}{\linewidth}{
    >{\centering\arraybackslash}X
    >{\centering\arraybackslash}p{2.2cm}
    >{\centering\arraybackslash}p{2.2cm}
    >{\centering\arraybackslash}p{2.2cm}
    >{\centering\arraybackslash}X
    >{\centering\arraybackslash}X
    >{\centering\arraybackslash}X
    >{\centering\arraybackslash}X
    >{\centering\arraybackslash}X}
    \toprule[1.5pt]
    \multirow{2}{*}{样本} & \multicolumn{3}{c}{DC-FRET结果} & \multicolumn{3}{c}{L-FRET结果} & \multicolumn{2}{c}{文献结果} \\
     & $E_{A,max}$ & $E_{D,max}$ & ${n_D/n_A}$ & $E_{D,max}$ & $E_{D,max}$ & ${n_D/n_A}$ & $E_{D,max}$ & $n_D/n_A$\\
    \midrule
    C32V & $0.318\pm0.036$ & $0.297\pm0.018$ & $1.071\pm0.144$ & $0.347$ & $0.344$ & $1.033$ & 0.311 & 1\\
    CVC & $0.795\pm0.018$ & $0.393\pm0.023$ & $2.023\pm0.113$ & $0.827$ & $0.419$ & $2.046$ & 0.414 & 2\\
    \bottomrule[1.5pt]
    \end{tabularx}
  \label{tab:Fretha手动双杂交}
\end{table*}

\section{Fretha软件测试}
\subsection{成像参数设置模块测试}
成像参数设置是软件的核心功能之一,直接影响到数据处理结果的准确性。
本测试旨在验证软件成像参数设置功能的正确性和灵活性,确保软件能够及时准确响应参数的更新设置。

测试时,通过界面操作更新参数,分别用表 \ref{tab:lurs_imaging_params} CV成像参数和CY参数处理单转CV体系标准质粒C32V和CY体系标准质粒C4Y样本的FRET图像数据,进行E-FRET分析计算结果$E_D$和$R_C$。将参数和数据匹配的实验组记为正面组,将参数不匹配数据的组别记为反面组,测试结果如表 \ref{表:测试参数更新} 所示。

\begin{table}[htbp]
  \centering
  \caption{更改参数C4Y和C32V质粒E-FRET测量结果}
  \begin{tabular}{ccccc}
  \toprule[1.5pt]
  实验组 & 参数 & 样本 & $E_D$ & $R_C$ \\
  \midrule
  \multirow{2}{*}{正面} & CV & C32V & $0.299\pm0.022$ & $0.981\pm0.113$ \\
  & CY & C4Y & $0.307\pm0.040$ & $0.994\pm0.096$ \\
  \multirow{2}{*}{反面} & CY & C32V & $0.382\pm0.054$ & $0.821\pm0.093$ \\
  & CV & C4Y & $0.221\pm0.037$ & $1.313\pm0.135$ \\
  \bottomrule[1.5pt]
  \end{tabular}
  \label{表:测试参数更新}
\end{table}

正面组的测量结果中,C32V的$E_D$为$0.299\pm0.022$,$R_C$为$0.981\pm0.113$,C4Y的$E_D$为$0.307\pm0.040$,$R_C$为$0.994\pm0.096$,均与文献值一致。
反面组的测量结果中,C32V的$E_D$为$0.382\pm0.054$,$R_C$为$0.821\pm0.093$,C4Y的$E_D$为$0.221\pm0.037$,$R_C$为$1.313\pm0.135$,与文献值存在较大偏差。
说明参数Fretha后台能够及时使用更新后的参数进行计算。

经过对界面和计算结果的检查和测试,成像参数设置模块的界面和功能均符合预期,软件界面和后台内存中的参数均能准备设置更新。

\subsection{数据检验模块测试}
如 \ref{sec:数据检验模块} 节所述,数据检验模块用于检查数据的完整性和安全性,防止非法数据对软件运行造成不可预测的影响。
本测试旨在验证软件数据检验模块的有效性,并检查软件对于异常数据的识别能力和隔离能力。

测试时,使用Fretha分别尝试打开FRET标准数据、FRET缺失数据、非FRET数据、空数据、异常数据,检查软件的数据检验模块的反应。
具体测试内容和结果如表 \ref{tab:测试数据完备性} 所示。

\begin{table*}[htbp]
  \centering
  \caption{Fretha数据检验模块测试结果 }
  \begin{tabular}{cccccc}
  \toprule[1.5pt]
  测试数据 & 视野文件夹内容 & 可打开 & 可计算 & 识别类别 & 是否符合预期\\
  \midrule

  \multirow{3}{*}{FRET标准数据} &
  \begin{tabular}[t]{@{}l@{}}
    DD.tif \\
    DA.tif \\
    AA.tif \\
  \end{tabular} &
  \multirow{3}{*}{\ding{51}} &
  \multirow{3}{*}{\ding{51}} &
  \multirow{3}{*}{FRET} &
  \multirow{3}{*}{\ding{51}}\\

  \multirow{2}{*}{FRET缺失数据} &
  \begin{tabular}[t]{@{}l@{}}
    DD.tif \\
    DA.tif \\
  \end{tabular} &
  \multirow{2}{*}{\ding{51}} & 
  \multirow{2}{*}{\ding{55}} &
  \multirow{2}{*}{Unknown} &
  \multirow{2}{*}{\ding{51}} \\

  \multirow{2}{*}{非FRET数据} &
  \begin{tabular}[t]{@{}l@{}}
    D.tif \\
    A.tif \\
  \end{tabular} &
  \multirow{2}{*}{\ding{51}} &
  \multirow{2}{*}{\ding{55}} &
  \multirow{2}{*}{Unknown} &
  \multirow{2}{*}{\ding{51}} \\

  空数据 &
  空 &
  \ding{55} &
  \ding{55} &
  Unknown &
  \ding{51} \\

  \multirow{3}{*}{损坏数据} &
  \begin{tabular}[t]{@{}c@{}}
    DD.tif(损坏) \\
    DA.tif \\
    AA.tif \\
  \end{tabular} &
  \multirow{3}{*}{\ding{51}} &
  \multirow{3}{*}{\ding{55}} &
  \multirow{3}{*}{Unknown} &
  \multirow{3}{*}{\ding{51}} \\
  
  \bottomrule[1.5pt]
  \end{tabular}
  \label{tab:测试数据完备性}
\end{table*}

结果表明,对于正常数据(FRET标准数据),数据检验模块能够正确识别数据类型,并且能够成功打开和计算数据,符合预期。
对于视野子文件夹中存在缺失数据、非FRET数据、空数据和损坏数据等情况,Fretha能够打开并显示在视野区,但无法打开或者参与自动计算,并且会被正常识别为“Unknown”类别以提醒用户该视野处于不可计算的状态。
点击“自动圈点”后,发现自动算法也能够成功忽略被识别为“Unknown”类别的视野,不会对其进行计算。
这些结果均符合预期,说明软件数据检验模块的功能正常,能够有效识别和阻止异常数据的计算。

\subsection{FRET图像处理模块测试}

FRET图像处理模块能够帮助用户进行FRET双杂交分析数据处理,其功能如 \ref{sec:FRET图像处理模块} 节所述。
在测试时,我们主要测试FRET图像处理的视图增强、ROI编辑和ROI状态栏更新功能,以验证软件的功能是否符合预期。

首先测试FRET视图增强功能。
分别切换表 \ref{tab:fretha_viewtype_list} 中的视图类型,查看视图增强的效果,过程中的软件界面和增强视图的截图如图 \ref{fig:视图测试} 所示。
结果表明,软件能够正确显示不同类型的视图,并且视图增强的效果明显,提高了用户在圈选ROI时的针对性和准确性。

\begin{figure*}[!htb]
  \centering
  \includegraphics[width=0.9\linewidth]{../figures/4/4_视图类型.png}
  \caption{Fretha图像处理视图切换测试结果}
  \label{fig:视图测试}
\end{figure*}

然后测试ROI绘制功能。
ROI编辑功能根据鼠标与ROI边框的位置关系,显示不同的鼠标样式。
分别移动鼠标位置到ROI内部、边框和外部的,观察鼠标的样式和按下时对应的功能。
测试结果如表 \ref{tab:ROI鼠标样式} 所示,表明软件能够正确显示不同的鼠标样式,且在不同位置对应的功能正确,响应迅速,保证了ROI绘制的操作体验。

\begin{table}
  \centering
  \caption{ROI绘制功能测试结果}
  \begin{tabular}{cccc}
    \toprule[1.5pt]
    鼠标位置 & 鼠标样式 & 按下效果 & 是否符合预期\\
    \midrule
    ROI内部 & 手型 & 移动ROI & \ding{51}\\
    ROI左边框 & 水平箭头 & 调整ROI宽度 & \ding{51} \\
    ROI右边框 & 水平箭头 & 调整ROI宽度 & \ding{51} \\
    ROI上边框 & 垂直箭头 & 调整ROI高度 & \ding{51} \\
    ROI下边框 & 垂直箭头 & 调整ROI高度 & \ding{51} \\
    ROI外部 & 十字形 & 新建ROI & \ding{51} \\
    \bottomrule[1.5pt]
  \end{tabular}
  \label{tab:ROI鼠标样式}
\end{table}

最后测试ROI状态栏的更新。
ROI状态栏上的数据更新发生在ROI被绘制更新以及视野切换时,因此通过编辑ROI和切换视野两种操作来测试ROI状态栏的更新,然后检查ROI状态栏的信息是否能够实时更新,结果如表 \ref{tab:ROI状态栏测试} 所示。
\begin{table}
  \centering
  \caption{ROI状态栏功能测试结果}
  \begin{tabular}{cccc}
    \toprule[1.5pt]
    \multirow{2}{*}{测试目标} & \multicolumn{2}{c}{ 测试结果} & \multirow{2}{*}{ 是否符合预期} \\
    & 更新ROI结果 & 切换视野结果 & \\
    \midrule
    DD通道信号 & 数据更新 & 重置为0 & \ding{51} \\
    DA通道信号 & 数据更新 & 重置为0 & \ding{51} \\
    AA通道信号 & 数据更新 & 重置为0 & \ding{51} \\
    DD通道SBR & 数据更新 & 重置为0 & \ding{51} \\
    DA通道SBR & 数据更新 & 重置为0 & \ding{51} \\
    AA通道SBR & 数据更新 & 重置为0 & \ding{51} \\
    DD通道背景 & 保持不变 & 数据更新 & \ding{51} \\
    DA通道背景 & 保持不变 & 数据更新 & \ding{51} \\
    AA通道背景 & 保持不变 & 数据更新 & \ding{51} \\
    $F_C$ & 数据更新 & 重置为0 & \ding{51} \\
    $E_D$ & 数据更新 & 重置为0 & \ding{51} \\
    $R_C$ & 数据更新 & 重置为0 & \ding{51} \\
    \bottomrule[1.5pt]
  \end{tabular}
  \label{tab:ROI状态栏测试}
\end{table}

上述测试结果表明,FRET图像处理的视图增强、ROI绘制和ROI状态栏更新功能均符合预期,软件能够正确显示不同类型的视图,辅助用户更好完成FRET双杂交分析图像处理工作。

\subsection{数据管理模块测试}

数据管理模块能够管理和组织数据区记录的数据,包括数据导入、数据导出、数据保存、数据计算等功能,如 \ref{sec:数据管理模块} 节所述。本节重点关注验证数据导入、数据导出和功能。

测试时的具体步骤如下:
\begin{enumerate}
  \item 使用导出数据功能导出圈点信息,检查文件生成和内容是否正确;
  \item 清空数据区,然后使用导入数据功能导入Fretha圈点数据(CSV),检查数据是否正确导入;
  \item 将导出文件另存为其他格式(XLSX),尝试导入XLSX文件,检查数据是否正确导入;
  \item 将另一组数据集生成的CSV文件导入,检查数据是否正确导入;
\end{enumerate}

测试结果如表 \ref{tab:数据管理模块测试结果} 所示。
当导入Fretha圈点数据(CSV),软件能够成功导入数据,数据准确无误。
导入XLSX格式的数据时,软件无法导入。
当导入不匹配Fretha数据(CSV)时,软件能够检测到数据不匹配,终止导入。
导出功能能够成功导出Fretha圈点数据(CSV),数据准确无误。
添加数据和删除数据功能均工作正常,且通过鼠标和快捷键均可以快速添加和删除数据。

\begin{table}[hbtp]
  \centering
  \caption{数据管理模块测试结果}
  \begin{tabular}{p{1.5cm} l l c} % 自定义各列宽度
    \toprule[1.5pt]
    {测试目标} & {测试内容} & {测试结果} & {是否符合预期} \\
    \midrule
    \multirow{2}{*}{清空数据} & 清空数据区 & 成功清空 & \ding{51} \\
     & 快捷键“C” & 成功清空 & \ding{51} \\
    筛选数据 & 筛选数据 & 成功筛选 & \ding{51} \\
    \multirow{3}{*}{导入数据} & 导入Fretha圈点数据(CSV) & 成功导入 & \ding{51} \\
     & 导入其他表格类型数据(XLSX) & 无法导入 & \ding{51} \\
     & 导入不匹配Fretha数据(CSV) & 无法导入 & \ding{51} \\
    导出数据 & 导出Fretha圈点数据(CSV) & 成功导出 & \ding{51} \\
    \multirow{2}{*}{添加数据} & 添加新数据 & 成功添加 & \ding{51} \\
     & 快捷键“A” & 成功添加 & \ding{51} \\
    \multirow{2}{*}{删除数据} & 删除数据 & 成功删除 & \ding{51} \\
     & 快捷键“D” & 成功删除 & \ding{51} \\
    开始计算 & 开始计算数据 & 成功跳转,计算准确 & \ding{51} \\

    \bottomrule[1.5pt]
  \end{tabular}
  \label{tab:数据管理模块测试结果}
\end{table}

\subsection{结果可视化模块测试}

结果可视化和保存模块能够帮助用户查看和保存数据处理结果,包括视图选择、图表结果、结果保存等,如 \ref{sec:结果可视化模块} 节所述。
本节对上述功能进行测试和验证,重点关注两种视图的显示和参数的更新,以及结果的保存功能。

测试L-FRET视图和BIN数据分箱功能,测试结果如图 \ref{fig:L-FRET视图测试} 所示。
L-FRET显示出良好的拟合效果,散点与趋势线符合预期。
测试所用的BIN参数为对$R_C$在$(0,5)$之间的数据按照0.01的步长进行分组并计算平均值,结果显示BIN参数更新后的L-FRET视图散点数明显减少,且趋势线有所变化。
\begin{figure}
  \centering
  \includegraphics[width=0.9\linewidth]{../figures/4/4_L-FRET视图测试.png}
  \caption[Fretha L-FRET视图测试结果]{Fretha L-FRET视图测试结果。图a为标准L-FRET视图,图b为应用BIN数据分箱后的L-FRET视图。}
  \label{fig:L-FRET视图测试}
\end{figure}

测试DC-FRET视图中调整线性拟合的数据范围参数设置功能时,首先设置线性拟合的$R_C$数据范围为$(0,0.5)$,均值拟合的$R_C$范围为$(2,10)$,然后设置线性拟合的$1/R_C$数据范围为$(0,1)$,均值拟合的$1/R_C$范围为$(1,10)$,观察图表中的趋势线和散点图更新。
测试结果如图 \ref{fig:DC-FRET视图测试} 所示,当扩大拟合数据的范围后,可以观察到趋势线的更新,斜率拟合段和均值拟合段均延伸至$R_C$为1的位置。
\begin{figure}
  \centering
  \includegraphics[width=0.9\linewidth]{../figures/4/4_DC-FRET参数测试.png}
  \caption[Fretha DC-FRET视图测试结果]{DC-FRET视图测试结果。图a为设置线性拟合的$R_C$数据范围为$(0,0.5)$,均值拟合的$R_C$范围为$(2,10)$时Fretha生成的趋势线和散点图;
  图b则设置线性拟合的$1/R_C$数据范围为$(0,1)$,均值拟合的$1/R_C$范围为$(1,10)$。}
  \label{fig:DC-FRET视图测试}
\end{figure}

测试保存的数据和报告是否能够正确保存,结果如图 \ref{fig:结果保存测试} 所示,软件能够正确生成结果文件,且数据和报告的内容准确无误,符合预期。
\begin{figure}[hbtp]
  \centering
  \includegraphics[width=0.6\linewidth]{../figures/2/2_保存数据.png}
  \caption{Fretha结果保存测试结果}
  \label{fig:结果保存测试}
\end{figure}

综合测试结果表明,Fretha结果可视化模块的功能正常,能够正确显示和保存数据处理结果,为用户提供了直观的数据分析和保存功能。

\section{Fretha稳定性测试}
稳定性测试旨在评估 Fretha 软件在长时间运行、高负载或异常操作下的可靠性。测试从:
\begin{enumerate}
  \item 压力测试:同时加载 10 个大型 FRET 数据集(每个数据集包含 50 个视野),连续执行参数设置、图像处理和结果计算操作,监测软件是否出现崩溃或内存泄漏。
  \item 长时间运行测试:保持软件连续运行 48 小时,期间每隔8小时进行数据导入、数据计算和保存,然后清空数据。记录软件内存占用情况,检测软件是否存在内存泄露
  \item 异常操作测试:快速重复进行参数切换、数据导入导出、ROI 编辑等操作,模拟用户高频使用场景,观察软件的响应稳定性。
\end{enumerate}

测试结果如表 \ref{tab:稳定性测试} 所示,Fretha 在压力测试下仍能稳定处理数据,未出现崩溃或内存溢出;长时间运行中的内存占用监测如图 \ref{fig:48小时内存变化} 所示,功能模块保持正常,资源占用无显著异常;高频操作下软件响应迅速,未出现卡顿或错误。这些结果证明 Fretha 具有良好的稳定性,能够满足科研和工业场景的长时间、高负载使用需求。
\begin{figure}[hbtp]
  \centering
  \includegraphics[width=0.7\linewidth]{../figures/4/4_48小时内存占用变化.png}
  \caption{Fretha软件48小时运行内存占用监测}
  \label{fig:48小时内存变化}
\end{figure}

\begin{table}[hbtp]
  \centering
  \caption{Fretha 软件稳定性测试结果}
  \label{tab:稳定性测试}
  \begin{tabular}{lp{5cm}p{5cm}}
  \toprule[1.5pt]
  {测试类型}         & {测试方法}                                                                 & {测试结果}                                                                 \\
  \midrule 
  \multirow{5}{*}{压力测试} 
    & \multirow{5}{5cm}{同时加载 10 个大型 FRET 数据集(每个数据集包含 50 个视野),连续执行参数设置、图像处理和结果计算操作,监测软件是否出现崩溃或内存泄漏。} 
    & - 处理 10 个数据集总耗时:42.5 分钟 \\
    &                                                                                
    & - 内存峰值占用:1.4 GB(稳定无泄漏) \\
    &                                                                                
    & - 操作成功率:100\% \\
  \midrule % 测试类型分隔线
  \multirow{5}{*}{长时间运行测试} 
    & \multirow{5}{5cm}{保持软件连续运行 48 小时,期间定期检查各功能模块(如图像处理、数据导出)是否正常工作,记录 CPU 和内存占用情况。} 
    & - CPU 平均使用率:22-25\%(波动<3\%) \\
    &                                                                                
    & - 内存平均占用:550-600 MB(无持续增长) \\
    &                                                                                
    & - 数据导出成功率:100\% \\
  \midrule % 测试类型分隔线
  \multirow{4}{*}{异常操作测试} 
    & \multirow{4}{5cm}{快速重复进行参数切换、数据导入导出、ROI 编辑等操作,模拟用户高频使用场景,观察软件的响应稳定性。} 
    & - 每秒操作次数:15-20 次/秒 \\
    &                                                                                
    & - 平均响应时间:0.3-0.5 秒 \\
    &                                                                                
    & - 未出现卡顿或界面冻结 \\
    \\
  \bottomrule[1.5pt]
  \end{tabular}
\end{table}

\section{本章小结}

本章对Fretha软件的各功能模块进行了全面的测试与验证,包括成像参数设置模块、数据检验模块、FRET图像处理模块、数据管理模块、结果可视化模块以及自动算法性能和软件稳定性测试。
测试结果表明,Fretha软件在功能实现、性能表现和稳定性方面均达到了预期目标。
成像参数设置模块、数据检验模块、FRET图像处理模块、数据管理模块、结果可视化模块等主要模块的正常运行支持了FRET双杂交分析数据处理的所有步骤的用户需求,保证了数据处理过程的安全和高效。

此外,LURS算法在处理速度、内存效率和硬件适应性方面展现出显著优势,为高通量实时分析提供了高效的解决方案。
稳定性测试进一步验证了Fretha在长时间运行、高负载和高频操作下的可靠性,未出现崩溃或性能下降的情况。
综合测试结果表明,Fretha软件功能完善、性能优越、稳定性良好,能够满足复杂科研和实际应用的需求。
\chapter{软件的应用和测试}

\section{引言}
\ifshowtext

Bcl-XL 作为抗凋亡蛋白家族核心成员,通过结合促凋亡蛋白 Bak/Bax 抑制线粒体凋亡通路\upcite{Adams1998The}。
其异常高表达与肿瘤耐药性密切相关,已被证实与多种癌症治疗抵抗机制直接关联\upcite{2006Mitochondrial}。
A1331852 作为新一代 BH3 模拟物,以皮摩尔级亲和力特异性靶向 Bcl-XL 疏水口袋,通过竞争性结合解除其对 Bak 的束缚,可显著削弱 Bcl-xL 与 Bak 的结合比例。
该变化可通过实时荧光共振能量转移信号定量检测\upcite{2006Mitochondrial, 2017New}。

本章我们应用我们开发的FRET双杂交分析数据处理软件,分析Bcl-xL和Bak互作时加入Bcl-xL抑制剂A1331852前后的互作变化情况\upcite {2020Stoichiometry}。
针对自动数据处理算法,我们对比了基于深度学习的自动数据处理方法,包括交互式医学图像分析软件ilastik和面向2D医学图像特化的分割大模型SAM\_Med2D,结果发现,计算结果也更接近手动处理的数据结果,速度最优,系统硬件配置要求也最低。
最后,在数据处理过程中,我们对软件的稳定性和功能进行验证,结果表明,Fretha软件运行稳定,所有功能符合预期。

\fi

\section{材料与方法}

\subsection{细胞培养和转染}
质粒方面,CFP-Bcl-XL 质粒由 A.P.Gilmore提供 ,YFP-Bak 质粒的构建方法此前已有报道 \upcite{warren2019bcl, sun2023regorafenib}。
细胞培养和转染条件如章节\ref{sec:细胞转染}所描述。
成像系统的参数如章节\ref{sec:成像条件}所描述,在成像时,每个通道的成像曝光时间为300ms。
药物 A1331852 购自美国新泽西州的 MedChemExpress(MCE)公司。

\subsection{数据处理操作}
在数据处理方法上,我们首先应用Fretha分别进行了手动数据处理和自动数据处理。然后我们与深度学习方法进行了自动数据处理,主要包括交互式深度学习医学图像处理软件ilastik和面向医学图像的分割大模型SAM\_Med2D,以和我们的自动处理方法进行对比。应用各种不同方法进行数据处理的具体做法如下:
\begin{enumerate}
    \item Fretha手动数据处理。首先将数据格式设定为Fretha匹配的数据结构,然后通过Fretha软件打开数据,在FRET图像处理圈点界面进行手动ROI选取,固定ROI大小为$5\times 5$,每个视野选取10至20个标定在细胞上的ROI,将数据记录到Fretha上。检查数据无误后,点击“开始计算”,获得手动FRET双杂交分析的实验结果。
    \item 基于Fretha的自动数据处理。导入数据后,点击软件界面上的“自动圈点”按钮,执行LURS算法自动在每个视野选取ROI,并记录数据到后台。检查数据无误后,点击“开始计算”,获得自动FRET双杂交分析的实验结果。
    \item 基于ilastik的自动数据处理。首先,我们从相同数据集中抽取随机视野,手动标注其中的好细胞、坏细胞、背景区域。取其中识别到的背景区域灰度均值作为背景灰度值,选取好细胞的区域灰度均值作为信号灰度值。使用Fretha内嵌的FRET定量计算功能和FRET双杂交分析求解功能,计算得到基于ilastik的自动分析的实验结果。
    \item 基于SAM\_Med2D的自动数据处理。首先,我们从相同数据集中抽取随机视野,手动标注人工圈点的ROI作为信号ROI的prompt,手动标注背景区域的ROI作为背景的prompt,将信号prompt和背景prompt输入SAM\_Med2D模型,使得大模型根据FRET图像特点进行能力特化,从而进行自动的数据处理。使用Fretha内嵌的FRET定量计算功能和FRET双杂交分析求解功能,计算得到基于SAM\_Med2D的自动分析的实验结果。
\end{enumerate}
为了保证不同方法在FRET定量分析和FRET双杂交分析计算的一致性,我们在维护Fretha的源码上新建了特殊的功能分支,并在软件中实现了根据输入图片和标注ROI的掩码二值化图片进行自动FRET计算功能。

我们应用FRET双杂交分析技术对照组(Control)和加药组(Medication)分别进行了FRET双杂交分析,以测量在加入药物A1331852前后Bcl-XL和Bak在细胞中相互作用的结合情况变化。

\section{结果}

\subsection{应用FRET双杂交技术分析Bcl-XL和Bak互作}

\section{软件的测试}
软件测试是确保软件质量和性能的重要环节。本部分将对软件的各项功能和性能进行全面测试,以验证其是否满足设计要求和用户需求。

\subsection{成像参数设置测试}
成像参数设置是软件的核心功能之一,它直接影响到数据处理结果的准确性。本测试旨在验证软件成像参数设置功能的正确性和灵活性,确保软件能够及时准确响应参数的更新设置。

通过界面操作更新参数,分别用CV成像参数和CY参数处理单转标准质粒C32V和C4Y样本的FRET图像数据,将数据导出后,检查E-FRET计算结果$E_D$和$R_C$。具体测试步骤如下:
\begin{enumerate}
    \item 打开软件,进入成像参数设置界面。
    \item 使用表\ref{tab:lurs_imaging_params}中CV质粒成像的参数进行设置,然后处理单独转染C32V质粒样本数据和单独转染C4Y质粒样本数据。
    \item 退出参数设置界面,再次进入参数设置界面。
    \item 使用表\ref{tab:lurs_imaging_params}中CY质粒成像的参数进行设置,然后处理单独转染C32V质粒样本数据和单独转染C4Y质粒样本数据。
    \item 关闭软件,重新打开软件,然后进入参数设置界面。
    \item 记录两次数据处理的结果。
\end{enumerate}

\begin{table*}[htbp]
    \centering
    \caption{ 切换参数对C32V质粒和C4Y质粒的E-FRET分析结果}
    \begin{tabularx}{\linewidth}{
    >{\centering\arraybackslash}X
    >{\centering\arraybackslash}X
    >{\centering\arraybackslash}X
    >{\centering\arraybackslash}X
    >{\centering\arraybackslash}X
    >{\centering\arraybackslash}X
    >{\centering\arraybackslash}X}
    \toprule
    \multirow{2}{*}{参数} & \multicolumn{2}{c}{C32V} & \multicolumn{2}{c}{C4Y} \\
    & $E_{D}$ & ${R_C}$ & $E_{D}$ & $R_C$ \\
    \midrule
    CV体系参数 & $0.29\pm0.02$ & $0.98\pm0.11$ & $0.22\pm0.03$ & $1.31\pm0.13$  \\
    CY体系参数 & $0.38\pm0.05$ & $0.82\pm0.09$ & $0.30\pm0.02$ & $1.02\pm0.12$  \\
    \bottomrule
    \end{tabularx}
    \label{表:测试参数更新}
\end{table*}

从软件界面上,在设置参数后重新进入参数设置界面,发现参数符合预期的新参数。在软件重启后,发现界面显示的参数也与最近一次更新的参数一致。

两次E-FRET分析的结果如表\ref{表:测试参数更新}所示。在计算功能上,在更新了CY成像参数后,处理C32V质粒计算得到的$E_D$和$R_C$结果分别为0.38和0.82,处理C4Y质粒,计算得到的$E_D$和$R_C$结果分别为0.30和1.02;
在使用CV参数处理C4Y质粒时,计算得到的$E_D$和$R_C$结果分别为0.22和1.31,处理C32V质粒时,计算得到的$E_D$和$R_C$结果分别为0.29和0.98。文献报道的C4Y质粒和C32V质粒的$E_D$为0.30,$R_C$结果为1。可以发现两次更新参数后,对应正确的质粒结果符合文献值,而不匹配的质粒的测量结果存在较大偏差。

经过以上测试,我们验证了成像参数设置模块的界面和功能均符合预期,参数设置响应准确。

\subsection{数据导入导出测试}
数据导入导出功能是软件与其他系统进行数据交互的重要手段。本测试旨在验证软件数据导入导出功能的正确性和兼容性。
确保软件能够正确地导入和导出支持的CSV数据文件,并且数据的完整性和准确性得到保证。

准备不同格式的测试数据文件,分别进行导入和导出操作,然后检查导入和导出的数据是否一致。
具体测试步骤如下:
\begin{enumerate}
    \item 准备包含不同类型数据的 CSV、Excel、JSON 文件。
    \item 打开软件,选择数据导入功能,依次导入上述测试文件,检查导入的数据是否正确显示。
    \item 对导入的数据进行一些修改和处理,然后选择数据导出功能,将数据导出为相同格式的文件。
    \item 比较原始文件和导出文件的数据内容,确保数据的完整性和准确性。
\end{enumerate}

测试结果表明,软件能够正确地导入和导出 CSV、Excel、JSON 等格式的数据文件,并且数据的完整性和准确性得到了保证。但在导入大型文件时,软件的导入速度较慢,需要进行优化。

\subsection{结果保存测试}
结果保存功能是软件的重要功能之一,它能够帮助用户保存分析结果和处理数据。本测试旨在验证软件结果保存功能的正确性和可靠性。

确保软件能够正确地保存各种类型的结果文件,如文本文件、图像文件、报告文件等,并且保存的文件能够被正确打开和查看。

在软件中进行各种操作,生成不同类型的结果文件,然后选择结果保存功能,将结果保存到指定的文件夹中。最后,检查保存的文件是否能够被正确打开和查看。具体测试步骤如下:
\begin{enumerate}
    \item 在软件中进行数据分析和处理,生成文本结果、图像结果和报告结果。
    \item 选择结果保存功能,分别将上述结果保存为文本文件、图像文件和报告文件。
    \item 打开保存的文件,检查文件内容是否与软件中显示的结果一致。
\end{enumerate}

经过测试,发现软件能够正确地保存各种类型的结果文件,并且保存的文件能够被正确打开和查看。但在保存文件时,软件没有提供文件覆盖提示功能,可能会导致用户误操作。

\subsection{软件的性能测试}
软件的性能是衡量软件质量的重要指标之一。本测试旨在评估软件在不同负载条件下的性能表现,如响应时间、吞吐量等。

确定软件在正常使用和高负载情况下的性能指标,为软件的优化和升级提供依据。

使用性能测试工具,模拟不同的用户负载,对软件的各项性能指标进行监测和分析。具体测试步骤如下:
\begin{enumerate}
    \item 选择合适的性能测试工具,如 JMeter、LoadRunner 等。
    \item 设计性能测试场景,包括不同的用户并发数、操作频率等。
    \item 运行性能测试,记录软件的响应时间、吞吐量、CPU 使用率等性能指标。
    \item 分析测试结果,找出性能瓶颈和问题所在。
\end{enumerate}

测试结果显示,软件在正常使用情况下性能表现良好,但在高负载情况下,响应时间明显增加,吞吐量下降。需要对软件进行优化,提高其性能和稳定性。

\subsection{软件的稳定性测试}
软件的稳定性是指软件在长时间运行过程中保持正常工作的能力。本测试旨在验证软件在长时间运行过程中的稳定性和可靠性。

确保软件在长时间运行过程中不会出现崩溃、死机等异常情况,保证软件的正常使用。

让软件连续运行一段时间,模拟实际使用场景,观察软件的运行状态和性能表现。具体测试步骤如下:
\begin{enumerate}
    \item 打开软件,进行一些基本的操作,如登录、数据查询等。
    \item 让软件连续运行 24 小时以上,期间不断进行各种操作,如数据导入、导出、分析等。
    \item 观察软件的运行状态,记录是否出现崩溃、死机、数据丢失等异常情况。
    \item 分析测试结果,评估软件的稳定性和可靠性。
\end{enumerate}

经过长时间的稳定性测试,发现软件在大部分时间内运行稳定,但偶尔会出现卡顿现象。需要对软件进行进一步的优化和调试,提高其稳定性。

\section{本章小结}
\chapter{总结与展望}

\section{本文的主要内容和总结}
本文围绕 FRET 双杂交分析数据处理的需求和难点,研发了一款FRET双杂交分析数据处理软件 Fretha。针对FRET图像处理时人工标注ROI的难点,分析了高质量ROI在明度和变异系数上的特点,提出了基于明度和均匀度的 ROI 选取算法LURS,显著提升了数据处理的效率和准确性。
本文的研究内容主要包括以下三个方面:

(1)根据FRET双杂交分析数据处理的流程和需求,研发了一款FRET双杂交分析数据处理软件Fretha。
完成了Fretha的分层架构设计、模块功能划分、界面设计、模块开发实现等工作。
通过对基本数据模型。
本文实现了成像参数设置模块和数据检验模块,可有效保证数据处理的前置条件和信息检验。
FRET图像处理模块能够切换增强视图类型,辅助ROI状态栏,优化了标注ROI时的信息反馈,并且设置了ROI的自动化选取功能,提升了数据处理的效率。
数据管理模块提供了数据筛选、数据导入导出、一键计算等功能,便于用户对数据进行管理和分析,实现了对数据模型的统一管理和在模块间的高效传递。
结果可视化模块提供了数据可视化、结果导出等功能,便于用户对数据进行分析和展示。
Fretha体现出针对FRET双杂交分析数据处理的专业性和实用性,提高了FRET双杂交分析数据处理的规范化和自动化。

(2)提出了基于明度和均匀度的 ROI 选取算法 LURS。
通过多通道自适应阈值分割和变异系数均匀性评估,实现了 ROI 的自动化选取。
LURS算法通过筛选每个ROI在局部的相对亮度和变异系数,有效排除低亮度背景和灰度突变区域。
在标准质粒C4Y、C10Y、C40Y、C80Y的E-FRET测量和$3^3$-FRET测量中,$E_{A}$和$E_{D}$值与文献报道误差小于5\%,$R_{C}$偏差不超过0.05。
在C32V和CVC模型质粒的FRET双杂交验证中,测量得到的C32V中C与V的结合计量比为1.06,在CVC中则为1.90,符合文献报道的结果。
引用LURS算法测量对照组细胞中Bcl-xL-Bak相互作用的化学计量比为1.87,加药A1331852处理组则为1.12,化学计量比测量结果与手动分析高度一致。
对比基于SAM-Med2D和ilastik的算法,LURS表现出良好的稳定性和鲁棒性,适用于高通量药物筛选等大数据量场景。

(3)对Fretha软件进行了系统性测试与评估,完成了对Fretha软件的功能测试、性能测试和稳定性测试。
Fretha的参数设置模块响应及时,能够成功更新成像参数,确保数据计算的准确性。
数据检验模块可以识别并隔离异常数据,确保软件运行的安全稳定。
FRET图像处理模块的图像显示正常,各个控件的操作响应及时,能够快速标注绘制ROI,更新ROI状态栏信息。
数据管理模块能够成功导入导出数据,支持一键计算功能,能够快速完成数据的计算和结果的导出。
结果可视化模块展示了FRET双杂交分析视图,支持多种格式的结果导出,便于用户对数据进行分析和展示。
性能测试表明,Fretha中LURS算法在1.4GB数据集上单ROI处理时间仅6.6 ms,内存占用800 MB,优于其他软件工具如ilastik(35.2 ms / 1.8 GB)和SAM-Med2D(50.7 ms / 14 GB)。
稳定性测试表明,测试了Fretha软件在48小时连续运行和高频操作下保持稳定,资源占用无显著波动。

\section{展望}
本文研发的FRET双杂交分析数据处理软件Fretha实现了规范化、自动化的FRET双杂交分析数据处理,提升了数据处理的效率和准确性。
Fretha的推出为活细胞FRET双杂交分析技术提供了标准化、自动化的解决方案,有望推动该技术在精准医疗和药物研发中的大规模应用。
尽管Fretha及LURS算法已实现 FRET 双杂交分析的自动化处理,面对未来可能的大数据量数据处理需求,今后的工作可以围绕以下几个方面进行改进和完善:

(1)集成LURS算法到显微镜在线成像系统。
本文提出的LURS算法在性能上已经达到了实时数据处理的实验要求,但是目前还需要将显微镜成像系统中的数据通过数据检验模块导入Fretha中离线处理。
未来可以考虑将LURS算法集成到显微镜成像系统中,实现实时成像和数据处理,研发智能化的在线成像系统。
在线成像系统可以实时监测细胞的荧光信号变化,自动选取ROI并进行数据处理,实时输出FRET双杂交分析结果。
从而加大显微镜通量,进一步提高FRET双杂交实验的效率。

(2)完善Fretha功能,适配更多细胞图像分析方法。
Fretha 目前支持 E-FRET、$3^3$-FRET、L-FRET和DC-FRET 等分析方法。
未来可根据用户需求,增加更多的分析方法和功能模块,如凋亡分析、细胞周期分析等。
Fretha 还可以通过引入开源深度学习推理库,如 OpenVINO、TensorRT 等,集成深度学习算法,丰富自动数据处理的算法库。
参考ilastik的模块化设计,后续可以将不同的分析方法和功能模块集成到Fretha中,用户可以自行选择需要的模块进行数据处理。

(3)搭建数据库和云服务平台。
Fretha 目前仅支持本地数据管理,未来可考虑搭建云服务平台,实现数据的在线存储、共享和分析,提高数据的可访问性和可用性。
Fretha的自动数据处理庞大的药物筛选数据也可以通过云服务进行分析,提高数据的处理效率。

通过持续迭代,Fretha 将逐步发展为集高效性、准确性和易用性于一体的 FRET 数据分析平台,为蛋白质互作研究和药物开发提供更强大的技术支持。

\backmatter
% 参考文献
\cleardoublepage
\pagestyle{emptypage}
\renewcommand{\chapterlabel}{\bibname} % 设置参考文献的页眉
\bibliographystyle{bstutf8}
\bibliography{ref/refs}
% \newpage
% \thispagestyle{plain}
% \mbox{}
% 附录
% \appendix
% \backmatter
% \input{data/appendix01}
% \input{data/appendix02}

% 致谢
% \cleardoublepage
% \renewcommand{\chapterlabel}{\ackname} % 设置参考文献的页眉
% \begin{ack}
三年研究生时光如白驹过隙,回首这段充实而难忘的科研岁月,心中满是感激与不舍。在此,谨向所有帮助过我的师长、亲人和朋友致以最诚挚的谢意。

首先,衷心感谢我的导师陈同生教授与胡敏老师。在课题研究与论文撰写过程中,两位老师始终以渊博的学识、严谨的治学态度给予我悉心指导。陈同生教授追求卓越的科研精神、高瞻远瞩的学术视野,以及胡敏老师精益求精的实践经验,不仅帮助我攻克了研究中的诸多难题,更教会我以严谨务实的态度探索科学真理。老师们对科研的热忱与执着,成为我不断前进的精神动力,这份师恩将铭记于心。

感恩我的家人,是你们毫无保留的支持与包容,让我能够心无旁骛地投入学术研究。每当遇到挫折,你们温暖的鼓励与理解,都是我最坚实的精神支柱。这份跨越山海的牵挂与爱意,是我逐梦路上最珍贵的力量源泉。

最后,感谢朝夕相处的同门师兄弟姐妹们。实验室里的热烈讨论、实验台前的并肩奋战,以及生活中的欢声笑语,共同构成了这段难忘的青春记忆。你们的友谊与支持,让科研之路不再孤单,也为我的研究生生活增添了无数温暖与感动。

谨以此文献给所有给予我帮助的人,愿未来继续以感恩之心,奔赴新的征程。
\end{ack}

% \newpage
% \thispagestyle{plain}
% \mbox{}

% 作者攻读学位期间发表的学术论文目录
\cleardoublepage
% \begin{resume}

  \section*{发表的学术论文} % 发表的和录用的合在一起

  \begin{enumerate}[{[}1{]}]
  \addtolength{\itemsep}{-.36\baselineskip}%缩小条目之间的间距,下面类似
  % \item An automatic design framework for passive analog circuits using Decision chain Model based on dynamic Conditional policy gradient
  \item 许莉芬, 曹霑懋, 郑明杰, 肖博健. 基于用户权威度和多特征融合的微博谣言检测模型[J]. 计算机工程与科学. 2024,  46(4): 752-760.
    
    \item 许莉芬, 曹霑懋, 刘聪. 融合转发用户特征的微博谣言检测模型(投稿至中文信息学报,已完成第三次返修)

  \item 肖博健, 曹霑懋, 许莉芬. 多任务学习用于不良言论与个体特征检测(已被计算机系统应用录用)


\end{enumerate}
\end{resume}
% \renewcommand{\chapterlabel}{\innoname} % 设置创新性说明页眉
\begin{inno}

\begin{enumerate}[label={[\arabic*]}]
  \item 本论文第四章对应研究成果已经发表在Journal of Biophotonics;(SCI三区,第一作者)
  \item 本论文第二章对应研究成果已经获批软件著作权一件。(已授权)
\end{enumerate}

\end{inno}

\end{document}
%%
