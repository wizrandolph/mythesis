\chapter{Fretha的测试}

\section{引言}
软件测试是确保软件质量和性能的重要环节。
结合Fretha的应用实验,我们对软件的功能模块进行了测试,包括成像参数设置模块、数据检验模块、FRET图像处理模块、数据管理模块、结果可视化模块等。
我们对集成在Fretha中的自动圈点算法LURS进行了性能测试,并对比了SAM-Med2D、ilastik等其他自动荧光信号提取的算法。
针对系统的稳定性,测试了软件在长时间运行和高频操作下的表现。
综合测试结果表明,Fretha在功能丰富度、性能表现和稳定性方面均达到了预期目标,能够满足复杂科研和高通量应用需求。

\section{材料与方法}
\subsection{细胞培养与转染}
\label{sec:细胞质粒}
标准质粒实验中,中国科学院细胞库提供的MCF-7细胞株在含有10\%胎牛血清、100 U/mL青霉素和100 $\mu$g/mL链霉素的DMEM培养基中培养。
在Bcl-xL-Bak药效检测实验中,将MCF-7细胞(4000个细胞/孔)接种于含DMEM培养基和10\%胎牛血清的96孔板(中国LABSELECT公司),置于37$^\circ \text{C}$、5\% $\text{CO}_2$培养箱中孵育12小时。每孔转染400 ng质粒,并以3:1或1:3的比例形成转染复合物。

Turbofect\texttrademark{}转染试剂购自美国赛默飞世尔科技公司。所有质粒均由Steven Vogel赠送:C17V(Addgene质粒\# 26395)、C32V(Addgene质粒\# 26396)、mVenus N1(Addgene质粒\# 27793)、mCerulean C1(Addgene质粒\# 27796)、CVC(Addgene质粒\# 27809)\upcite{koushik2006cerulean,thaler2005quantitative}。CFP(青色荧光蛋白)-YFP(黄色荧光蛋白)二聚体质粒包括YFP-G4-CFP(C4Y)、YFP-G10-CFP(C10Y)、YFP-G40-CFP(C40Y)和YFP-G80-CFP(C80Y),由Christian Wahl-Schott赠送 \upcite{butz2016}。CFP-Bcl-xL质粒由A. P. Gilmore提供 \upcite{warren2019bcl},YFP-Bak质粒的构建方法已先前报道过 \upcite{sun2023regorafenib}。药物A1331852购自美国新泽西州MCE公司。

\subsection{FRET成像系统}
\label{sec:成像条件}
\ifshowtext
本研究中,所有实验数据均使用自主研发的多模态FRET自动化成像系统获取 \upcite{sun2022automated}。
对于CY标准质粒,实验选用了20倍的0.45NA物镜(Olympus,日本)和6\%光照强度。
模型质粒实验,选用了20倍的0.45NA物镜(Olympus,日本)和50\%光照强度。
实验过程中,在AA通道寻找视野,然后依次捕获AA、DA和DD通道的荧光图像。

对于CV质粒,串扰因子$a$和$b$通过单转Venus质粒测量,串扰因子$c$和$d$通过单转Cerulean质粒测量,系统校正因子$G$和$k$和$\varepsilon_{YFP}(\lambda)/\varepsilon_{CFP}(\lambda)$是由标准质粒C17V和C32V测量。

对于CY质粒,串扰因子$a$和$b$通过单转YFP质粒测量,串扰因子$c$和$d$通过单转CFP质粒测量,系统校正因子$G$和$k$和$\varepsilon_{YFP}(\lambda)/\varepsilon_{CFP}(\lambda)$是由标准质粒C4Y、C10Y、C40Y和C80Y质粒测量。
所有的FRET成像参数如表 \ref{tab:lurs_imaging_params} 所示。

\begin{table}[htbp]
    \centering
    \caption{FRET成像系统参数}
    \begin{tabular}{ccc}
        \toprule[1.5pt]
        参数名 & CV质粒成像参数 & CY质粒成像参数 \\
        \midrule
        a & 0.206 & 0.160\\
        b & 0.040 & 0.002\\
        c & 0.047 & 0.003\\
        d & 0.789 & 0.784\\
        G & 4.224 & 6.430\\
        k & 0.635 & 0.406\\
        $\varepsilon_{YFP}(\lambda)/\varepsilon_{CFP}(\lambda)$ & 0.077 & 0.064\\
        \bottomrule[1.5pt]
    \end{tabular}
    \label{tab:lurs_imaging_params}
\end{table}
\fi

\subsection{使用Fretha手动处理数据}
\label{sec:Fretha手动处理数据}
使用Fretha手动处理数据的步骤如下:
\begin{enumerate}
  \item 打开Fretha软件,点击“FRET参数”按钮,按照表 \ref{tab:lurs_imaging_params} 中的参数设置成像参数;
  \item 返回开始界面,点击“浏览”按钮,选择数据文件夹,点击“开始”按钮,进入数据处理界面;
  \item 进入FRET图像处理模块,选择三通道信号背景比大于3且区域均匀的ROI,记录到数据区;
  \item 点击“筛选数据”,排除异常的数据和离群值;
  \item 点击“开始计算”,进入结果展示页界面,点击“保存结果”按钮,保存数据处理结果。
\end{enumerate}

\section{结果}

\subsection{E-FRET和$3^3$-FRET处理结果}
Fretha手动处理数据时选取100个ROI,然后统计Fretha数据区自动计算的E-FRET和$3^3$-FRET结果。
通过ZEN在每个质粒中同样选取了100个ROI,并对应标注了每个ROI的背景值,使用Excel扣除每个背景后编写公式进行E-FRET和$3^3$-FRET定量计算,然后统计结果。

如表 \ref{tab:Fretha手动E-FRET结果} 所示,Fretha手动处理的标准质粒C4Y的效率为$E_A=0.291\pm0.020$,$E_D=0.307\pm0.040$,受体与供体的浓度比为$R_C=0.994\pm0.096$。对比C4Y的文献值$E_A=0.296\pm0.001$,$E_D=0.299\pm0.004$,$R_C=1$,发现使用Fretha手动数据处理方法的结果均与文献值一致。
C10Y、C40Y和C80Y的效率和浓度比也与文献值一致,说明Fretha软件的手动数据处理功能达到了预期目标。
Fretha测量的C4Y、C10Y、C40Y和C80Y的$R_C$的误差分别为0.006、0.008、0.009和0.010,均小于0.01,说明Fretha手动数据处理的精度较高。
\begin{table*}[hbtp]
  \centering
  \caption{手动定量$3^3$-FRET和E-FRET分析标准质粒结果}
  \begin{tabularx}{\linewidth}{
    >{\centering\arraybackslash}p{1cm}
    >{\centering\arraybackslash}X
    >{\centering\arraybackslash}X
    >{\centering\arraybackslash}X
    >{\centering\arraybackslash}X
    >{\centering\arraybackslash}X
    >{\centering\arraybackslash}X
  }
  \toprule[1.5pt]
  \multirow{2}{*}{样本} & \multicolumn{3}{c}{Fretha手动处理结果} & \multicolumn{3}{c}{文献结果} \\
   & $E_{A}$ & $E_{D}$ & ${R_C}$ & $E_A$ & $E_{D}$ & $R_C$ \\
  \midrule
  C4Y  & $0.291\pm0.020$ & $0.307\pm0.040$ & $0.994\pm0.096$ & $0.296\pm0.001$ & $0.299\pm0.004$ & $1$ \\
  C10Y & $0.243\pm0.031$ & $0.230\pm0.022$ & $0.992\pm0.083$ & $0.228\pm0.002$ & $0.223\pm0.003$ & $1$ \\
  C40Y & $0.159\pm0.018$ & $0.155\pm0.011$ & $0.991\pm0.078$ & $0.156\pm0.002$ & $0.158\pm0.002$ & $1$ \\
  C80Y & $0.118\pm0.019$ & $0.117\pm0.012$ & $1.041\pm0.109$ & $0.112\pm0.001$ & $0.116\pm0.002$ & $1$ \\
  \bottomrule[1.5pt]
  \end{tabularx}
  \label{tab:Fretha手动E-FRET结果}
\end{table*}

\subsection{模型质粒FRET双杂交验证实验}
\label{sec:模型质粒FRET双杂交验证实验}

为了模拟FRET双杂交实验,本节使用Fretha手动测量了活的MCF-7细胞中存在自由供体(Cerulean,C)和自由受体(Venus,V)时固定质粒C32V和CVC的$E_{A,max}$、$E_{D,max}$和$n_D/n_A$值。图\ref{fig:Fretha手动双杂交}展示了共表达 C32V / CVC 且含有游离的 C(C32V + C,CVC + C)(上半部分)或游离的 V(C32V + V,CVC + V)(下半部分)的活 MCF7 细胞的三张荧光图像(DD、AA 和 DA)(左侧),使用Fretha手动标注的ROI(中间),以及DC-FRET和L-FRET的结果图(右侧)。
\begin{figure}
  \centering
  \includegraphics[width=1\linewidth]{../figures/3/Fretha手动双杂交数据处理.png}
  \caption{Fretha手动处理的C32V和CVC的FRET双杂交分析结果}
  \label{fig:Fretha手动双杂交}
\end{figure}

DC-FRET数据处理过程中的具体参数设置如下:
C32V,选取$R_C$在0-0.7之间的数据用于斜率拟合$E_{A,max}$,选取$1/R_C$在0-0.8之间的数据用于斜率拟合$E_{D,max}$,
测定的$E_{A,max}$为0.318,$E_{D,max}$为0.297,$n_D/n_A$为1.071;
对CVC,选取$R_C$在0-0.2之间的数据用于斜率拟合$E_{A,max}$,选取$1/R_C$在0-1.2之间的数据用于斜率拟合$E_{D,max}$,
测定的$E_{A,max}$为0.795,$E_{D,max}$为0.393,$n_D/n_A$为2.023。
L-FRET测量得到C32V的$E_{A,max}$为0.347,$E_{D,max}$为0.344,$n_D/n_A$为1.033,CVC的$E_{A,max}$为0.827,$E_{D,max}$为0.419,$n_D/n_A$为2.046。
文献报道的C32V的$E_{D,max}$为0.311,$n_D/n_A$为1,CVC的$E_{D,max}$为0.414,$n_D/n_A$为2。

所有的测试结果如表 \ref{tab:Fretha手动双杂交} 所示。
使用Fretha进行DC-FRET和L-FRET的测量结果均与文献值一致,说明Fretha手动数据处理在准确度上满足预期。
\begin{table*}[htbp]
  \centering
  \caption{手动处理C32V、CVC的FRET双杂交分析结果}
  \begin{tabularx}{\linewidth}{
    >{\centering\arraybackslash}X
    >{\centering\arraybackslash}p{2.2cm}
    >{\centering\arraybackslash}p{2.2cm}
    >{\centering\arraybackslash}p{2.2cm}
    >{\centering\arraybackslash}X
    >{\centering\arraybackslash}X
    >{\centering\arraybackslash}X
    >{\centering\arraybackslash}X
    >{\centering\arraybackslash}X}
    \toprule[1.5pt]
    \multirow{2}{*}{样本} & \multicolumn{3}{c}{DC-FRET结果} & \multicolumn{3}{c}{L-FRET结果} & \multicolumn{2}{c}{文献结果} \\
     & $E_{A,max}$ & $E_{D,max}$ & ${n_D/n_A}$ & $E_{D,max}$ & $E_{D,max}$ & ${n_D/n_A}$ & $E_{D,max}$ & $n_D/n_A$\\
    \midrule
    C32V & $0.318\pm0.036$ & $0.297\pm0.018$ & $1.071\pm0.144$ & $0.347$ & $0.344$ & $1.033$ & 0.311 & 1\\
    CVC & $0.795\pm0.018$ & $0.393\pm0.023$ & $2.023\pm0.113$ & $0.827$ & $0.419$ & $2.046$ & 0.414 & 2\\
    \bottomrule[1.5pt]
    \end{tabularx}
  \label{tab:Fretha手动双杂交}
\end{table*}

\section{Fretha软件测试}
\subsection{成像参数设置模块测试}
成像参数设置是软件的核心功能之一,直接影响到数据处理结果的准确性。
本测试旨在验证软件成像参数设置功能的正确性和灵活性,确保软件能够及时准确响应参数的更新设置。

测试时,通过界面操作更新参数,分别用表 \ref{tab:lurs_imaging_params} CV成像参数和CY参数处理单转CV体系标准质粒C32V和CY体系标准质粒C4Y样本的FRET图像数据,进行E-FRET分析计算结果$E_D$和$R_C$。将参数和数据匹配的实验组记为正面组,将参数不匹配数据的组别记为反面组,测试结果如表 \ref{表:测试参数更新} 所示。

\begin{table}[htbp]
  \centering
  \caption{更改参数C4Y和C32V质粒E-FRET测量结果}
  \begin{tabular}{ccccc}
  \toprule[1.5pt]
  实验组 & 参数 & 样本 & $E_D$ & $R_C$ \\
  \midrule
  \multirow{2}{*}{正面} & CV & C32V & $0.299\pm0.022$ & $0.981\pm0.113$ \\
  & CY & C4Y & $0.307\pm0.040$ & $0.994\pm0.096$ \\
  \multirow{2}{*}{反面} & CY & C32V & $0.382\pm0.054$ & $0.821\pm0.093$ \\
  & CV & C4Y & $0.221\pm0.037$ & $1.313\pm0.135$ \\
  \bottomrule[1.5pt]
  \end{tabular}
  \label{表:测试参数更新}
\end{table}

正面组的测量结果中,C32V的$E_D$为$0.299\pm0.022$,$R_C$为$0.981\pm0.113$,C4Y的$E_D$为$0.307\pm0.040$,$R_C$为$0.994\pm0.096$,均与文献值一致。
反面组的测量结果中,C32V的$E_D$为$0.382\pm0.054$,$R_C$为$0.821\pm0.093$,C4Y的$E_D$为$0.221\pm0.037$,$R_C$为$1.313\pm0.135$,与文献值存在较大偏差。
说明参数Fretha后台能够及时使用更新后的参数进行计算。

经过对界面和计算结果的检查和测试,成像参数设置模块的界面和功能均符合预期,软件界面和后台内存中的参数均能准备设置更新。

\subsection{数据检验模块测试}
如 \ref{sec:数据检验模块} 节所述,数据检验模块用于检查数据的完整性和安全性,防止非法数据对软件运行造成不可预测的影响。
本测试旨在验证软件数据检验模块的有效性,并检查软件对于异常数据的识别能力和隔离能力。

测试时,使用Fretha分别尝试打开FRET标准数据、FRET缺失数据、非FRET数据、空数据、异常数据,检查软件的数据检验模块的反应。
具体测试内容和结果如表 \ref{tab:测试数据完备性} 所示。

\begin{table*}[htbp]
  \centering
  \caption{Fretha数据检验模块测试结果 }
  \begin{tabular}{cccccc}
  \toprule[1.5pt]
  测试数据 & 视野文件夹内容 & 可打开 & 可计算 & 识别类别 & 是否符合预期\\
  \midrule

  \multirow{3}{*}{FRET标准数据} &
  \begin{tabular}[t]{@{}l@{}}
    DD.tif \\
    DA.tif \\
    AA.tif \\
  \end{tabular} &
  \multirow{3}{*}{\ding{51}} &
  \multirow{3}{*}{\ding{51}} &
  \multirow{3}{*}{FRET} &
  \multirow{3}{*}{\ding{51}}\\

  \multirow{2}{*}{FRET缺失数据} &
  \begin{tabular}[t]{@{}l@{}}
    DD.tif \\
    DA.tif \\
  \end{tabular} &
  \multirow{2}{*}{\ding{51}} & 
  \multirow{2}{*}{\ding{55}} &
  \multirow{2}{*}{Unknown} &
  \multirow{2}{*}{\ding{51}} \\

  \multirow{2}{*}{非FRET数据} &
  \begin{tabular}[t]{@{}l@{}}
    D.tif \\
    A.tif \\
  \end{tabular} &
  \multirow{2}{*}{\ding{51}} &
  \multirow{2}{*}{\ding{55}} &
  \multirow{2}{*}{Unknown} &
  \multirow{2}{*}{\ding{51}} \\

  空数据 &
  空 &
  \ding{55} &
  \ding{55} &
  Unknown &
  \ding{51} \\

  \multirow{3}{*}{损坏数据} &
  \begin{tabular}[t]{@{}c@{}}
    DD.tif(损坏) \\
    DA.tif \\
    AA.tif \\
  \end{tabular} &
  \multirow{3}{*}{\ding{51}} &
  \multirow{3}{*}{\ding{55}} &
  \multirow{3}{*}{Unknown} &
  \multirow{3}{*}{\ding{51}} \\
  
  \bottomrule[1.5pt]
  \end{tabular}
  \label{tab:测试数据完备性}
\end{table*}

结果表明,对于正常数据(FRET标准数据),数据检验模块能够正确识别数据类型,并且能够成功打开和计算数据,符合预期。
对于视野子文件夹中存在缺失数据、非FRET数据、空数据和损坏数据等情况,Fretha能够打开并显示在视野区,但无法打开或者参与自动计算,并且会被正常识别为“Unknown”类别以提醒用户该视野处于不可计算的状态。
点击“自动圈点”后,发现自动算法也能够成功忽略被识别为“Unknown”类别的视野,不会对其进行计算。
这些结果均符合预期,说明软件数据检验模块的功能正常,能够有效识别和阻止异常数据的计算。

\subsection{FRET图像处理模块测试}

FRET图像处理模块能够帮助用户进行FRET双杂交分析数据处理,其功能如 \ref{sec:FRET图像处理模块} 节所述。
在测试时,我们主要测试FRET图像处理的视图增强、ROI编辑和ROI状态栏更新功能,以验证软件的功能是否符合预期。

首先测试FRET视图增强功能。
分别切换表 \ref{tab:fretha_viewtype_list} 中的视图类型,查看视图增强的效果,过程中的软件界面和增强视图的截图如图 \ref{fig:视图测试} 所示。
结果表明,软件能够正确显示不同类型的视图,并且视图增强的效果明显,提高了用户在圈选ROI时的针对性和准确性。

\begin{figure*}[!htb]
  \centering
  \includegraphics[width=0.9\linewidth]{../figures/4/4_视图类型.png}
  \caption{Fretha图像处理视图切换测试结果}
  \label{fig:视图测试}
\end{figure*}

然后测试ROI绘制功能。
ROI编辑功能根据鼠标与ROI边框的位置关系,显示不同的鼠标样式。
分别移动鼠标位置到ROI内部、边框和外部的,观察鼠标的样式和按下时对应的功能。
测试结果如表 \ref{tab:ROI鼠标样式} 所示,表明软件能够正确显示不同的鼠标样式,且在不同位置对应的功能正确,响应迅速,保证了ROI绘制的操作体验。

\begin{table}
  \centering
  \caption{ROI绘制功能测试结果}
  \begin{tabular}{cccc}
    \toprule[1.5pt]
    鼠标位置 & 鼠标样式 & 按下效果 & 是否符合预期\\
    \midrule
    ROI内部 & 手型 & 移动ROI & \ding{51}\\
    ROI左边框 & 水平箭头 & 调整ROI宽度 & \ding{51} \\
    ROI右边框 & 水平箭头 & 调整ROI宽度 & \ding{51} \\
    ROI上边框 & 垂直箭头 & 调整ROI高度 & \ding{51} \\
    ROI下边框 & 垂直箭头 & 调整ROI高度 & \ding{51} \\
    ROI外部 & 十字形 & 新建ROI & \ding{51} \\
    \bottomrule[1.5pt]
  \end{tabular}
  \label{tab:ROI鼠标样式}
\end{table}

最后测试ROI状态栏的更新。
ROI状态栏上的数据更新发生在ROI被绘制更新以及视野切换时,因此通过编辑ROI和切换视野两种操作来测试ROI状态栏的更新,然后检查ROI状态栏的信息是否能够实时更新,结果如表 \ref{tab:ROI状态栏测试} 所示。
\begin{table}
  \centering
  \caption{ROI状态栏功能测试结果}
  \begin{tabular}{cccc}
    \toprule[1.5pt]
    \multirow{2}{*}{测试目标} & \multicolumn{2}{c}{ 测试结果} & \multirow{2}{*}{ 是否符合预期} \\
    & 更新ROI结果 & 切换视野结果 & \\
    \midrule
    DD通道信号 & 数据更新 & 重置为0 & \ding{51} \\
    DA通道信号 & 数据更新 & 重置为0 & \ding{51} \\
    AA通道信号 & 数据更新 & 重置为0 & \ding{51} \\
    DD通道SBR & 数据更新 & 重置为0 & \ding{51} \\
    DA通道SBR & 数据更新 & 重置为0 & \ding{51} \\
    AA通道SBR & 数据更新 & 重置为0 & \ding{51} \\
    DD通道背景 & 保持不变 & 数据更新 & \ding{51} \\
    DA通道背景 & 保持不变 & 数据更新 & \ding{51} \\
    AA通道背景 & 保持不变 & 数据更新 & \ding{51} \\
    $F_C$ & 数据更新 & 重置为0 & \ding{51} \\
    $E_D$ & 数据更新 & 重置为0 & \ding{51} \\
    $R_C$ & 数据更新 & 重置为0 & \ding{51} \\
    \bottomrule[1.5pt]
  \end{tabular}
  \label{tab:ROI状态栏测试}
\end{table}

上述测试结果表明,FRET图像处理的视图增强、ROI绘制和ROI状态栏更新功能均符合预期,软件能够正确显示不同类型的视图,辅助用户更好完成FRET双杂交分析图像处理工作。

\subsection{数据管理模块测试}

数据管理模块能够管理和组织数据区记录的数据,包括数据导入、数据导出、数据保存、数据计算等功能,如 \ref{sec:数据管理模块} 节所述。本节重点关注验证数据导入、数据导出和功能。

测试时的具体步骤如下:
\begin{enumerate}
  \item 使用导出数据功能导出圈点信息,检查文件生成和内容是否正确;
  \item 清空数据区,然后使用导入数据功能导入Fretha圈点数据(CSV),检查数据是否正确导入;
  \item 将导出文件另存为其他格式(XLSX),尝试导入XLSX文件,检查数据是否正确导入;
  \item 将另一组数据集生成的CSV文件导入,检查数据是否正确导入;
\end{enumerate}

测试结果如表 \ref{tab:数据管理模块测试结果} 所示。
当导入Fretha圈点数据(CSV),软件能够成功导入数据,数据准确无误。
导入XLSX格式的数据时,软件无法导入。
当导入不匹配Fretha数据(CSV)时,软件能够检测到数据不匹配,终止导入。
导出功能能够成功导出Fretha圈点数据(CSV),数据准确无误。
添加数据和删除数据功能均工作正常,且通过鼠标和快捷键均可以快速添加和删除数据。

\begin{table}[hbtp]
  \centering
  \caption{数据管理模块测试结果}
  \begin{tabular}{p{1.5cm} l l c} % 自定义各列宽度
    \toprule[1.5pt]
    {测试目标} & {测试内容} & {测试结果} & {是否符合预期} \\
    \midrule
    \multirow{2}{*}{清空数据} & 清空数据区 & 成功清空 & \ding{51} \\
     & 快捷键“C” & 成功清空 & \ding{51} \\
    筛选数据 & 筛选数据 & 成功筛选 & \ding{51} \\
    \multirow{3}{*}{导入数据} & 导入Fretha圈点数据(CSV) & 成功导入 & \ding{51} \\
     & 导入其他表格类型数据(XLSX) & 无法导入 & \ding{51} \\
     & 导入不匹配Fretha数据(CSV) & 无法导入 & \ding{51} \\
    导出数据 & 导出Fretha圈点数据(CSV) & 成功导出 & \ding{51} \\
    \multirow{2}{*}{添加数据} & 添加新数据 & 成功添加 & \ding{51} \\
     & 快捷键“A” & 成功添加 & \ding{51} \\
    \multirow{2}{*}{删除数据} & 删除数据 & 成功删除 & \ding{51} \\
     & 快捷键“D” & 成功删除 & \ding{51} \\
    开始计算 & 开始计算数据 & 成功跳转,计算准确 & \ding{51} \\

    \bottomrule[1.5pt]
  \end{tabular}
  \label{tab:数据管理模块测试结果}
\end{table}

\subsection{结果可视化模块测试}

结果可视化和保存模块能够帮助用户查看和保存数据处理结果,包括视图选择、图表结果、结果保存等,如 \ref{sec:结果可视化模块} 节所述。
本节对上述功能进行测试和验证,重点关注两种视图的显示和参数的更新,以及结果的保存功能。

测试L-FRET视图和BIN数据分箱功能,测试结果如图 \ref{fig:L-FRET视图测试} 所示。
L-FRET显示出良好的拟合效果,散点与趋势线符合预期。
测试所用的BIN参数为对$R_C$在$(0,5)$之间的数据按照0.01的步长进行分组并计算平均值,结果显示BIN参数更新后的L-FRET视图散点数明显减少,且趋势线有所变化。
\begin{figure}
  \centering
  \includegraphics[width=0.9\linewidth]{../figures/4/4_L-FRET视图测试.png}
  \caption[Fretha L-FRET视图测试结果]{Fretha L-FRET视图测试结果。图a为标准L-FRET视图,图b为应用BIN数据分箱后的L-FRET视图。}
  \label{fig:L-FRET视图测试}
\end{figure}

测试DC-FRET视图中调整线性拟合的数据范围参数设置功能时,首先设置线性拟合的$R_C$数据范围为$(0,0.5)$,均值拟合的$R_C$范围为$(2,10)$,然后设置线性拟合的$1/R_C$数据范围为$(0,1)$,均值拟合的$1/R_C$范围为$(1,10)$,观察图表中的趋势线和散点图更新。
测试结果如图 \ref{fig:DC-FRET视图测试} 所示,当扩大拟合数据的范围后,可以观察到趋势线的更新,斜率拟合段和均值拟合段均延伸至$R_C$为1的位置。
\begin{figure}
  \centering
  \includegraphics[width=0.9\linewidth]{../figures/4/4_DC-FRET参数测试.png}
  \caption[Fretha DC-FRET视图测试结果]{DC-FRET视图测试结果。图a为设置线性拟合的$R_C$数据范围为$(0,0.5)$,均值拟合的$R_C$范围为$(2,10)$时Fretha生成的趋势线和散点图;
  图b则设置线性拟合的$1/R_C$数据范围为$(0,1)$,均值拟合的$1/R_C$范围为$(1,10)$。}
  \label{fig:DC-FRET视图测试}
\end{figure}

测试保存的数据和报告是否能够正确保存,结果如图 \ref{fig:结果保存测试} 所示,软件能够正确生成结果文件,且数据和报告的内容准确无误,符合预期。
\begin{figure}[hbtp]
  \centering
  \includegraphics[width=0.6\linewidth]{../figures/2/2_保存数据.png}
  \caption{Fretha结果保存测试结果}
  \label{fig:结果保存测试}
\end{figure}

综合测试结果表明,Fretha结果可视化模块的功能正常,能够正确显示和保存数据处理结果,为用户提供了直观的数据分析和保存功能。

\section{Fretha稳定性测试}
稳定性测试旨在评估 Fretha 软件在长时间运行、高负载或异常操作下的可靠性。测试从:
\begin{enumerate}
  \item 压力测试:同时加载 10 个大型 FRET 数据集(每个数据集包含 50 个视野),连续执行参数设置、图像处理和结果计算操作,监测软件是否出现崩溃或内存泄漏。
  \item 长时间运行测试:保持软件连续运行 48 小时,期间每隔8小时进行数据导入、数据计算和保存,然后清空数据。记录软件内存占用情况,检测软件是否存在内存泄露
  \item 异常操作测试:快速重复进行参数切换、数据导入导出、ROI 编辑等操作,模拟用户高频使用场景,观察软件的响应稳定性。
\end{enumerate}

测试结果如表 \ref{tab:稳定性测试} 所示,Fretha 在压力测试下仍能稳定处理数据,未出现崩溃或内存溢出;长时间运行中的内存占用监测如图 \ref{fig:48小时内存变化} 所示,功能模块保持正常,资源占用无显著异常;高频操作下软件响应迅速,未出现卡顿或错误。这些结果证明 Fretha 具有良好的稳定性,能够满足科研和工业场景的长时间、高负载使用需求。
\begin{figure}[hbtp]
  \centering
  \includegraphics[width=0.7\linewidth]{../figures/4/4_48小时内存占用变化.png}
  \caption{Fretha软件48小时运行内存占用监测}
  \label{fig:48小时内存变化}
\end{figure}

\begin{table}[hbtp]
  \centering
  \caption{Fretha 软件稳定性测试结果}
  \label{tab:稳定性测试}
  \begin{tabular}{lp{5cm}p{5cm}}
  \toprule[1.5pt]
  {测试类型}         & {测试方法}                                                                 & {测试结果}                                                                 \\
  \midrule 
  \multirow{5}{*}{压力测试} 
    & \multirow{5}{5cm}{同时加载 10 个大型 FRET 数据集(每个数据集包含 50 个视野),连续执行参数设置、图像处理和结果计算操作,监测软件是否出现崩溃或内存泄漏。} 
    & - 处理 10 个数据集总耗时:42.5 分钟 \\
    &                                                                                
    & - 内存峰值占用:1.4 GB(稳定无泄漏) \\
    &                                                                                
    & - 操作成功率:100\% \\
  \midrule % 测试类型分隔线
  \multirow{5}{*}{长时间运行测试} 
    & \multirow{5}{5cm}{保持软件连续运行 48 小时,期间定期检查各功能模块(如图像处理、数据导出)是否正常工作,记录 CPU 和内存占用情况。} 
    & - CPU 平均使用率:22-25\%(波动<3\%) \\
    &                                                                                
    & - 内存平均占用:550-600 MB(无持续增长) \\
    &                                                                                
    & - 数据导出成功率:100\% \\
  \midrule % 测试类型分隔线
  \multirow{4}{*}{异常操作测试} 
    & \multirow{4}{5cm}{快速重复进行参数切换、数据导入导出、ROI 编辑等操作,模拟用户高频使用场景,观察软件的响应稳定性。} 
    & - 每秒操作次数:15-20 次/秒 \\
    &                                                                                
    & - 平均响应时间:0.3-0.5 秒 \\
    &                                                                                
    & - 未出现卡顿或界面冻结 \\
    \\
  \bottomrule[1.5pt]
  \end{tabular}
\end{table}

\section{本章小结}

本章对Fretha软件的各功能模块进行了全面的测试与验证,包括成像参数设置模块、数据检验模块、FRET图像处理模块、数据管理模块、结果可视化模块以及自动算法性能和软件稳定性测试。
测试结果表明,Fretha软件在功能实现、性能表现和稳定性方面均达到了预期目标。
成像参数设置模块、数据检验模块、FRET图像处理模块、数据管理模块、结果可视化模块等主要模块的正常运行支持了FRET双杂交分析数据处理的所有步骤的用户需求,保证了数据处理过程的安全和高效。

此外,LURS算法在处理速度、内存效率和硬件适应性方面展现出显著优势,为高通量实时分析提供了高效的解决方案。
稳定性测试进一步验证了Fretha在长时间运行、高负载和高频操作下的可靠性,未出现崩溃或性能下降的情况。
综合测试结果表明,Fretha软件功能完善、性能优越、稳定性良好,能够满足复杂科研和实际应用的需求。