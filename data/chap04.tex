\chapter{软件的应用和测试}

\section{引言}
\ifshowtext

Bcl-XL 作为抗凋亡蛋白家族核心成员,通过结合促凋亡蛋白 Bak/Bax 抑制线粒体凋亡通路\upcite{Adams1998The}。
其异常高表达与肿瘤耐药性密切相关,已被证实与多种癌症治疗抵抗机制直接关联\upcite{2006Mitochondrial}。
A1331852 作为新一代 BH3 模拟物,以皮摩尔级亲和力特异性靶向 Bcl-XL 疏水口袋,通过竞争性结合解除其对 Bak 的束缚,可显著削弱 Bcl-xL 与 Bak 的结合比例。
该变化可通过实时荧光共振能量转移信号定量检测\upcite{2006Mitochondrial, 2017New}。

本章我们应用我们开发的FRET双杂交分析数据处理软件,分析Bcl-xL和Bak互作时加入Bcl-xL抑制剂A1331852前后的互作变化情况\upcite {2020Stoichiometry}。
针对自动数据处理算法,我们对比了基于深度学习的自动数据处理方法,包括交互式医学图像分析软件ilastik和面向2D医学图像特化的分割大模型SAM\_Med2D,结果发现,计算结果也更接近手动处理的数据结果,速度最优,系统硬件配置要求也最低。
最后,在数据处理过程中,我们对软件的稳定性和功能进行验证,结果表明,Fretha软件运行稳定,所有功能符合预期。

\fi

\section{材料与方法}

\subsection{细胞培养和转染}
质粒方面,CFP-Bcl-XL 质粒由 A.P.Gilmore提供 ,YFP-Bak 质粒的构建方法此前已有报道 \upcite{warren2019bcl, sun2023regorafenib}。
细胞培养和转染条件如章节\ref{sec:细胞转染}所描述。
成像系统的参数如章节\ref{sec:成像条件}所描述,在成像时,每个通道的成像曝光时间为300ms。
药物 A1331852 购自美国新泽西州的 MedChemExpress(MCE)公司。

\subsection{数据处理操作}
在数据处理方法上,我们首先应用Fretha分别进行了手动数据处理和自动数据处理。然后我们与深度学习方法进行了自动数据处理,主要包括交互式深度学习医学图像处理软件ilastik和面向医学图像的分割大模型SAM\_Med2D,以和我们的自动处理方法进行对比。应用各种不同方法进行数据处理的具体做法如下:
\begin{enumerate}
    \item Fretha手动数据处理。首先将数据格式设定为Fretha匹配的数据结构,然后通过Fretha软件打开数据,在FRET图像处理圈点界面进行手动ROI选取,固定ROI大小为$5\times 5$,每个视野选取10至20个标定在细胞上的ROI,将数据记录到Fretha上。检查数据无误后,点击“开始计算”,获得手动FRET双杂交分析的实验结果。
    \item 基于Fretha的自动数据处理。导入数据后,点击软件界面上的“自动圈点”按钮,执行LURS算法自动在每个视野选取ROI,并记录数据到后台。检查数据无误后,点击“开始计算”,获得自动FRET双杂交分析的实验结果。
    \item 基于ilastik的自动数据处理。首先,我们从相同数据集中抽取随机视野,手动标注其中的好细胞、坏细胞、背景区域。取其中识别到的背景区域灰度均值作为背景灰度值,选取好细胞的区域灰度均值作为信号灰度值。使用Fretha内嵌的FRET定量计算功能和FRET双杂交分析求解功能,计算得到基于ilastik的自动分析的实验结果。
    \item 基于SAM\_Med2D的自动数据处理。首先,我们从相同数据集中抽取随机视野,手动标注人工圈点的ROI作为信号ROI的prompt,手动标注背景区域的ROI作为背景的prompt,将信号prompt和背景prompt输入SAM\_Med2D模型,使得大模型根据FRET图像特点进行能力特化,从而进行自动的数据处理。使用Fretha内嵌的FRET定量计算功能和FRET双杂交分析求解功能,计算得到基于SAM\_Med2D的自动分析的实验结果。
\end{enumerate}
为了保证不同方法在FRET定量分析和FRET双杂交分析计算的一致性,我们在维护Fretha的源码上新建了特殊的功能分支,并在软件中实现了根据输入图片和标注ROI的掩码二值化图片进行自动FRET计算功能。

我们应用FRET双杂交分析技术对照组(Control)和加药组(Medication)分别进行了FRET双杂交分析,以测量在加入药物A1331852前后Bcl-XL和Bak在细胞中相互作用的结合情况变化。

\section{结果}

\subsection{应用FRET双杂交技术分析Bcl-XL和Bak互作}

\section{软件的测试}
软件测试是确保软件质量和性能的重要环节。本部分将对软件的各项功能和性能进行全面测试,以验证其是否满足设计要求和用户需求。

\subsection{成像参数设置测试}
成像参数设置是软件的核心功能之一,它直接影响到数据处理结果的准确性。本测试旨在验证软件成像参数设置功能的正确性和灵活性,确保软件能够及时准确响应参数的更新设置。

通过界面操作更新参数,分别用CV成像参数和CY参数处理单转标准质粒C32V和C4Y样本的FRET图像数据,将数据导出后,检查E-FRET计算结果$E_D$和$R_C$。具体测试步骤如下:
\begin{enumerate}
    \item 打开软件,进入成像参数设置界面。
    \item 使用表\ref{tab:lurs_imaging_params}中CV质粒成像的参数进行设置,然后处理单独转染C32V质粒样本数据和单独转染C4Y质粒样本数据。
    \item 退出参数设置界面,再次进入参数设置界面。
    \item 使用表\ref{tab:lurs_imaging_params}中CY质粒成像的参数进行设置,然后处理单独转染C32V质粒样本数据和单独转染C4Y质粒样本数据。
    \item 关闭软件,重新打开软件,然后进入参数设置界面。
    \item 记录两次数据处理的结果。
\end{enumerate}

\begin{table*}[htbp]
    \centering
    \caption{ 切换参数对C32V质粒和C4Y质粒的E-FRET分析结果}
    \begin{tabularx}{\linewidth}{
    >{\centering\arraybackslash}X
    >{\centering\arraybackslash}X
    >{\centering\arraybackslash}X
    >{\centering\arraybackslash}X
    >{\centering\arraybackslash}X
    >{\centering\arraybackslash}X
    >{\centering\arraybackslash}X}
    \toprule
    \multirow{2}{*}{参数} & \multicolumn{2}{c}{C32V} & \multicolumn{2}{c}{C4Y} \\
    & $E_{D}$ & ${R_C}$ & $E_{D}$ & $R_C$ \\
    \midrule
    CV体系参数 & $0.29\pm0.02$ & $0.98\pm0.11$ & $0.22\pm0.03$ & $1.31\pm0.13$  \\
    CY体系参数 & $0.38\pm0.05$ & $0.82\pm0.09$ & $0.30\pm0.02$ & $1.02\pm0.12$  \\
    \bottomrule
    \end{tabularx}
    \label{表:测试参数更新}
\end{table*}

从软件界面上,在设置参数后重新进入参数设置界面,发现参数符合预期的新参数。在软件重启后,发现界面显示的参数也与最近一次更新的参数一致。

两次E-FRET分析的结果如表\ref{表:测试参数更新}所示。在计算功能上,在更新了CY成像参数后,处理C32V质粒计算得到的$E_D$和$R_C$结果分别为0.38和0.82,处理C4Y质粒,计算得到的$E_D$和$R_C$结果分别为0.30和1.02;
在使用CV参数处理C4Y质粒时,计算得到的$E_D$和$R_C$结果分别为0.22和1.31,处理C32V质粒时,计算得到的$E_D$和$R_C$结果分别为0.29和0.98。文献报道的C4Y质粒和C32V质粒的$E_D$为0.30,$R_C$结果为1。可以发现两次更新参数后,对应正确的质粒结果符合文献值,而不匹配的质粒的测量结果存在较大偏差。

经过以上测试,我们验证了成像参数设置模块的界面和功能均符合预期,参数设置响应准确。

\subsection{数据导入导出测试}
数据导入导出功能是软件与其他系统进行数据交互的重要手段。本测试旨在验证软件数据导入导出功能的正确性和兼容性。
确保软件能够正确地导入和导出支持的CSV数据文件,并且数据的完整性和准确性得到保证。

准备不同格式的测试数据文件,分别进行导入和导出操作,然后检查导入和导出的数据是否一致。
具体测试步骤如下:
\begin{enumerate}
    \item 准备包含不同类型数据的 CSV、Excel、JSON 文件。
    \item 打开软件,选择数据导入功能,依次导入上述测试文件,检查导入的数据是否正确显示。
    \item 对导入的数据进行一些修改和处理,然后选择数据导出功能,将数据导出为相同格式的文件。
    \item 比较原始文件和导出文件的数据内容,确保数据的完整性和准确性。
\end{enumerate}

测试结果表明,软件能够正确地导入和导出 CSV、Excel、JSON 等格式的数据文件,并且数据的完整性和准确性得到了保证。但在导入大型文件时,软件的导入速度较慢,需要进行优化。

\subsection{结果保存测试}
结果保存功能是软件的重要功能之一,它能够帮助用户保存分析结果和处理数据。本测试旨在验证软件结果保存功能的正确性和可靠性。

确保软件能够正确地保存各种类型的结果文件,如文本文件、图像文件、报告文件等,并且保存的文件能够被正确打开和查看。

在软件中进行各种操作,生成不同类型的结果文件,然后选择结果保存功能,将结果保存到指定的文件夹中。最后,检查保存的文件是否能够被正确打开和查看。具体测试步骤如下:
\begin{enumerate}
    \item 在软件中进行数据分析和处理,生成文本结果、图像结果和报告结果。
    \item 选择结果保存功能,分别将上述结果保存为文本文件、图像文件和报告文件。
    \item 打开保存的文件,检查文件内容是否与软件中显示的结果一致。
\end{enumerate}

经过测试,发现软件能够正确地保存各种类型的结果文件,并且保存的文件能够被正确打开和查看。但在保存文件时,软件没有提供文件覆盖提示功能,可能会导致用户误操作。

\subsection{软件的性能测试}
软件的性能是衡量软件质量的重要指标之一。本测试旨在评估软件在不同负载条件下的性能表现,如响应时间、吞吐量等。

确定软件在正常使用和高负载情况下的性能指标,为软件的优化和升级提供依据。

使用性能测试工具,模拟不同的用户负载,对软件的各项性能指标进行监测和分析。具体测试步骤如下:
\begin{enumerate}
    \item 选择合适的性能测试工具,如 JMeter、LoadRunner 等。
    \item 设计性能测试场景,包括不同的用户并发数、操作频率等。
    \item 运行性能测试,记录软件的响应时间、吞吐量、CPU 使用率等性能指标。
    \item 分析测试结果,找出性能瓶颈和问题所在。
\end{enumerate}

测试结果显示,软件在正常使用情况下性能表现良好,但在高负载情况下,响应时间明显增加,吞吐量下降。需要对软件进行优化,提高其性能和稳定性。

\subsection{软件的稳定性测试}
软件的稳定性是指软件在长时间运行过程中保持正常工作的能力。本测试旨在验证软件在长时间运行过程中的稳定性和可靠性。

确保软件在长时间运行过程中不会出现崩溃、死机等异常情况,保证软件的正常使用。

让软件连续运行一段时间,模拟实际使用场景,观察软件的运行状态和性能表现。具体测试步骤如下:
\begin{enumerate}
    \item 打开软件,进行一些基本的操作,如登录、数据查询等。
    \item 让软件连续运行 24 小时以上,期间不断进行各种操作,如数据导入、导出、分析等。
    \item 观察软件的运行状态,记录是否出现崩溃、死机、数据丢失等异常情况。
    \item 分析测试结果,评估软件的稳定性和可靠性。
\end{enumerate}

经过长时间的稳定性测试,发现软件在大部分时间内运行稳定,但偶尔会出现卡顿现象。需要对软件进行进一步的优化和调试,提高其稳定性。

\section{本章小结}