\chapter{Fretha的测试}

\section{引言}
\ifshowtext

本章我们应用我们开发的FRET双杂交分析数据处理软件,分析Bcl-xL和Bak互作时加入Bcl-xL抑制剂A1331852前后的互作变化情况\upcite {2020Stoichiometry}。
针对自动数据处理算法,我们对比了基于深度学习的自动数据处理方法,包括交互式医学图像分析软件ilastik和面向2D医学图像特化的分割大模型SAM\_Med2D,结果发现,计算结果也更接近手动处理的数据结果,速度最优,系统硬件配置要求也最低。
最后,在数据处理过程中,我们对软件的稳定性和功能进行验证,结果表明,Fretha软件运行稳定,所有功能符合预期。

\fi

\section{Fretha功能测试}
软件测试是确保软件质量和性能的重要环节。结合Fretha的应用实验,我们对软件的功能模块进行了测试,包括成像参数设置、数据导入导出、结果保存、软件性能和稳定性等方面。测试结果表明,Fretha软件在大部分情况下能够正常工作,但在一些极端情况下可能会出现问题,需要进一步优化和改进。

\subsection{成像参数设置模块测试}
成像参数设置是软件的核心功能之一,直接影响到数据处理结果的准确性。本测试旨在验证软件成像参数设置功能的正确性和灵活性,确保软件能够及时准确响应参数的更新设置。

在测试方法上,本节通过界面操作更新参数,分别用CV成像参数和CY参数处理单转标准质粒C32V和C4Y样本的FRET图像数据,将数据导出后,检查E-FRET计算结果$E_D$和$R_C$。具体测试步骤如下:
\begin{enumerate}
    \item 打开软件,进入成像参数设置界面。
    \item 使用表 \ref{tab:lurs_imaging_params} 中CV质粒成像的参数进行设置,然后处理单独转染C32V质粒样本数据和单独转染C4Y质粒样本数据。
    \item 退出参数设置界面,再次进入参数设置界面。
    \item 使用表 \ref{tab:lurs_imaging_params} 中CY质粒成像的参数进行设置,然后处理单独转染C32V质粒样本数据和单独转染C4Y质粒样本数据。
    \item 关闭软件,重新打开软件,然后进入参数设置界面。
    \item 记录两次数据处理的结果。
\end{enumerate}

\begin{table*}[htbp]
    \centering
    \caption{ 切换参数对C32V质粒和C4Y质粒的E-FRET分析结果}
    \begin{tabularx}{\linewidth}{
    >{\centering\arraybackslash}X
    >{\centering\arraybackslash}X
    >{\centering\arraybackslash}X
    >{\centering\arraybackslash}X
    >{\centering\arraybackslash}X
    >{\centering\arraybackslash}X
    >{\centering\arraybackslash}X}
    \toprule
    \multirow{2}{*}{参数} & \multicolumn{2}{c}{C32V} & \multicolumn{2}{c}{C4Y} \\
    & $E_{D}$ & ${R_C}$ & $E_{D}$ & $R_C$ \\
    \midrule
    CV体系参数 & $0.29\pm0.02$ & $0.98\pm0.11$ & $0.22\pm0.03$ & $1.31\pm0.13$  \\
    CY体系参数 & $0.38\pm0.05$ & $0.82\pm0.09$ & $0.30\pm0.02$ & $1.02\pm0.12$  \\
    \bottomrule
    \end{tabularx}
    \label{表:测试参数更新}
\end{table*}

从软件界面上,在设置参数后重新进入参数设置界面,发现参数符合预期的新参数。在软件重启后,发现界面显示的参数也与最近一次更新的参数一致。
说明软件界面能够准确显示最新的参数设置。

两次E-FRET分析的结果如表 \ref{表:测试参数更新} 所示。在计算功能上,在更新了CY成像参数后,处理C32V质粒计算得到的$E_D$和$R_C$结果分别为0.38和0.82,处理C4Y质粒,计算得到的$E_D$和$R_C$结果分别为0.30和1.02;
在使用CV参数处理C4Y质粒时,计算得到的$E_D$和$R_C$结果分别为0.22和1.31,处理C32V质粒时,计算得到的$E_D$和$R_C$结果分别为0.29和0.98。文献报道的C4Y质粒和C32V质粒的$E_D$为0.30,$R_C$结果为1。可以发现两次更新参数后,对应正确的质粒结果符合文献值,而不匹配的质粒的测量结果存在较大偏差。
说明软件内存中的参数也正确更新。

经过对界面和计算结果的检查和测试,成像参数设置模块的界面和功能均符合预期,软件界面和后台内存中的参数均能准备设置更新。

\subsection{数据完备性模块测试}
如 \ref{sec:数据完备性检验模块} 节所述,数据完备性检验模块是软件的核心功能之一,它能够帮助用户检查数据的完整性和准确性。本测试旨在验证软件数据完备性检验模块的正确性和有效性。

测试方法上,分别尝试导入FRET标准数据、FRET缺失数据、非FRET数据、空数据、异常数据,检查软件的数据完备性检验模块的反应。测试内容和结果如表\ref{tab:测试数据完备性}所示。

\begin{table*}[htbp]
  \centering
  \caption{Fretha数据完备性检验模块测试结果 }
  \begin{tabular}{ccccc}
  \toprule
  测试数据 & 视野文件夹内容 & 可打开 & 可计算 & 识别类别\\
  \midrule

  \multirow{3}{*}{FRET标准数据} &
  \begin{tabular}[t]{@{}l@{}}
    DD.tif \\
    DA.tif \\
    AA.tif \\
  \end{tabular} &
  \multirow{3}{*}{\ding{51}} &
  \multirow{3}{*}{\ding{51}} &
  \multirow{3}{*}{FRET} \\

  \multirow{2}{*}{FRET缺失数据} &
  \begin{tabular}[t]{@{}l@{}}
    DD.tif \\
    DA.tif \\
  \end{tabular} &
  \multirow{2}{*}{\ding{51}} & 
  \multirow{2}{*}{\ding{55}} &
  \multirow{2}{*}{Unknown} \\

  \multirow{2}{*}{非FRET数据} &
  \begin{tabular}[t]{@{}l@{}}
    D.tif \\
    A.tif \\
  \end{tabular} &
  \multirow{2}{*}{\ding{51}} &
  \multirow{2}{*}{\ding{55}} &
  \multirow{2}{*}{Unknown} \\

  空数据 &
  空 &
  \ding{55} &
  \ding{55} &
  Unknown \\

  \multirow{3}{*}{损坏数据} &
  \begin{tabular}[t]{@{}c@{}}
    DD.tif(损坏) \\
    DA.tif \\
    AA.tif \\
  \end{tabular} &
  \multirow{3}{*}{\ding{51}} &
  \multirow{3}{*}{\ding{55}} &
  \multirow{3}{*}{Broken} \\
  
  \bottomrule
  \end{tabular}
  \label{tab:测试数据完备性}
\end{table*}

\subsection{FRET图像处理模块测试}

打开Fretha软件,导入FRET标准数据,进入FRET图像处理模块,进行数据处理。

首先测试FRET视图增强功能。
分别切换表 \ref{tab:fretha_viewtype_list} 中的视图类型,查看视图增强的效果,过程中的软件界面和增强视图的截图如图 \ref{fig:视图测试} 所示。
结果表明,软件能够正确显示不同类型的视图,并且视图增强的效果明显,提高了用户在圈选ROI时的针对性和准确性。

\begin{figure*}[!htb]
  \centering
  \includegraphics[width=0.8\linewidth]{../figures/4/4_视图类型.png}
  \caption{Fretha图像处理视图切换测试结果}
  \label{fig:视图测试}
\end{figure*}

ROI编辑功能根据鼠标与ROI边框的位置关系,显示不同的鼠标样式。
测试方法上,我们测试了鼠标在ROI边框上、内部、外部的样式,结果表明软件能够正确显示不同的鼠标样式,提高了用户对ROI的操作体验,如表\ref{tab:ROI鼠标样式}所示。

\begin{table}
  \centering
  \caption{ROI绘制功能测试结果}
  \begin{tabular}{cccc}
    \toprule
    鼠标位置 & 鼠标样式 & 按下效果 & 是否符合预期\\
    \midrule
    ROI内部 & 手型 & 移动ROI & \ding{51}\\
    ROI左边框 & 水平箭头 & 调整ROI宽度 & \ding{51} \\
    ROI右边框 & 水平箭头 & 调整ROI宽度 & \ding{51} \\
    ROI上边框 & 垂直箭头 & 调整ROI高度 & \ding{51} \\
    ROI下边框 & 垂直箭头 & 调整ROI高度 & \ding{51} \\
    ROI外部 & 十字形 & 新建ROI & \ding{51} \\
    \bottomrule
  \end{tabular}
  \label{tab:ROI鼠标样式}
\end{table}

然后测试ROI状态栏的更新。
ROI状态栏上的数据更新发生在ROI被绘制更新以及视野切换时,因此通过编辑ROI和切换视野两种操作来测试ROI状态栏的更新,然后检查ROI状态栏的信息是否能够实时更新,结果如表\ref{tab:ROI状态栏}所示。
\begin{table}
  \centering
  \caption{重绘ROI测试状态栏数据更新}
  \begin{tabular}{cccc}
    \toprule
    状态栏信息 & 更新ROI测试结果 & 切换视野测试结果 & 是否符合预期 \\
    \midrule
    DD通道信号 & 数据更新 & 重置为0 & \ding{51} \\
    DA通道信号 & 数据更新 & 重置为0 & \ding{51} \\
    AA通道信号 & 数据更新 & 重置为0 & \ding{51} \\
    DD通道SBR & 数据更新 & 重置为0 & \ding{51} \\
    DA通道SBR & 数据更新 & 重置为0 & \ding{51} \\
    AA通道SBR & 数据更新 & 重置为0 & \ding{51} \\
    DD通道背景 & 保持不变 & 数据更新 & \ding{51} \\
    DA通道背景 & 保持不变 & 数据更新 & \ding{51} \\
    AA通道背景 & 保持不变 & 数据更新 & \ding{51} \\
    $F_C$ & 数据更新 & 重置为0 & \ding{51} \\
    $E_D$ & 数据更新 & 重置为0 & \ding{51} \\
    $R_C$ & 数据更新 & 重置为0 & \ding{51} \\
    \bottomrule
  \end{tabular}
  \label{tab:ROI状态栏}
\end{table}

上述测试结果表明,FRET图像处理的视图增强、ROI编辑和ROI状态栏更新功能均符合预期,软件能够正确显示不同类型的视图,辅助用户更好完成FRET双杂交分析数据处理中的ROI选取功能。 

\subsection{数据管理模块测试}
数据管理模块的数据导入导出功能是软件与其他系统进行数据交互的重要手段。
本测试旨在验证软件数据导入导出功能的正确性和兼容性。
确保软件能够正确地导入和导出支持的CSV数据文件,并且数据的完整性和准确性得到保证。

准备不同格式的测试数据文件,分别进行导入和导出操作,然后检查导入和导出的数据是否一致。
具体测试步骤如下:
\begin{enumerate}
    \item 准备包含不同类型数据的 CSV、Excel、JSON 文件。
    \item 打开软件,选择数据导入功能,依次导入上述测试文件,检查导入的数据是否正确显示。
    \item 对导入的数据进行一些修改和处理,然后选择数据导出功能,将数据导出为相同格式的文件。
    \item 比较原始文件和导出文件的数据内容,确保数据的完整性和准确性。
\end{enumerate}

测试结果表明,软件能够正确地导入和导出 CSV、Excel、JSON 等格式的数据文件,并且数据的完整性和准确性得到了保证。
但在导入大型文件时,软件的导入速度较慢,需要进行优化。

\subsection{结果可视化和保存测试}
结果保存功能是软件的重要功能之一,它能够帮助用户保存分析结果和处理数据。本测试旨在验证软件结果保存功能的正确性和可靠性。

确保软件能够正确地保存各种类型的结果文件,如文本文件、图像文件、报告文件等,并且保存的文件能够被正确打开和查看。

在软件中进行各种操作,生成不同类型的结果文件,然后选择结果保存功能,将结果保存到指定的文件夹中。最后,检查保存的文件是否能够被正确打开和查看。具体测试步骤如下:
\begin{enumerate}
    \item 在软件中进行数据分析和处理,生成文本结果、图像结果和报告结果。
    \item 选择结果保存功能,分别将上述结果保存为文本文件、图像文件和报告文件。
    \item 打开保存的文件,检查文件内容是否与软件中显示的结果一致。
\end{enumerate}

经过测试,发现软件能够正确地保存各种类型的结果文件,并且保存的文件能够被正确打开和查看。但在保存文件时,软件没有提供文件覆盖提示功能,可能会导致用户误操作。

\section{Fretha性能分析}

\subsection{自动算法性能测试}
在相同硬件配置(Intel\textsuperscript{\textregistered} Xeon E5-2678 v3 @ 2.50GHz处理器,NVIDIA\textsuperscript{\textregistered} GeForce RTX 3090 GPU)下,我们对LURS算法与两种主流深度学习方法(ilastik和SAM-Med2D)进行了系统性性能对比。
实验结果表明,LURS在处理速度、内存效率和硬件适应性方面展现出显著优势。  

如表\ref{tab5}所示,LURS方法在1.4GB数据集(包含30个视野的药物处理组和对照组)中仅需20秒即可提取700个可分析信号,单ROI处理时间低至6.6 ms。
相比之下,ilastik和SAM-Med2D的单ROI处理时间分别为35.2 ms和50.7 ms,LURS的速度较ilastik提升5.3倍,较SAM-Med2D提升7.7倍。  

内存利用率方面,LURS仅占用约800 MB内存,分别为ilastik(~1.8 GB)的44\%和SAM-Med2D(~14 GB)的5.7\%。这一优化使得LURS在资源受限的环境中仍能高效运行。
此外,LURS完全依赖CPU资源,无需专用GPU加速,使其能够直接集成到实时显微镜成像系统中,为高通量筛选应用提供了硬件无关性和部署灵活性。  

上述结果表明,LURS算法在保证数据处理精度的同时,通过优化计算流程和内存管理,显著提升了处理效率并降低了硬件依赖,为实时、高通量的FRET数据分析提供了更优的解决方案。

\begin{table*}[htbp]
    \centering
    \caption{不同算法的性能对比}
    \begin{tabular}{cccc}
    \toprule
    方法 & 单ROI处理时间(ms) & 内存占用 & 硬件依赖 \\
    \midrule
    LURS & 6.6 & ~800 MB & CPU \\
    ilastik & 35.2 & ~1.8 GB & GPU/CPU \\
    SAM-Med2D & 50.7 & ~14 GB & GPU \\
    \bottomrule
    \end{tabular}
    \label{tab5}
\end{table*}

\section{Fretha稳定性测试}

\section{本章小结}