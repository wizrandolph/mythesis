\begin{cabstract}

\ifshowtext
福斯特共振能量转移(Förster Resonance Energy Transfer, FRET)技术被广泛用于探究活细胞中生物大分子的动态相互作用过程,如生物大分子的构象变化、蛋白质相互作用、信号通路中蛋白质间的调节机制等,在研究生命科学基础问题、 疾病的发生发展和药物研发等方面具有广阔的应用前景。
FRET双杂交分析不仅可以在活细胞内原位实时测量蛋白质间真实的相互作用,还能给出蛋白质结合的相对亲和力及化学计量比。
目前主流的活细胞FRET双杂交分析方法是基于以供体/受体为中心的FRET效率($E_D$或$E_A$)与自由受体/供体浓度($A_{free}$或$D_{free}$)之间关系,即${E_D-A_{free}}$或${E_A-D_{free}}$进行朗缪尔拟合的方法(本文称为朗缪尔拟合方法L-FRET),但是其鲁棒性较差,对细胞样本制备和数据分析的要求极为苛刻。
为此,有团队发展了基于FRET效率与受供体浓度比($R_C$),即$E_D-R_c$或$E_A-1/R_C$线性分离的方法(本文称为线性拟合方法DC-FRET)。
实验证明,DC-FRET具有较好的鲁棒性,并且降低了对细胞样本的要求。
\fi

\end{cabstract}

\ckeywords{\ifshowtext FRET;FRET定量分析;荧光图像;FRET双杂交分析;自动数据处理 \fi}

\begin{eabstract}

\ifshowtext
Urban flooding is one of the major environmental challenges affecting human society. Understanding the mechanisms of how geographic factors influence flooding is crucial for developing effective disaster mitigation strategies. This study innovatively integrates street-view imagery data with machine learning methods to extract surface features such as building ratio (SVB), vegetation ratio (SVV), road ratio (SVR), and terrain ratio (SVT), while incorporating variables like population density and land use. Comparative analyses were conducted using XGBoost, RandomForest, LightGBM, and Linear Regression models. The results indicate that: (1) The XGBoost model performed the best, with an MAE of 0.2089, R² of 0.0596, and RMSE of 0.4083, significantly outperforming other models and demonstrating its advantages in capturing the nonlinear relationships of flood factors. (2) SHAP value analysis revealed that impervious surface ratio and population density are key factors influencing flood susceptibility, while surface features extracted from street-view imagery also showed significant contributions, further highlighting the spatial heterogeneity of urban flooding. By introducing street-view imagery data, this study effectively improves the accuracy and interpretability of flood susceptibility prediction, providing new technical tools and decision-making support for urban planning and flood risk management.
\fi
\end{eabstract}

\ekeywords{ \ifshowtext FRET, FRET quantitative analysis, FRET two-hybrid assay, Automated data processing \fi}

