\chapter{总结与展望}

\section{本文的主要内容和总结}
本文围绕 FRET 双杂交分析数据处理的自动化需求,设计并开发了专用软件 Fretha,系统研究了 FRET 双杂交分析的关键技术,提出了基于明度和均匀度的自动 ROI 选取算法(LURS),显著提升了数据处理的效率和准确性。

研究内容主要包括以下方面:
\begin{enumerate}
    \item 软件架构设计:基于 FRET 双杂交分析流程,采用分层架构设计,实现了成像参数设置、数据校验、图像处理、数据管理和结果可视化五大核心模块,支持 E - FRET、3³ - FRET 和 DC - FRET 等多种分析方法。
    \item 自动算法开发:提出 LURS 算法,通过多通道自适应阈值分割和变异系数均匀性评估,实现了 ROI 的自动化选取,处理速度较人工操作提升 80\%,在标准质粒验证中 $E_{A}$、$E_{D}$ 测量误差小于 5\%,$R_{C}$ 偏差不超过 0.05。
    \item 实验验证与测试:通过模型质粒和活细胞实验验证了算法的准确性,在 Bcl - xL - Bak 相互作用分析中,化学计量比测量结果与手动分析高度一致($n_{D}/n_{A}$ 偏差 < 6\%),且软件在高通量场景下表现出良好的稳定性和鲁棒性。
\end{enumerate}

\section{未来发展展望}
尽管 Fretha 已实现 FRET 双杂交分析的自动化处理,但仍存在优化空间,未来可从以下方向拓展:
\begin{enumerate}
    \item 集成LURS算法到显微镜在线成像系统。
    LURS算法在性能上已经达到了实验要求,但是目前还需要通过软件的方式导入到显微镜成像系统中,未来可以考虑将LURS算法集成到显微镜成像系统中,实现实时成像和数据处理。
    \item 功能完善与场景适配
    Fretha 目前仅支持 E - FRET、3³ - FRET 和 DC - FRET 等基本分析方法,未来可根据用户需求,增加更多的分析方法和功能模块,如蛋白质互作网络分析、药物筛选等。未来,可以参考ilastik的模块化设计,将不同的分析方法和功能模块集成到Fretha中。
    \item 数据库与云服务
    Fretha 目前仅支持本地数据管理,未来可考虑搭建云服务平台,实现数据的在线存储、共享和分析,提高数据的可访问性和可用性。庞大的药物筛选数据也可以通过云服务进行分析,提高数据的处理效率。
\end{enumerate}

通过持续迭代,Fretha 将逐步发展为集高效性、准确性和易用性于一体的 FRET 数据分析平台,为蛋白质互作研究和药物开发提供更强大的技术支持。