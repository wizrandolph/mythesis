\chapter{总结与展望}

\section{本文的主要内容和总结}
本文围绕 FRET 双杂交分析数据处理的需求和难点,研发了一款FRET双杂交分析数据处理软件 Fretha。针对FRET图像处理时人工标注ROI的难点,分析了高质量ROI在明度和变异系数上的特点,提出了基于明度和均匀度的 ROI 选取算法LURS,显著提升了数据处理的效率和准确性。
本文的研究内容主要包括以下三个方面:

(1)根据FRET双杂交分析数据处理的流程和需求,研发了一款FRET双杂交分析数据处理软件Fretha。
完成了Fretha的分层架构设计、模块功能划分、界面设计、模块开发实现等工作。
通过对基本数据模型。
本文实现了成像参数设置模块和数据检验模块,可有效保证数据处理的前置条件和信息检验。
FRET图像处理模块能够切换增强视图类型,辅助ROI状态栏,优化了标注ROI时的信息反馈,并且设置了ROI的自动化选取功能,提升了数据处理的效率。
数据管理模块提供了数据筛选、数据导入导出、一键计算等功能,便于用户对数据进行管理和分析,实现了对数据模型的统一管理和在模块间的高效传递。
结果可视化模块提供了数据可视化、结果导出等功能,便于用户对数据进行分析和展示。
Fretha体现出针对FRET双杂交分析数据处理的专业性和实用性,提高了FRET双杂交分析数据处理的规范化和自动化。

(2)提出了基于明度和均匀度的 ROI 选取算法 LURS。
通过多通道自适应阈值分割和变异系数均匀性评估,实现了 ROI 的自动化选取。
LURS算法通过筛选每个ROI在局部的相对亮度和变异系数,有效排除低亮度背景和灰度突变区域。
在标准质粒C4Y、C10Y、C40Y、C80Y的E-FRET测量和$3^3$-FRET测量中,$E_{A}$和$E_{D}$值与文献报道误差小于5\%,$R_{C}$偏差不超过0.05。
在C32V和CVC模型质粒的FRET双杂交验证中,测量得到的C32V中C与V的结合计量比为1.06,在CVC中则为1.90,符合文献报道的结果。
引用LURS算法测量对照组细胞中Bcl-xL-Bak相互作用的化学计量比为1.87,加药A1331852处理组则为1.12,化学计量比测量结果与手动分析高度一致。
对比基于SAM-Med2D和ilastik的算法,LURS表现出良好的稳定性和鲁棒性,适用于高通量药物筛选等大数据量场景。

(3)对Fretha软件进行了系统性测试与评估,完成了对Fretha软件的功能测试、性能测试和稳定性测试。
Fretha的参数设置模块响应及时,能够成功更新成像参数,确保数据计算的准确性。
数据检验模块可以识别并隔离异常数据,确保软件运行的安全稳定。
FRET图像处理模块的图像显示正常,各个控件的操作响应及时,能够快速标注绘制ROI,更新ROI状态栏信息。
数据管理模块能够成功导入导出数据,支持一键计算功能,能够快速完成数据的计算和结果的导出。
结果可视化模块展示了FRET双杂交分析视图,支持多种格式的结果导出,便于用户对数据进行分析和展示。
性能测试表明,Fretha中LURS算法在1.4GB数据集上单ROI处理时间仅6.6 ms,内存占用800 MB,优于其他软件工具如ilastik(35.2 ms / 1.8 GB)和SAM-Med2D(50.7 ms / 14 GB)。
稳定性测试表明,测试了Fretha软件在48小时连续运行和高频操作下保持稳定,资源占用无显著波动。

\section{展望}
本文研发的FRET双杂交分析数据处理软件Fretha实现了规范化、自动化的FRET双杂交分析数据处理,提升了数据处理的效率和准确性。
Fretha的推出为活细胞FRET双杂交分析技术提供了标准化、自动化的解决方案,有望推动该技术在精准医疗和药物研发中的大规模应用。
尽管Fretha及LURS算法已实现 FRET 双杂交分析的自动化处理,面对未来可能的大数据量数据处理需求,今后的工作可以围绕以下几个方面进行改进和完善:

(1)集成LURS算法到显微镜在线成像系统。
本文提出的LURS算法在性能上已经达到了实时数据处理的实验要求,但是目前还需要将显微镜成像系统中的数据通过数据检验模块导入Fretha中离线处理。
未来可以考虑将LURS算法集成到显微镜成像系统中,实现实时成像和数据处理,研发智能化的在线成像系统。
在线成像系统可以实时监测细胞的荧光信号变化,自动选取ROI并进行数据处理,实时输出FRET双杂交分析结果。
从而加大显微镜通量,进一步提高FRET双杂交实验的效率。

(2)完善Fretha功能,适配更多细胞图像分析方法。
Fretha 目前支持 E-FRET、$3^3$-FRET、L-FRET和DC-FRET 等分析方法。
未来可根据用户需求,增加更多的分析方法和功能模块,如凋亡分析、细胞周期分析等。
Fretha 还可以通过引入开源深度学习推理库,如 OpenVINO、TensorRT 等,集成深度学习算法,丰富自动数据处理的算法库。
参考ilastik的模块化设计,后续可以将不同的分析方法和功能模块集成到Fretha中,用户可以自行选择需要的模块进行数据处理。

(3)搭建数据库和云服务平台。
Fretha 目前仅支持本地数据管理,未来可考虑搭建云服务平台,实现数据的在线存储、共享和分析,提高数据的可访问性和可用性。
Fretha的自动数据处理庞大的药物筛选数据也可以通过云服务进行分析,提高数据的处理效率。

通过持续迭代,Fretha 将逐步发展为集高效性、准确性和易用性于一体的 FRET 数据分析平台,为蛋白质互作研究和药物开发提供更强大的技术支持。