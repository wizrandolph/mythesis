\chapter{总结与展望}

\section{本文的主要内容和总结}
本文围绕 FRET 双杂交分析数据处理的需求和难点,研发了一款FRET双杂交分析数据处理软件 Fretha,提高了FRET双杂交分析技术的规范化水平。
针对FRET图像处理时繁杂的人工标注ROI工作,本文提出了基于明度和均匀度的 ROI 选取算法LURS,基于此实现了自动FRET双杂交分析,显著提高了数据处理的效率。
本文的研究内容主要包括以下三个方面:

(1)根据FRET双杂交分析数据处理的流程和需求,研发了一款FRET双杂交分析数据处理软件Fretha。
完成了Fretha的分层架构设计、模块功能划分、界面设计、模块开发实现等工作。
通过对基本数据模型。
本文实现了成像参数设置模块和数据检验模块,可有效保证数据处理的前置条件和信息检验。
FRET图像处理模块能够切换增强视图类型,辅助ROI状态栏,优化了标注ROI时的信息反馈,提升了数据处理的效率。
数据管理模块提供了数据筛选、数据导入导出、一键计算等功能,便于用户对数据进行管理和分析,实现了对数据模型的统一管理和在模块间的高效传递。
结果可视化模块提供了数据可视化、结果导出等功能,便于用户对数据进行分析和展示。
Fretha的专业性和易用性,降低了FRET双杂交分析技术的使用门槛,提高了规范化程度,对于推广FRET双杂交分析技术应用具有重要意义。

(2)对Fretha软件进行了系统性测试与评估,完成了对Fretha软件的功能测试和可靠性测试。
经过$3^3$-FRET和E-FRET分析验证,Fretha测量的$E_{A}$、$E_{D}$和$R_C$值与文献报道值的平均相对误差分别为3.9\%、2.2\%和0.8\%。
FRET双杂交验证实验中,Fretha测量得到的C32V中$E_{A,max}$为0.318,$E_{D,max}$为0.297,$n_D/n_A$为1.071,CVC中$E_{A,max}$为0.795,$E_{D,max}$为0.393,$n_D/n_A$为2.023。
Fretha的参数设置模块响应及时,能够成功更新成像参数,确保数据计算的准确性。
数据检验模块可以识别并隔离异常数据,确保软件运行的安全稳定。
FRET图像处理模块的图像显示正常,各个控件的操作响应及时,能够快速标注绘制ROI,更新ROI状态栏信息。
数据管理模块能够成功导入导出数据,支持一键计算功能,能够快速完成数据的计算和结果的导出。
结果可视化模块展示了FRET双杂交分析视图,支持多种格式的结果导出,便于用户对数据进行分析和展示。
性能测试表明,Fretha中LURS算法在1.4GB数据集上单ROI处理时间仅6.6 ms,内存占用800 MB,优于其他软件工具如ilastik(35.2 ms / 1.8 GB)和SAM-Med2D(50.7 ms / 14 GB)。
可靠性测试表明,测试了Fretha软件在48小时连续运行和高频操作下保持稳定,资源占用无显著波动。

(3)提出了基于明度和均匀度的 ROI 选取算法 LURS,实现了自动FRET双杂交分析。
通过多通道自适应阈值分割和变异系数均匀性评估,实现了 ROI 的自动化选取。
LURS算法通过筛选每个ROI在局部的相对亮度和变异系数,有效排除低亮度背景和灰度突变区域。
在标准质粒C4Y、C10Y、C40Y、C80Y的E-FRET测量和$3^3$-FRET测量中,$E_{A}$和$E_{D}$值与文献报道值的平均相对误差不超过5.1\%。
在C32V和CVC模型质粒的FRET双杂交验证中,自动分析得到的C32V中C与V的结合计量比为1.063,在CVC中则为1.900,符合文献报道的结果。
应用自动FRET双杂交分析方法测得对照组细胞中Bcl-xL-Bak相互作用的化学计量比为1.87,加药A1331852处理组则为1.12,化学计量比测量结果与手动分析高度一致,验证了A1331852药物能够降低Bcl-xL和Bak的结合程度。
基于LURS的自动FRET双杂交分析方法的处理速度比手动分析提高了85.7\%以上,且对比基于SAM-Med2D和ilastik的算法表现出良好的准确性和鲁棒性,更适用于高通量药物筛选等大数据量场景。

\section{展望}
本文研发的FRET双杂交分析数据处理软件Fretha实现了规范化、自动化的FRET双杂交分析数据处理,提升了数据处理的效率和准确性。
Fretha和LURS算法的提出为活细胞FRET双杂交分析技术提供了简单化、标准化和自动化的解决方案,有望推动该技术在精准医疗和药物研发中的大规模应用。
尽管Fretha及LURS算法已实现 FRET 双杂交分析的自动化处理,面对未来日益提高的大数据量数据处理需求,今后的工作可以围绕以下几个方面进行改进和完善:

(1)集成LURS算法到显微镜在线成像系统。
本文提出的LURS算法在性能上已经达到了实时数据处理的实验要求,但是目前还需要将显微镜成像系统中的数据通过数据检验模块导入Fretha中离线处理。
未来可以考虑将LURS算法集成到显微镜成像系统中,实现实时成像和数据处理,研发智能化的在线成像系统。
在线成像系统可以实时监测细胞的荧光信号变化,自动选取ROI并进行数据处理,实时输出FRET双杂交分析结果。
从而加大显微镜通量,进一步提高FRET双杂交实验的效率。

(2)持续完善Fretha功能,继续优化LURS算法同时丰富算法生态,并搭建数据库和云服务平台。
Fretha 目前支持多模态FRET分析方法如E-FRET、$3^3$-FRET、L-FRET、DC-FRET等,未来还可以结合其他细胞图像分析方法如凋亡细胞识别技术、细胞周期统计等,扩展Fretha的功能。
Fretha 当前的分层解耦设计可以方便地引入开源深度学习推理库如 OpenVINO、TensorRT 等,从而支持部署和运行训练好的深度学习模型,以不断适配迅速发展的深度学习算法。
参考ilastik、ImageJ的模块化和插件化设计,Fretha的功能扩展可以以插件形式提供,并支持第三方开发者进行二次开发。
Fretha 目前仅支持本地数据管理,未来可考虑搭建在线数据库和云服务平台,实现数据的在线存储、共享和分析,提高数据的可访问性和可用性。
Fretha的自动数据处理庞大的药物筛选数据也可以通过云服务进行分析,通过平台化的数据管理和分析,为药物研发不断更新和完善数据来源。

(3)应用Fretha进行大规模的生物大分子相互作用探究。
本文在标准质粒完成了对Fretha的验证,并在具体靶点对Bcl-xL和Bak上开展了应用场景的验证。
为进一步验证Fretha的通用性和适用性,还需要进一步在多对靶点、多种细胞系、多种药物处理条件下进行验证。
通过在复杂病理模型中开展大规模的生物大分子相互作用探究,可以为Fretha的应用提供更多的验证数据,更好的发挥其在生物学研究、精准医疗、药物研发等领域的作用。
